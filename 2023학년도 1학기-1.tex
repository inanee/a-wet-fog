\documentclass[article,chapter,openany,amsmath,gremph,lwarp]{oblivoir}

\usepackage{fapapersize}
\usefapapersize{210mm,297mm,10mm,60mm,30mm,32mm}
\usepackage{epsfig}
\usepackage{physics}
\usepackage{lipsum}
\usepackage{marginnote}
\usepackage{caption}
\usepackage{subcaption}
\usepackage{tasks}
                 % Creating Exercises and Solutions
\usepackage{capt-of}   

%%%%%%%%%%%%%%%%%%%%%%%%%%%%%%%%%%%%%%%%%%%%%%%%%%%%%%
%%%%%%%%%%%% 이제 문서를 시작.
%%%%%%%%%%%%%%%%%%%%%%%%%%%%%%%%%%%%%%%%%%%%%%%%%%%%%%
\usepackage[utf8]{inputenc}
\usepackage[usenames,dvipsnames,table]{xcolor}
\usepackage{tikz}
\usepackage[framemethod=TikZ]{mdframed}
\mdfsetup{skipabove=\topskip,skipbelow=\topskip}
\usepackage{amsthm}


\mdfdefinestyle{theoremstyle}{%
linecolor=lightgray,linewidth=2pt,%
frametitlerule=true,%
frametitlebackgroundcolor=gray!20,
innertopmargin=\topskip,
}
\mdtheorem[style=theoremstyle]{defn}{Definition}[section]



\mdfdefinestyle{theoremstyle}{%
linecolor=lightgray,linewidth=2pt,%
frametitlerule=true,%
frametitlebackgroundcolor=gray!20,
innertopmargin=\topskip,
}
\mdtheorem[style=theoremstyle]{task}{Task}[section]

\newtheorem{thm}{Theorem}
\newtheorem{lem}{Lemma}
\newtheorem{prop}{Proposition}
\newtheorem{rem}{Remark}
\newtheorem{exam}{Example}[section]



\ifPDFTeX
	\usepackage{mathpazo}
\else\ifLuaOrXeTeX
	\setmainfont{TeX Gyre Pagella}
 	\setsansfont[Scale=.95]{TeX Gyre Heros}
%% \setkomain/sansfont : see oblivoir-simpledoc.
	\setkomainfont(HCRBatangLVT)(*-Bold)(UnGraphic)
	\setkosansfont[NanumGothic]()[](HCRDotumLVT)
%% 수학 폰트
%	\usepackage{unicode-math}
%	\setmathfont{Asana-Math.otf}
\fi\fi

\ifLuaTeX
\def\interHANGUL{InterHangul}
\else\ifXeTeX
\def\interHANGUL{interhchar}
\fi\fi

%% logos
\usepackage{kotex-logo}
%% index for \koTeX
%\let\ORIGkoTeX \koTeX
%\def\koTeX{\ORIGkoTeX\index{ko.TeX}}

%%% marginfix
%\usepackage{marginfix}

%% 좌우 마진의 marginpar 위치가 혼선을 보인다면
%% 다음을 선언한다.
\strictpagecheck

\usepackage{cite}


%\ifPDFTeX
\usepackage{graphicx}
\DeclareGraphicsExtensions{.pdg,.png,.jpg}
%\fi
%% pdf 정보
\makeatletter
\if@lwarp
\usepackage{xcolor}
\def\epigraph#1#2{#1\quad #2}
\else
\usepackage[dvipsnames]{xcolor}
\hypersetup{%
    bookmarks=true,%
    plainpages=false,%
    colorlinks=true,%
    pdfauthor={Karnes Kim},%
    pdfcreator={Karnes Kim}
}
\fi




\makeatother

\nonfrenchspacing
%% nonfrench를 설정하는 경우에는 \xspaceskip도 정해주자.
%% 여기서는 눈에 띄도록 이 길이를 넉넉하게 잡았음.
%% 일반적으로 .6 내지 .7 정도를 권장함.
\xspaceskip=.8em plus .1em minus .1em

%% 이 아래 명령들은 본문에서 바꿀 수 있다.
%% 행간설정. 두번째 인자는 fn/float에 적용되는 행간.
\SetHangulspace{1.5}{1.1}
%% quotespacing을 설정함. \noadjustqutespacing이 default. \adjustquotespacing
%\adjustquotespacing
%% float/fn의 좁은 행간 설정을 disable. \adjustfloatfnspacing
\noadjustfloatfnspacing

%% snugshade 환경에 칠할 배경색.
\definecolor{shadecolor}{gray}{0.85}


%%% from memman.tex. modified.
%% 본문에서 한 번 사용하기 위해 설정한 chapter
%% style 예제. 이 예제는 memman에 있다.
%% 한글판을 위하여 조금 수정함.
\makeatletter
\newlength{\numberheight}
\newlength{\barlength}
\makechapterstyle{veelo}{%
   \setlength{\beforechapskip}{40pt}
   \setlength{\midchapskip}{25pt}
   \setlength{\afterchapskip}{40pt}
   \renewcommand{\chapnamefont}{\normalfont\LARGE\flushright}
   \renewcommand{\chapnumfont}{\normalfont\HUGE}
   \renewcommand{\chaptitlefont}{\normalfont\HUGE\bfseries\flushright}
   \renewcommand{\printchaptername}{}%
   \renewcommand{\prechapternum}{% <= 이 명령을 정의. 
            \chapnamefont\MakeUppercase{chapter}}
   \renewcommand{\postchapternum}{}% <= 이 명령을 정의. 여기서는 비움.
   \renewcommand{\chapternamenum}{}
   \setlength{\numberheight}{18mm}
   \setlength{\barlength}{\paperwidth}
   \addtolength{\barlength}{-\textwidth}
   \addtolength{\barlength}{-\spinemargin}
   \renewcommand{\printchapternum}{%
      \makebox[0pt][l]{% 
      \hspace{.8em}%
      \resizebox{!}{\numberheight}{\chapnumfont \thechapter}% 
      \hspace{.8em}%
      \rule{\barlength}{\numberheight}
     }
   }
   \makeoddfoot{plain}{}{}{\thepage}
}

\makechapterstyle{mycompanion}{%
  \chapterstyle{default}
  \renewcommand*{\chapnamefont}{\normalfont\LARGE\scshape}
  \renewcommand*{\printchaptername}{\raggedleft\chapnamefont \@chapapp}
  \renewcommand*{\prechapternum}{\raggedleft\chapnamefont \pre@chapter}
  \renewcommand*{\chapnumfont}{\normalfont\Huge}
  \setlength{\chapindent}{\marginparsep}
  \renewcommand*{\printchaptertitle}[1]{%
    \begin{adjustwidth}{}{-\chapindent}
      \raggedleft \chaptitlefont ##1\par\nobreak
    \end{adjustwidth}}} 

\makeatother 

%%% User's commands
%% 사용자 명령들. 인덱스 관련 명령.
\newcommand\dispcmd[1]{%
    \texttt{\textbackslash #1}%
    \index{명령!\textbackslash #1}%
    \index{#1@\textbackslash #1}%
}

\newcommand\cls[1]{%
    \texttt{#1}\ 클래스%
    \index{클래스!#1}%
    \index{#1~클래스}%
}

\newcommand\pkg[1]{%
    \texttt{#1}\ 패키지%
    \index{패키지!#1}%
    \index{#1~패키지}%
}

\newcommand\thisclass{%
    \texttt{memoir}\ 클래스%
    \index{클래스!memoir}%
    \index{memoir~클래스}%
}

\newcommand\env[1]{%
    \textsf{#1} 환경%
    \index{환경!#1}%
    \index{#1~환경}%
}

\newcommand\wi[2][\empty]{%
    \ifx#1\empty
        \index{#2}#2%
    \else
        \index{#1!#2}#2%
    \fi
}





%% showcommand/showenv 명령.
%% 만약 command/env 보여주기를 끄려면
%% \showcommandfalse를 선언한다.
\makeatletter
\newif\if@showcommand\@showcommandtrue
\newcommand\showcommandtrue{\@showcommandtrue}
\newcommand\showcommandfalse{\@showcommandfalse}

\strictpagechecktrue

\newcommand\showcommand[1]{%
	\if@showcommand
	 \checkoddpage\ifoddpage
      \marginpar{\small\texttt{\textbackslash #1}}%
   \else
      \marginpar{\hfill\small\texttt{\textbackslash #1}}%
   \fi
	\fi
}

\newcommand\showenv[1]{%
	\if@showcommand
	 \checkoddpage\ifoddpage
    \marginpar{\small\textit{#1}(env.)}%
   \else
    \marginpar{\hfill\small\textit{#1}(env.)}%
   \fi
	\fi
}
\makeatother

%% MakeShortVerb
\MakeShortVerb{\|} 
% \DeleteShortVerb{\|}

%% 인덱스의 hyperpage 처리를 위해서
\newcommand\bfhypidx[1]{%
	\textbf{\hyperpage{#1}}%
}

%% replace \bigskip
\newcommand\alineskip{%
	\vspace{\onelineskip}%
}

%% showindexmark
%% 여백을 충분히 확보하고 다음 행을
%% 활성화해볼 것.
%\showindexmarktrue

%% index를 만든다. 
\makeindex

%% newlist
%% 새로운 리스트를 만드는 것이 정말 너무나 간단하다.
\newcommand\queryfont{\raggedleft\sffamily\small}
\newcommand\listofqueriesname{Queries~목록}
\newlistof{listofqueries}{loq}{\listofqueriesname}
\newlistentry{query}{loq}{0}
\newcommand{\query}[2][\empty]{%
	\refstepcounter{query}
	\par\noindent\fbox{Q?}~\begingroup\queryfont #2\endgroup%
	\addcontentsline{loq}{query}{\protect\numberline{\thequery}#2}\par
	\ifx#1\empty\else\index{Query!#1}\fi
}

%% title page
\newcommand\MakeTitle{%
 \begin{titlingpage}
 \setcounter{page}{-1}%
 \newlength\tmplen\setlength\tmplen{\textwidth}\addtolength\tmplen{40mm}
   \begin{minipage}{\tmplen}
   \begin{center}
     \LARGE\bfseries\thetitle \\ \vskip\onelineskip
     \normalfont\normalsize\theauthor
   \end{center}
   \vspace*{100mm}
 \end{minipage}
  \end{titlingpage}
}

%% 각주. footmisc는 memoir와 함께 쓸 수 있다.
%% 참고. [bottom] footnote를 위해서라면 
%% 다음과 같이 할 것을 권장함. raggedbottom에서 동작함.
%\renewcommand*{\footnoterule}{\kern-3pt\vfill
%    \hrule width 0.4\columnwidth \kern 2.6pt}
%% 각주를 floats 아래 두는 memoir 명령.
\feetbelowfloat

%% 밑줄.
\ifLuaTeX\else
\usepackage[normalem]{ulem}
\fi

%% 사소한 설정
\def\util#1{\texttt{#1}\index{Utility!#1}}




\newenvironment{exper}[1]
  {\mdfsetup{
    roundcorner=5pt,
    frametitle={\colorbox{white}{ \space$\langle\langle$#1$\rangle\rangle$ \space}},
    innertopmargin=10pt,
    frametitleaboveskip=-\ht\strutbox,
    frametitlealignment=\center
    }
  \begin{mdframed}%
  }
  {\end{mdframed}}

%%%
%\headnamereftrue

%%%%%%%%%%%%%%%%%%%%%%%%%%%%%%%%%%%%%%%%%%%%%%%%%%%%%%
%%%%%%%%%%%% 이제 문서를 시작.
%%%%%%%%%%%%%%%%%%%%%%%%%%%%%%%%%%%%%%%%%%%%%%%%%%%%%%
\begin{document}

\title{{물안개(物眼開)}}
\author{조인환 T}


%% 아래 \MakeTitle 명령은 이 문서의 preamble에
%% 정의되어 있다. 표지 디자인은 이 정의를 참고하여
%% 수정하여 볼 것.
\MakeTitle
\cleardoublepage

\frontmatter
% ToC, etc
%% 차례 페이지들. 여기서는 demo chapter style과
%% Ruled 페이지 스타일을 적용한다.
\chapterstyle{article}

\tableofcontents
%\clearpage
\listoffigures
%\clearpage
\listoftables
%% listofqueries는 \newlistof로 만든
%% local \listofx임.
%\clearpage
\listofqueries

\mainmatter

%% hangul 페이지 스타일을 테스트하기
%% 위하여 제1편 앞에 한 챕터를 둠. 다만 chapter의 첫 면은
%% chapter 페이지 스타일을 따른다.
\chapterstyle{default}

\copypagestyle{part}{empty}
%%%%%%%%%% Ruled를 사용하되 partmark를 정의함.
\copypagestyle{MyRuled}{Ruled}
\newlength{\MyRuledheadwidth}
\setlength{\MyRuledheadwidth}{\textwidth}
\addtolength{\MyRuledheadwidth}{\marginparsep}
\addtolength{\MyRuledheadwidth}{\marginparwidth}
\makerunningwidth{MyRuled}{\MyRuledheadwidth}
\makeheadrule{MyRuled}{\MyRuledheadwidth}{\normalrulethickness}
\makeheadposition{MyRuled}{flushright}{flushleft}{flushright}{flushleft}
\makeatletter
\makepsmarks{MyRuled}{%
  \let\@mkboth\markboth
  \def\partmark##1{\markboth{\hparttitlehead. ##1}{##1}}
  \def\chaptermark##1{\markright{\hchaptertitlehead. ##1}}
  \def\sectionmark##1{}
}
\def\partmark#1{\markboth{\hparttitlehead. #1}{#1}}
\makeatother
\oblivoirchapterstyle{veelo}
\pagestyle{MyRuled}


\part{뉴턴의 역학법칙}
\pagestyle{MyRuled}

\chapter{기본 물리량 \\(Fundamental Physical quantity)}

\section{Orientation} 
\begin{flushleft}


이 문서는 충남과학고등학교 1학년 학생들의 물리학 수업을 위하여 제작되었다.

충남과학고의 2023학년도 신입생의 경우 물리학1의 내용을 1학기 동안 배우고 이후 2학기부터 고급물리학과 일반물리학, 물리학 실험을 
수강하게 된다. 

이 교수학습자료의 필자는 조인환으로 충남 대천출신이다. 공차는 것을 좋아하고 학생들을 가르치고 있다. 새로운 것에 도전하는 것에 흥미를 느끼고 
많은 사랑을 할 수 있는 사람이 되는 것이 비전이다. \footnote{서로 자기소개하는 시간을 갖자.}\\
이 자료는 다양한 대학 전공 서적을 인용하고 그 가르침의 깊이가 부끄러울 정도로 짧아 자료의 활용은 충남과학고등학교 학생들에게까지만 제공하고자 한다.
\\
\marginpar{
  \begin{center}
  \includegraphics[width=4cm]{사진모음/언어.jpg}\captionof{figure}{언어}
  \end{center}} 

필자가 학생들에게 해주고 싶은 말이 하나 있다. 모든 학문은 그 논리를 펴내가는 언어와 같다는 것이다. 
따라서 물리학을 배울 때에도 물리학이라는 나라에서 통용되는 언어를 잘 익히고 이를 이용해 소통하고 의미있는 작품들을 만들어가는 것이다. 
소통이 명확하려면 첫째로는 그 언어의 구성요소를 명확히 알아야 한다. 고등학교 수준의 물리학은 그 언어를 암기하고 배워나가는 과정에
가깝다. 더 나아가서는 이를 이용해 간단한 대화(문항을 푸는 것)들을 익히게 되는 것이다. 
이런것들이 차츰 쌓이면 본인만의 다양한 이야기들을 집필할 수 있을것이다. 
따라서 우리 학생들은 이전에 알고 있던 경험의 언어들과 물리학의 언어를 \textbf{혼동하고 혼용하지 말고 정확하고 격식있게} 대화하기 위해
물리학 공부를 엄밀히 해주기를 바란다.  
물리학이 다른 언어들과 다른 점은 다양한 자연현상을 공통의 이유로 축약하여 설명하려 한다는 것이다. 
자연에 있는 만물의 이치를 알기 위해 우리의 머릿속으로 다양한 현상을 갖고와 물리학 언어의 규칙 속에서
설명하는 것이 우리가 하는 물리학이 되겠다. 따라서 우리 학생들은 다양한 주변의 현상들을 있는 그대로 받아들이는 한편 그 현상들의 
불변하는 규칙을 찾으려고 노력한다면 그것이 물리학을 하고 있는것이라고 생각한다.


이 문서는 물리학 수업을 처음듣는 신입생 학생들이 올바르게 물리학을 접하기 위하여 오랜기간 제작되었으므로 이를 정독하고 
궁금한 것들을 인터넷 검색과 교사 질의를 통해 자율적으로 학습하기를 바란다.


이 문서에서 다루는 몇가지 구성의 규칙, 특성을 설명하겠다.
\begin{mdframed}\tightlists
\begin{itemize}[-]
 \item 각 \textbf{Chapter}는 물리학의 큰 영역과 범위를 같이 한다.(고전역학,열역학, 상대론, 전자기학, 광학, 파동, 통계물리, 현대물리, 양자 등)
 \item \textbf{강조하고자 하는 어구는 굵은 글씨로 나타내며}, 이러한 어구는 정확한 뜻을 이해하여야 연속적인 학습을 할 수 있다.
 \item 그 섹션에서 꼭 알고 넘어가야 하는 물리학의 언어들은 \textbf{Definition}으로 두었다. 
 \item 생소하거나 설명이 적은 부분이 있다. 각주가 붙은 꼬리말들을 이용하거나 부록에 설명이 있는 경우가 있다. 그럼에도 부족한 물음은 다양한 방법을 이용해 해결해보자. 
 \item \textbf{Task}는 설명으로만 듣기에는 시원하지 않은 내용들의 예제를 두었다. 잘 풀어보며 비슷한 문항을 만들어보는 등 시험을 대비하여보자.
 \item 물리학 용어는 다수 영어로 번역하여 가로()에 두었다. 다양한 논문과 원서의 내용을 이해하기 위해 용어들을 잘 알아두자. 각 용어를 아는 것은 각 식의 문자들을 
 암기하는데에도 도움이된다.  
\end{itemize}
\end{mdframed}



\section{어떤 물리현상을 다루는가}

과학과 공학은 측정과 Data의 정량적인 비교에 바탕을 두고 있다. 따라서 이들을 다루는 규칙과 단위를 명확히 아는 것은 매우 중요하다. 
물리학은 연역적\footnote{연역적 추론은 논리학 용어로, 이미 알고 있는 판단을 근거로 새로운 판단을 유도하는 추론이다. 또한 이것은 다른 결론을 배제한다. 반드시 
한가지 결론이 나야 하는 것으로 A이기 위해서는 B여야 한다고 추론한다.}인 사고로 확장한다. 따라서 가장 재료가 되는 기본 물리량들에 대해서 명확히 알고 이것들을 조합하여
새로운 물리량들을 정의하는 것에 익숙해져보자.

물리학에 있어 물리량의 측정이란 매우 중요하고 심사숙고해야 하는 과정이다. 
역사적 과정을 통해 정의되어있는 물리량들의 국제단위계를 알아두고 
극정밀도를 요구하는 현대사회에서 가장 기본적인 \textbf{시간, 길이, 질량}이 어떻게 정의되었는지 확인해보자. 

\subsection{길이 (length)}
모든 물리량은 표준과 비교하여 각각 고유한 단위로 측정하게 된다. 예컨데 미터(m)는 길이라는 물리량의 단위이며, 표준은 그것의 1.0을 의미한다. 
정확한 1.0m의 표준을 정하는 것은 매우 신중해야한다. 왜냐하면 \textbf{수소원자의 길이부터 천체사이의 거리를 정확하게 측정}해주어야 하기 때문이다. 
1792년 프랑스 공화국이 도량형법을 제정할 때에는 북극점에서 적도까지 거리의 1천만 분의 1로 정의하였지만 실요상의 이유로 폐기하고 백금-이리듐 합금막대에
새겨진 가는 두 눈금 사이의 길이로 1미터를 정의하였다. 그러나 현대의 과학기술은 미터원기에 새겨져 있는 두 눈금 사이의 길이보다 
더 정밀한 표준이 필요하게 되었다. 1983년에 미터는 특정한 시간간격 동안 빛이 진행한 길이로서 제 17차 도량형 총회에서 다음과 같이 정의하였다. 



  \begin{defn}[ 1m ]
    미터는 1초의 1/299,792,458인 시간간격 동안 빛이 진공 속에서 진행한 경로의 길이이다. 
  \end{defn}



\subsection{질량 (mass)}
질량의 SI 표준은 파리 근교의 국제도량형국에 보관되어 있는 질량 1kg으로 정한 백금-이리듐 원통이다. 이러한 표준원기는 다른 표준원기를
복제 및 비교하기 위해 움직이고 그 사용이 매우 제한적이다. 이러한 원기가 있더라도 원자의 질량은 원자들 상호간에 비교하는 것이 훨씬 정확하기에 
탄소-12 원자의 질량을 국제 협약에 따라 \textbf{12 원자질량단위(u)}의 질량으로 정의하였다. 이들 두 표준 사이의 관계는 


  \begin{equation}
    1u=1.66053886\times 10^{-27}kg
  \end{equation}

이며, 이를 이용하여 다른 원자들의 상대적인 질량을 실험적으로 상당히 정밀하게 결정할 수 있다. 현재는 물리상수인 플랑크 상수 값($h$)을 기준으로 
질량을 정의한다. 
\marginpar{
  \begin{center}
  \includegraphics[width=4cm]{표준원기.jpg}\captionof{figure}{표준원기}
  \end{center}} 


\subsection{시간 time}
통상적으로 시간이라하면 두가지의 관점이 있다. \textbf{사건이 언제 일어났는가?}\footnote[1]{시간의 어느 한 시점을 뜻하는 말로 "시각"이 있다.}와\textbf{ 사건이 얼마나 지속되었는가?}이다. 
이러한 경우를 정의하기 위해 스스로 되풀이되는 것들을 시간의 표준으로 사용하게 되었다. 이전에는 지구의 자전을 기준으로 맞추어 시간간격을 
측정하였다. 하지만 극정밀도를 얻기 위해 이후에는 원자시계가 계발되었다. 1967년 개최된 제13차 도량형총회에서는 세슘시계에 기초를 둔 표준 
초를 다음과 같이 채택하였다.

       \begin{defn}[1초]
       1초는 세슘-133 원자에서 방출된 특정한 파장의 빛이 \\9,192,631,770번 진동하는데 걸리는 시간이다.
       \end{defn}

\end{flushleft}

%% 본문시작.
%% 본문은 companion chapter style로 식자할 것인데,
%% default를 한 번 부른 이유는 앞서 사용한 demo초
%% 글꼴 설정에 영향을 받기 때문이다.

%% 제목은 두 줄로 식자하되, heading과 toc에는 개행 없이.
% \chapter[memoir 클래스에서 한글을 쓰자][memoir 클래스에서...]{memoir
%   클래스에서\\ 한글을 쓰자} % -> [toc][heading]{title}
% \chapter[memoir 클래스에서 한글을 쓰자]{memoir 클래스에서\\ 한글을
%   쓰자} % -> [toc,heading]{title} %% 이 부분이 memoir와 다르다.
% \chapter{memoir 클래스에서 한글을 쓰자} % {toc,heading,title}
% \chapter[memoir 클래스에서 한글을 쓰자][memoir 클래스에서
% 한글...]{memoir 클래스에서\\ 한글을 쓰자}
% \chapter[memoir 클래스에서 한글...]{memoir 클래스에서\\ 한글을 쓰자}
% 주의: titleref에 \\ 문자가 들어가면 안된다. titleref에서 사용하는
% 것은 두번째 옵션인자이므로, 다음과 같이 두번째 옵션 인자를 밝혀줄
% 것.

\chapter{뉴턴의 3가지 역학법칙}
\chapterprecis{\noindent 이 장에서는 뉴턴의 역학법칙을 다양한 관점과 도구를 통해 이해한다.}


이제 본격적으로 뉴턴의 운동법칙을 배우고자 한다. 뉴턴의 운동법칙은 기계적인 결정론과 연관된다. 
프랑스의 수학자 라플라스의 에세이에선 "우주에 있는 모든 원자의 정확한 위치와 운동량을 알고 있는 존재가 있다면, 이것은 뉴턴의 운동법칙을 이용해 과거, 
현재의 모든 현상을 설명해주고, 미래까지 예언할 수 있다"고 서술하였다.
그 말의 의미를 곱씹다보면 우리가 배울 뉴턴의 운동법칙이 와닿을 것이다.
본격적으로 뉴턴의 운동법칙을 학습하기에 앞서 2가지를 학습하자 \\
\marginpar{
  \begin{center}
    \includegraphics[width=4cm]{laplace_jpg.jpg}
    \captionof{figure}{Pierre Simon Laplace\\(1749.3.23.-1827.3.5.)}
  \end{center}} 

\noindent
첫째로는 벡터량이다. 이를 위해 한번쯤은 듣거나 책에서 봤을 아래의 수식을 잘 들여다보자. 
    \begin{equation}\label{eq:newton's law 1}
      \va{F}=m\va{a} 
    \end{equation}
이 짧은 수식을 이해하는 것은 뉴턴 역학을 절반 이상을 이해하는 것인만큼 함축하는 것이 아주 많은 식이다. 
잘 보면 질량을 뜯하는 문자에는 화살표가 붙지 않고 힘과 가속도 부분의 문자는 화살표가 붙는다.
이 차이는 \textbf{질량은 양만 있는 스칼라량, 힘과 가속도는 양과 방향이 있는 벡터량}이라는 차이가 있다. 
벡터는 스칼라보다 다루기가 매우 까다롭다. 벡터에 대해서 부록을 통해 학습해보자.


두번째는 다른것으로 대체 정의할 수 없는 물리량인 시간 $t$, 길이 $l$, 질량 $m$을 이용하여 속도와 가속도를 정의하여 보자. 
이 때 변위 벡터 $\vec{s}$는 움직이는 물체의 나중 위치벡터 $\va{x}$와 처음 위치벡터 $\va{x}_0$의 차이다. 
즉, $\vec{s} = \vec{x}-\vec{x_0}$이다. 
위치는 기준점에 따라 달라진다. 
하지만 변위벡터는 기준점과 관계없이 변화량만 다루기 때문에 기준이 필요치 않아 유용하다. 
이러한 관계는 시각과 시간의 차이와 유사하다.

\marginpar{
  \begin{center}
    \includegraphics[width=5.5cm]{사진모음/벡터의 표현.png}
    \captionof{figure}{위치벡터와 변위벡터}
  \end{center}} 



  \begin{defn}[속도와 가속도]
\begin{equation}\label{def:acceleration}
  \va{v} = \frac{\va{x_2}-\va{x_1}}{t_2-t_1} \Rightarrow \lim_{\Delta{t}\to 0}\frac{\Delta{\va{x}}}{\Delta{t}}=\dv{\va{x}}{t}
\end{equation}
\begin{equation}
  \va{a} = \frac{\va{v_2}-\va{v_1}}{t_2-t_1} \Rightarrow \lim_{\Delta{t}\to 0}\frac{\Delta{\va{v}}}{\Delta{t}}=\dv{\va{v}}{t}
\end{equation}

이 때 시간간격을 무한소($\Delta t \rightarrow 0$)로 취했다. 
만약 무한소가 아니라면 그 시간간격 동안의 \textbf{평균 속도($\vec{v}_{avg}$)와, 평균 가속도($\vec{a}_{avg}$)}라고 하고, 
위의 정의와 같이 무한소를 취한 경우는 \textbf{순간 속도, 순간 가속도}라고 한다.\footnote{Eq. \ref{def:acceleration}는 시간에 대한 변위 함수의 미분으로
볼 수 있다. 미분의 방법에 대하여 부록에서 공부하고 오자.}
  \end{defn}






이 장을 읽은 후에는 다음을 이해하길 기대한다.



\tightlists
\begin{itemize}
\item 힘이 벡터임을 알고 벡터의 특징, 벡터합과 곱을 할 수 있다.\footnote{부록의 "vector"으로 이동하여 학습하고 오자.}
\item 갈릴레이의 사고실험을 접하고, 뉴턴의 제1 운동법칙을 이해한다.
\item 벡터를 이용해 한 물체에 작용하는 힘(외부력)의 자유물체도를 그릴 수 있다. 
\item 실험을 통해 질량의 개념을 받아들이고 운동량 개념을 접한다.
\item 운동량의 시간변화량과 힘의 관계를 받아들이고 운동량 보존 법칙\\을 이해한다.
\item 힘과 운동량 관계를 통해 뉴턴 제2 운동법칙을 설명할 수 있다.
\item 뉴턴의 제3 운동법칙을 이해한다.
\item 간단하고 다양한 물리상황에 뉴턴의 역학법칙을 예외없이 적용시킬수 있다.
\end{itemize}

\section{뉴턴의 운동법칙과 $\va{F}=m\va{a}$}
뉴턴의 운동법칙은 원자크기부터 거대한 천체운동까지 적용할 수 있는 매우 유용한 도구이다. \footnote{광속에 가까운 상황(상대성이론)
과 원자 단위의 매우 작은 미시세계(양자역학)에 대해서 설명할 때는 다른 설명체계를 가져와야 한다.} 실제 생활간에는 경험에 따른
선지식이 생기고 그것에는 많은 오개념들이 있다. 예를 들면 우리는 발로찬 축구공이 아무런 힘을 주지 않더라도 정지한다고 경험한다. 
하지만 이것은 뉴턴 역학에 맞지 않는것이다.
이러한 생각들을 지혜롭게 해석하고 풀어간 앞선 현인들의 생각을 통해 자연스럽게 뉴턴역학을 습득하여보자.  

\subsection{갈릴레이의 사고실험과 관성(뉴턴의 제1 운동법칙)}
물리학에서 유용하게 사용되는 사고실험을 도입해보자. 사고실험은 실제의 자연현상에는 고려할 것이 너무 많아 
머릿속에서만 조건을 단순하게 가정하고 이론을 바탕으로 일어날 현상을 예측해보는 실험으로 물리학에서 유용하다. 
뉴턴이전에는 갈릴레이라는 천재가 있었다. 그는 매끄러운 경사면 위에 공을 굴리면 반대쪽 경사면의 같은 높이까지 공이 올라간다는 
점을 관찰하였다. 정확히는 같지 않았지만 사면을 매끄럽게 하고 공을 사용하여 마찰력을 최대한 줄이면 거의 같은 높이까지 
공이 올라간다. 이 때 사면을 점차 펼친다면 공은 영원히 굴러갈 것이다. 그는 이것이 자연스러운 운동이라고 생각했다. 
\marginpar{
  \begin{center}
  \includegraphics[width=4cm]{갈릴레이.jpg}\captionof{figure}{Galilei,Galileo(1564-1642)}
  \end{center}} 

\begin{figure}[h]
  \centering
   \includegraphics[width=5cm]{갈릴레이의 사고실험.jpg}
   \caption{갈릴레이의 사고실험}\label{fig:Galilei's thought}
 \end{figure}
그는 사고실험을 통해 즉, 머릿속의 실험을 통해 \textbf{물체가 힘을 받지 않으면 계속 원래 운동상태를 유지}한다고 증명하여 
당시의 힘을 받지않을 때 물체는 자연스럽게 정지한다는 자연관을 반박하고 \textbf{관성}의 개념을 발견했다. 


\subsection{힘 (Force)}\label{sec:힘}
\begin{flushleft}
  
필자가 강조하는 $\va{F}=m\va{a}$ 를 다시 들여다보자. 이제, 좌변의 F를 들여다보자. 
이것은 힘을 나타내며 생긴모양으로 벡터임을 알 수 있다. 이에 힘은 벡터의 성질을 따른다. 
우변에 있는 가속도 $\va{a}$는 속도의 변화를 나타내며 Eq. \ref{def:acceleration}에서 정의하였다.

따라서 $\va{F}=m\va{a}$로 알고 있던 뉴턴의 운동법칙은 \textbf{힘이란 관성을 갖는 질량의 운동상태를 바꾸어주는
(가속도를 만드는) "원인"}으로 해석이 가능하다. 이 때 운동
상태는 속도를 뜻하고 속도의 변화인 가속도는 운동상태의 바뀜으로 해석한다. 
더 정확하게는 \textbf{1kg으로 정의된 물체에 힘을 작용시켜 물체의 가속도로 $1m/s^2$을 얻을 때에 물체에 작용한 힘이 
1N의 크기를 갖는다.}                                                              

좌변의 F에 들어갈 수 있는 원인이 되는 힘은 가볍게는 "밀고 당기는 힘"으로 알고 있을것이다. 하지만 이러한 힘들은 기본적인 4개의 
힘의 다른표현에 불과하다. 기본적인 힘은 4가지 종류로 \textbf{중력, 전자기력, 약력, 강력}이 있다. 

\begin{figure}[h]
  \centering
   \includegraphics[width=8cm]{funddamental interaction.jpg}
   \caption{기본적인 상호작용}
 \end{figure}


이 이외에 들어보았던 마찰력, 장력(실이 당기는 힘), 부력, 자기력 등은 모두 이 4가지 힘의 서로 다른 표현이며 특히, 대부분은 전자기력의 표현이다. 
약력과 강력은 원자 크기에서 작용하여 일반물리학 수준에서는 다루지 않는다. \footnote{주의할 점으로 이후에 배울 구심력과 원심력은 
"원인"에 해당하는 힘이 아니다. 즉 좌변에 원심력과 구심력이라는 원인은 고려하면 안된다.}


모든 힘, 운동상태의 변화는 본질적으로 물체쌍 사이의 상호작용에 의한 것이다.

\marginpar{
  \begin{center}
  \includegraphics[width=4.0cm]{사진모음/손흥민 슛.jpg}\captionof{figure}{힘이 주어지는 사건은 혼자 일어나지 않는다.}
  \end{center}} 
예를 들어 손흥민 선수가 슛을 하여 축구공의 운동에 가속도가 있는 상황을 보자. 
공의 입장에서 \textbf{손흥민의 발이 공을 찼다.} 이 때에는 힘을 준 손흥민의 발이 있고
힘을 받은 공이 있다. 이렇듯 \textbf{힘은 단일로 일어나지 않으며 힘을 준 주어와 받는 목적어가 필연적이다.} 
이러한 맥락에서 우주의 4가지 기본적인 힘들은 모두 상호작용(interaction)이라고 한다. 
\end{flushleft}





\subsection{자유물체도 Free Diagram}
이제 힘에 대해서 배웠으니 실제 실생활의 문제들을 분석하기 위해 힘을 표현하는 "자유물체도"를 배우고 적용해보자. 
\textbf{자유물체도는 물체(계)에 작용하는 모든 "힘(외부력)"을 벡터 화살표로 표시한다.} \footnote{여기에서
벡터의 꼬리는 물체의 중심 또는 무게 중심에 두는것으로 약속한다. 물체의 병진운동의 경우 물체를 점질량으로 근사하여 하나의 점 또는 
그림의 편의상 크기를 가지는 점으로 표시한다.}
이것은 관심있는 물체(계)에만 집중하기 위해 고안되었다. 자유물체도를 명확히 이해하기 위해서는 "외부력\footnote{이후의 외부력을 
외력이라고 줄여서 언급하겠다.} (external force)"과 "계 (system)"를 정의해야 한다. 
\textbf{계는 한 개 이상의 물체로 구성되어 있으며 계의 외부에서 계에 작용하는 힘을 외부력이라고 한다.} 
만약 물체들이 서로 단단하게 붙어있으면 이 계를 하나의 물체로 간주하며 이 계에 작용하는 힘을 외부력이라고 한다. 
중요한 것은 물체 내부에서 상호작용하는 내부력은 고려하지 않는다. \footnote{"조별 자유물체도 그리기" 활동을 수행하여보자.}

이 때 물체에 작용하는 모든 외부력들을 표현하기에 외부력의 벡터합을 \textbf{알짜힘}(net force)이라고 하며 
$\vec{F}_{net}$으로 표현한다. 그리고 이 때 대상이되는 물체는 계라고 하며 이 계의 총질량이 m이 된다.
따라서 뉴턴의 역학법칙식을 다음과 같이 확장하여 쓸 수 있다.          

    \begin{defn}[알짜힘과 뉴턴의 운동 제2법칙]
    \begin{equation}\label{eq:newton's law 2}
      \sum\vec{F}_{ext}=\vec{F}_{net}=m\va{a} 
    \end{equation}
  \end{defn}

  \clearpage
  \begin{task}[줄의 장력, 수직항력]
 \begin{flushleft}
  {\IfFileExists{사진모음/연습하기1.PNG}
  {\includegraphics[width=0.8\linewidth]{사진모음/연습하기1.PNG}
 \label{fig:연습1}}%
  {\rule{\linewidth}{4cm}}}
 \end{flushleft}
\end{task}

\begin{task}[object on a horizontal object]
  \begin{flushleft}
    {\IfFileExists{사진모음/연습5.png}
    {\includegraphics[width=0.6\linewidth]{사진모음/연습5.png}
  \label{fig:물체 위의 물체}}%
    {\rule{\linewidth}{4cm}}}
   \end{flushleft}
\end{task}

\section{뉴턴의 운동법칙 적용}

우리 주변에 있는 상황을 분석하며 뉴턴의 역학법칙을 적용해보고 거대한 통찰력을 느껴보자.

\subsection{지표면에서의 자유낙하운동과 만유인력의 법칙}\label{sec:gravity}
간단한 상황인 "자유낙하" 상황을 고려해보자. section \ref{sec:힘}에서 배웠듯이 힘은 4가지 상호작용만 있고 
우리가 다루는 대부분의 경우 중력과 전자기력 두 가지 경우에서 분석이 끝난다. 
이 경우에는 커다란 지구가 당기는 만유인력 즉, 중력만이 작용하는 경우라고 할 수 있다. 

행성들의 움직임을 설명하고자 뉴턴은 질량을 가진 물체가 서로 당기는 힘인 만유인력을 다음과 같이 정의하였다. 

\marginpar{
  \begin{center}
  \includegraphics[width=4cm]{Isac Newton.jpg}\captionof{figure}{Isac Newton(1642-1727)}
  \end{center}} 
  

    \begin{defn}[뉴턴의 만유인력의 법칙]

       \centering{\IfFileExists{law of universal gravity.png}
  {\includegraphics[width=0.4\linewidth]{law of universal gravity.png}
  \captionof{figure}{만유인력}\label{fig:law of universal gravity}}%
  {\rule{\linewidth}{4cm}}}



    \begin{equation}\label{eq:universal gravity}
     \overrightarrow{F}_{grv}= \overrightarrow{F_{1}}= -\overrightarrow{F_{2}}
      =-\frac{Gm_1m_2}{r^2}\vu{r}  \footnote{단위벡터 $\vu{r}$에 대해서는 부록의 극좌표계를 보고오자.}
    \end{equation}

  \begin{itemize}
    \item  $\overrightarrow{F}_{grv}$: 두 질량간의 힘
    \item G: 중력상수($=6.67\cross 10^{-11} N m^2/kg^2$)
    \item $m_1$: 첫번째 물체의 질량
    \item $m_2$: 두번째 물체의 질량
    \item  r= 두 점질량간의 거리 
    \item  $\vu{r}$ = 두 물체의 중심을 이은 위치벡터로 서로 당기는 방향이다. 
  \end{itemize}
  
  \end{defn}





  자 이제 토마토를 초기속력 ($v_0$)으로 연직 위로 던졌다가 토마토를 던진 위치에서 
받는 경우에 토마토에 작용하는 힘을 자유물체도로 표현해보자.
\clearpage

\centering{
\begin{figure}[h]
  \centering
   \includegraphics[width=3cm]{자유낙하 토마토.png}
   \caption{자유낙하하는 물체의 free-diagram}\label{fig:free-diagram(tomato)}
 \end{figure}}
 
 \begin{flushleft}
 자유낙하하는 토마토는 만유인력을 나타내는 Eq. \ref{eq:universal gravity}와 뉴턴의 운동 법칙에 따라 힘을 받는다.
    \begin{equation}
        -\frac{GM_{E}m_{t}}{R^2}\vu{r}=m_{t}\va{a}
    \end{equation}
\marginpar{
  \begin{center}
  \includegraphics[width=4cm]{사진모음/지구반지름.jpg}\captionof{figure}{지구반지름}
  \end{center}} 
  

    
    이 때에 두 질량간의 거리(R)는 지구 중심과 토마토간의 거리이다. 토마토가 3m 정도에서 떨어진다고하면 이것은 지구의 반지름($R_E=$6400km)보다 훨씬 작다. 그래서
우리가 관심있는 지표면에서의 자유낙하 상황에서는 힘을 일정히 볼 수 있다.
    \begin{equation}
        -\frac{GM_{E}m_{t}}{R^2}\vu{r}\approx {-\frac{GM_{E}m_{t}}{R_{E}^2}}\vu{r} 
    \end{equation}

    \begin{equation}
      m_{t}({-\frac{GM_{E}}{R_{E}^2}}) =m_{t}\va{a} =-m_{t}g \vu{j}
    \end{equation}
     이러한 경우 뉴턴의 제2 운동법칙에 따라 
 가속도가 일정하고 이러한 상황을 특별한 경우로 \textbf{등가속도 운동}이라고 한다.
    이 때에 자유낙하하는 물체의 가속도 g($9.8m/s^2$)를 중력가속도라고 한다.
    
    \subsection{등가속도 운동}
    등가속도 운동의 경우를 분석하여 보자. 이것은 속도와 가속도에 대해 정의한 Eq.\ref{def:acceleration}로 부터 시작한다. 
    가속도가 일정할 때는 순간가속도와 평균가속도가 같으며 이를 다음과 같이 쓸 수 있다. 
    
    \begin{equation}
      a =\frac{v-v_0}{t-0}
    \end{equation}
      여기서 $v_0$는 시간 t=0에서의 속도이고 $v$는 이후의 시간 t에서의 속도이다. 이것은 다음과 같이 표기할 수 있다.
    \begin{equation}
      v=v_0+at\label{eq:등가속도 운동1}
    \end{equation}
같은 맥락으로 다음의 식을 유도할 수 있다. 
\begin{align}
      &v_{avg} =\frac{x-x_0}{t-0}\\
      &x=x_0+v_{avg}t\\
      &v_{avg}=\frac{1}{2}(v_0+v)\\
      &x-x_0=v_0t+\frac{1}{2}at^2\label{eq:등가속도 운동2}
\end{align}
Eq. \ref{eq:등가속도 운동1}과 \ref{eq:등가속도 운동2}을 연립하여 시간변수를 없애 등가속도 운동에서 주로 사용하는 식 3가지를 유도할 수 있다. 
\begin{defn}[등가속도 운동공식]
    \begin{align}\label{eq:등가속도 운동공식}
     &v=v_0+at\\
     &s=v_0t+\frac{1}{2}at^2\\
     &2as=v^2-v^2_{0}
    \end{align}
  \end{defn}
 이때에 s는 이동거리를 뜻하며 1차원의 등가속도 운동을 다루기 때문에 이는 변위의 크기 $x-x_0$와 값이 같다. 
 이외의 물리량들도 방향을 포함하지 않으며 양과 음의 값으로 다루기만 한다.
이러한 등가속도 운동공식의 문자들을 함수로하여 위치-시간, 속도-시간, 가속도-시간 관계를 그래프로 표현하면 다음과 같다. 
\footnote{유용한 수학적 도구인 미분과 적분을 이용하면 가속도 운동을 분석하는데 도움이 된다. 부록에서 학습하여 보자.}

\begin{figure}[h]
  \centering
   \includegraphics[width=15cm]{사진모음/등가속도 운동.PNG}
   \caption{등가속도 운동 그래프} \label{fig:free-fallig}
 \end{figure}
\clearpage

\begin{task}
지면에서 연직 방향으로 질량 100g의 물체를 $39.2m/s$의 속도로 던져 올렸다. 5초 후 이 물체의 속도와 높이를 구하시오. (단, 공기 저항이 없고, 
$g=9.8m/s^2$이다. )

\end{task}


  \begin{task}[도르래(pully)]
 \begin{flushleft}
  {\IfFileExists{사진모음/연습3.png}
  {\includegraphics[width=0.4\linewidth]{사진모음/연습3.png}
  \label{fig:도르래상황}}%
  {\rule{\linewidth}{4cm}}}
 \end{flushleft}
 \textbf{팽팽한 줄의 장력은 같은 줄에서 어느 위치에서든 같다.} \\만약 위치에 따라 장력이 다르면 어떤 일이 일어날까?
\end{task}


  \begin{task}[도르래(pully)-2]
 \begin{flushleft}
  {\IfFileExists{사진모음/도르래 연습.png}
  {\includegraphics[width=0.4\linewidth]{사진모음/도르래 연습.png}
  \label{fig:도르래상황2}}%
  {\rule{\linewidth}{4cm}}}
 \end{flushleft}
 \textbf{1) A의 가속도의 크기는?\\ 2) 실이 A를 당기는 힘은?\\3) B가 정지상태에서 5초간 이동한 거리는? } \\
 (단, 중력가속도의 크기는 $10m/s^2$이다.)
\end{task}


\end{flushleft}

\section{실험으로 알아보는 힘의 특성}
\begin{flushleft}
이번 section의 concept들은 아래의 실험을 재료로 설명할 것이다. 아래의 실험 과정을 수행하여보고 
머릿속에 실험장면과 데이터를 있는 그대로 받아들여보자. 
 

\marginpar{
  \begin{center}
  \includegraphics[width=5.5cm]{스마트카트 동영상 설명자료.png}
  \captionof{figure}{YouTube : Pasco smartcart 검색}
  \end{center}} 

\begin{exper}{ \Large {수레의 분열 실험} }

  \label{exp:Momentum}

  \begin{enumerate}
  \item 준비물 : 역학용 스마트 카트, 수레 멈춤용 막대, 레일, 추(200g), 파스코 소프트웨어와 PC
  \footnote{실험전에 충분히 파스코 실험기기의 사용법을 익혀두자.}\\


\item 실험 방법 : 

  \IfFileExists{운동량 보존 실험.PNG}
  {\includegraphics[width=0.8\linewidth]{운동량 보존 실험.PNG}
  \captionof{figure}{실험 장면}\label{fig:Momentum}}%
  {\rule{\linewidth}{4cm}}%


  \begin{enumerate}
    \item 스마트카트와 CAPSTONE 프로그램을 연결하여 스마트 카트의 물리량을 컴퓨터로 확인한다.
    \item 레일 정중앙에 스마트 카트를 압축시켜 놓는다.
    \item 카트의 운동에 영향을 주지 않도록 조심하여 두 카트를 분열시킨다. 이 때 분열되어 힘을 받아 생성된 초기 수레의 최대 속도를 
    소프트웨어로 측정하고 기록한다.
    \item 두 카트의 질량(조작변인)을 달리하며 위의 과정을 반복한다. 질량과 분열속도의 곱-운동량(종속변인)을 표로 정리한다.
    \item 분열될때에 두 카트의 시간-속도 그래프를 보고서에 첨부한다.
  \end{enumerate}

  
  \phantom{dh}3. 실험해석 
  \begin{enumerate}
    \item 분열하는 도중에 보존되는 양이 있는가? 그 물리량은 무엇인가?
    \item 왜 보존된다고 생각하는가?
    \item 위의 해석을 통해 질량을 어떻게도 정의할 수 있겠는가?
  \end{enumerate}

\end{enumerate}
\end{exper}


\subsection{운동량(momentum)}


갈릴레이와 뉴턴 사이에는 데카르트라는 현인이 있었다. 그는 갈릴레이와 같이 한 물질은 자신의 운동상태를 
지속하려는 경향 '관성'을 갖는다고 생각하였다. 따라서 물체의 운동에서 보존되는 양으로 운동량의 개념을 생각해내었다. 
이후 뉴턴은 수학적으로 더 엄밀하게 운동량을 그 물체의 질량과 속도벡터를 곱한것으로 다음과 같이 정의하였다. 

\marginpar{
  \begin{center}
  \includegraphics[width=4cm]{Decartes.jpg}\captionof{figure}{Decaretes(1596-1650)}
  \end{center}} 


  \begin{defn}[운동량]

\begin{equation}
  \va{p} = m\va{v}
\end{equation}
운동량은 속도벡터에 질량이 배수된 \textbf{벡터}임으로 계산에 주의해야 한다.  

  \end{defn}

\marginpar{
  \begin{flushleft}
   운동량의 단위는 kg$\cdot$m/s, 또는 N$\cdot$s 이다. 
   $1$kg$\cdot$m/s=1N$\cdot$s 
  \end{flushleft}}

\noindent



\subsection{운동량과 힘의 관계(뉴턴의 제2 운동법칙)}
  
Eq. \titleref{eq:newton's law 2}($\vec{F}_{net}=m\vec{a}$)을 참고하여 실험 \ref{exp:Momentum}을 분석하여 보자. 
수레가 서로 용수철로 묵여 있을 때부터 두 수례를 하나의 계로 생각해보자.
 분열하는 순간에 두 수레로 이루어진 계에 작용하는 외력 $\vec{F}_{ext}$은\footnote{힘을 서술할 때 주어와 목적어 모두가 
계의 성분이라면 그것은 내력이다. 예를 들어 수레 A가 수레 B를 미는 힘은 수레A,B가 계의 성분임으로 내력이다.} 없다. 이에 의해 
계의 운동상태의 변화는 없어야 한다. 이전에 우리는 운동상태의 변화를 물체의 가속도로 두었으나 
이제는 가속도가 두 수레에 대해 존재하므로 대표할만한 가속도가 없다. 그렇다면 그 순간 계에 변하지 않은 운동상태의 변화를 대변하는 양은 무엇일까? \\
데카르트가 도입한 운동량을 사용하여 보자. 실험 \ref{exp:Momentum}에서 벡터로 이루어진 운동량 벡터의 전체합은 0이었다.
\begin{equation}
  m_1\va{v}_{1i}+m_2\va{v}_{2i}=m_1\va{v}_{1f}+m_2\va{v}_{2f}=0
\end{equation}



\textbf{외력이 없는 경우 두 개의 입자로 이루어진 계의 분열에서 변하지 않은 값은 바로 선운동량이다.}
\footnote{추후에 회전운동에는 병진운동의 운동량과 구분되는
각운동량이 등장한다. 따라서 이를 구분하기 위해 선운동량(linear momentum)이라고 한다.}
가속도를 운동상태의 변화로 두었던 우리의 생각을 확장하여 여러 입자로 이루어진 계에 대해 운동량 변화를 
운동상태의 변화로 확장시킬 수 있다. 
\marginpar{
  \begin{center}
  \includegraphics[width=4.0cm]{사진모음/운동량/수레의 분열.png}\captionof{figure}{두 수레 계에 작용하는 외력은 없다.}
  \end{center}} 

    \begin{defn}
      \centering{\textbf{선운동량의 시간변화율은 입자에 작용하는 알짜힘과 같다.}} 
    \begin{equation}\label{eq:newton's law 3}
      \vec{F}_{net}
      =\frac{\Delta \sum \vec{p}_{net}}{\Delta t}=\dv{\vec{p}_{net}}{t}
    \end{equation}
  \end{defn}
  
   $\va{p}=m\va{v}$으로 정의하므로 Eq. \ref{eq:newton's law 3}는 우리가 익히 사용하는 $\va{F} =m\va{a}$를 포함함을 알 수 있다.
  하지만 로켓과 같이 질량이 변하며 운동하는 물체의 운동을 분석할 때에는 외력이 없는데 로켓이 나아가는 것을 설명할 수 없다.
  이런경우엔 Eq. \ref{eq:newton's law 3}을 활용해야만 한다. 한마디로 더 넓은 경우에 대해서 설명해주는 더 범용적인 Eq.라고 할 수 있다. 
  실제 뉴턴이 집필한 프린키피아에도 Eq. \ref{eq:newton's law 3}의 식으로 적혀있는것을 이후의 과학자들이 $\va{F} =m\va{a}$로 
  변형한것이다.  
  \marginpar{
  \begin{center}
  \includegraphics[width=4.0cm]{사진모음/로켓.jpg}\captionof{figure}{로켓의 추진}
  \end{center}} 

  \noindent 
  수레의 분열 상황과 같이 \textbf{충돌, 분열 등 2개 이상의 물체로 이루어진 계에 작용하는 외력이 0일 때
   계의 운동량 변화가 없이 보존되는 것을 "운동량 보존법칙"}이라고 한다. 

   \begin{task}[운동량 보존법칙]
    
  \begin{flushleft}
     수평면에 정지해 있는 물체 B를 향해 물체 A가 $4v$의 속력으로 등속 직선 운동하고있다. 그래프는 A와 B가 충돌한 이후부터 
     B의 위치를 시간에 따라 나타낸 것이다. A와 B의 질량은 2kg으로 같다. A와 B는 충돌한 후 붙어서 함께 운동하며, 2초일 때 벽과 
     충돌 후 반대 방향으로 함께 운동한다. 각 질문에 답하여라. 
  
 \begin{tasks}[label=(\arabic*)](1)
      \task 속력 $v$의 값을 구하시오. \\
      \task A의 1초부터 6초까지 운동량 변화량을 구하시오.\\
   \end{tasks}
    
 \end{flushleft}
\end{task}
\end{flushleft}

\marginpar{
  \begin{center}
  \includegraphics[width=5.5cm]{사진모음/운동량/시간위치 그래프.png}\captionof{figure}{위치-시간 그래프}
  \end{center}} 




\subsection{운동량과 충격량(Impulse)}

\begin{flushleft}
  
Eq. \ref{eq:newton's law 3}에 따라서 우리는 한가지 물리량을 정의할 수 있다. 한 공이 힘을 받는 단순한 경우는 다른 
공과 충돌한 경우이다. 공이 받는 힘 $\vec{F}(t)$는 충돌하는 동안 변하면서 공의 선운동량 $\vec{p}$를 바꾼다. 이 것은 Eq. \ref{eq:newton's law 3}
에 따라서 다음과 같이 쓸 수 있다. 

\marginpar{
  \begin{center}
  \includegraphics[width=4.0cm]{사진모음/운동량/공의 충돌.png}\captionof{figure}{충돌}
  \end{center}} 


\begin{equation}
  d\vec{p}=\vec{F}(t)dt
\end{equation}

충돌 직적의 시각 $t_i$부터 충돌 직후의 시각 $t_f$까지 위의 식을 정적분 해주면 다음과 같이 표기할 수 있다. 

\begin{equation}
  \int_{t_i}^{t_f} d\vec{p}= \int_{t_i}^{t_f} \vec{F}(t)dt
\end{equation}
이 식의 좌측항 $\vec{p}_f-\vec{p}_i= \Delta\vec{p}$ 선운동량의 변화량을 뜻하고 오른쪽 항을 충격량이라고 부른다. 충격량은 
충격력과 시간간격의 곱으로 이를 정의 하면 아래와 같다. 

\begin{defn}[충격량의 정의]
  \begin{align}
 &I = \int_{t_i}^{t_f}\vec{F}(t)dt\\
 & \textrm{if}, \phantom{o} F=\textrm{const},\phantom{o}I=F \Delta t
\end{align}
\centering{\textbf{선운동량의 변화량은 물체에 가해지는 충격량과 같다.}}
\end{defn}
이를 기하학적인 그래프로 확장하면 $\vec{F}(t)$의 함수를 시간 변수에 대해 적분하면 충격량(Impulse) $\vec{I}$을 구할 수 있다.
따라서 적분의 기하학적 의미에 따라 F-t그래프에서 그래프 개형과 t축 사이의 넓이를 구한다면 충격량을 알 수 있다. 

\marginpar{
  \begin{center}
  \includegraphics[width=4.5cm]{사진모음/운동량/F-t 그래프.png}\captionof{figure}{F-t 그래프의 넓이는 충격량과 같다.}
  \end{center}} 
\end{flushleft}

\begin{task}[운동량 충격량 관계]
    
  \begin{flushleft}
      호수 위에서 질량이 다른($m, 2m$)의 두 보트 A,B가 일정한 바람의 힘($F$)을 받아 움직인다. 
      두 보트가 정지 상태에서 출발하여 
      거리 $L$만큼 떨어진 선착장에 도착하였다. 어느 보트가 더 큰 운동량을 갖고 선착장에 도착하는가? 단, 모든 마찰력은 무시하고 가능한한
      정량적으로 비교하여보라.
 \end{flushleft}
\end{task}

   
\subsection{작용-반작용(뉴턴의 제3 운동법칙)}

\begin{flushleft}
힘의 특성을 파악하기 위해 분열하는 두 수레의 실험 상황을 들여다보자.

분열되는 두 수레의 계에는 외력이 0이므로 두 수레의 전체 운동량이 보존된다. 
\begin{equation}
  m_1 \va{v_1} +m_2 \va{v_2} =0      \Rightarrow   m_1 \va{v_1} =-m_2 \va{v_2}
\end{equation}
이것은 Eq.\ref{eq:newton's law 3}에 따라 아래와 같이 전개 수 있다.\footnote{운동량 보존과 작용-반작용 법칙은 맥이 같다.}
\marginpar{
  \begin{center}
  \includegraphics[width=4cm]{분열.png}\captionof{figure}{분열}
  \end{center}} 

\begin{equation}
  \dv{(m_1 \va{v_1})}{t} =-\dv{(m_2 \va{v_2})}{t} 
\end{equation}

\begin{equation}
  \vec{F}_{21}=-\vec{F}_{12}
\end{equation}
좌변항은 물체1이 계인 경우로 물체2로부터 받는 힘을 나타내고, 우변항은 물체2가 1로부터 받는 힘이다. 
이 두 힘은 크기가 정확히 갖고 방향이 반대이다.

이것이 바로 뉴턴의 제3 운동법칙 작용-반작용 법칙이다. 작용과 반작용에 해당하는 힘의 서술은 힘을 주는 주어와 힘이 작용하는 
목적어가 서로 상호교환 하면 된다. 예를 들어 \textbf{물체 1이 물체 2를 미는 힘의 반작용은 물체 2가 물체 1을 미는힘이 된다.}
이때 주의할 점은 작용 반작용 관계에 있는 힘들은 서로 다른 물체에 작용하여 서로 상쇄되지 않는다. 
\textbf{한 물체에 서로다른 방향의 같은 힘이 작용하여 정지해있는 힘의 평형과 잘 구분하여야 한다.}
\newpage
\begin{task}[object on a horizontal object]
  \begin{flushleft}
    {\IfFileExists{사진모음/운동방정식/reactioan_task.eps}
    {\includegraphics[width=0.6\linewidth]{사진모음/운동방정식/reactioan_task.eps}
  \label{fig:물체옆의물체}}%
    {\rule{\linewidth}{4cm}}}
   \end{flushleft}
\end{task}



\subsection{뉴턴의 운동법칙}
이제까지 뉴턴 이전의 현인들의 생각을 알아보고 간단한 실험을 진행했다. 이를 통해 최종적으로 필자가 말하고자 한것은 
3가지로 축약가능한 뉴턴의 운동법칙이다. 이를 정리하면 아래와 같다. 


\begin{defn}[뉴턴의 3가지 운동법칙]
  \begin{enumerate}
  \item First motion law : 계에 외부력이 작용하지 않으면($\sum \overrightarrow{F_{ext}} = 0$) 계는 운동상태를 유지한다. 이것을 \textbf{관성법칙}이라고 한다.
  \item Second motion law: $\sum \overrightarrow{F_{ext}} \neq 0$이고 $\Delta m =0$ 이라면   $\overrightarrow{F_{net}}
      =\dv{\overrightarrow{p_{net}}}{t} =\dv{m \va{v}}{t}= m\va{a}$ 이것을 \textbf{가속도 법칙}이라고 한다.
  \item Third motion law: 힘은 상호작용으로 물체쌍 사이에서 일어나고 서로에게 작용하는 힘은 같은 크기로 반대방향이다. 
  ($\overrightarrow{F_{21}}=-\overrightarrow{F_{12}}$) 이를 \textbf{작용-반작용 법칙}이라고 한다.
\end{enumerate}
\end{defn}

이 3가지 운동법칙은 역학의 문제 상황에 예외없이 적용되는 명제들이다. 
이러한 예외없음이 비약적인 과학발전을 일으켰고 주변 물체들의 운동에 대한 이해를 인간에게 갖고 왔다. 
뉴턴의 운동법칙으로 시작되는 고전역학의 정립을
과학혁명이라고 말하는 이유도 여기에 있다. \footnote{물리학에는 
너무나 많은 문자들이 등장한다. 이 문자들은 물리량을 추상적으로 표현하며 [m]미터와 같은 단위와는 그 의미가 다르다. 
"물리학 문자게임"을 통해 다양한 문자의 의미를 정확히 알고 친숙해져보자.} 

\end{flushleft}


\subsection{더 다룰만한 상황 : 경사면}
\begin{flushleft}
  
지금부터는 상황이 조금 추가된 상황을 자유물체도와 뉴턴의 운동법칙을 적용하여 분석하도록 해보자. 아래의 빈칸의 그림에 있는 상황에서 각 물체에
작용하는 힘을 자유물체도로 표현하고 각 물체에 대한 운동방정식($\va{F}=m\va{a}$)을 적용하여 물체의 운동을 예측해보자. 



\begin{task}[slope inclined plane]
    \begin{flushleft}
  {\IfFileExists{사진모음/운동방정식/slope_task.eps}
  {\includegraphics[width=0.6\linewidth]{사진모음/운동방정식/slope_task.eps}
  \label{fig:빗면}}%
  {\rule{\linewidth}{4cm}}}
 \end{flushleft}
\end{task}


\begin{task}[object on the object]
  \begin{flushleft}
    {\IfFileExists{사진모음/연습4.png}
    {\includegraphics[width=0.6\linewidth]{사진모음/연습4.png}
  \label{fig:물체위의물체}}%
    {\rule{\linewidth}{4cm}}}
   \end{flushleft}
\end{task}


\begin{task} [조별 운동분석 직소활동]
  각 조별로 위의 물체의 상황과 같이 물체의 상황을 만들어보고 
  이에 대한 자유물체도와 운동방정식을 정리하여보자. 
  이후에는 다른 조들의 상황을 정답없이 공유하여보고 이에 대해
  토론해보며 운동분석에 자신감을 갖도록하자. 
\end{task}

    \end{flushleft}     

 
% \subsection{포물선 운동}


% \subsection{등속 원운동}


% \subsection{진자의 운동}


% \section{케플성의 행성운동법칙}


\part{에너지와 열현상}
\pagestyle{MyRuled}
\chapter{에너지와 열역학}
\chapterprecis{\noindent 이 장에서는 에너지에 대해 배우고 열현상을 해석한다.}

\begin{flushleft}
\section{일과 에너지}
    
에너지라는 단어는 고대 그리스어 energia에서 en(안, 내부)+ ergon(일) 파생되었다.
 기원전 4세기 아리스토텔레스의 기록에서 사용되었고, 
 현대의 엄밀한 정의와 달리 에네르게이아는 포괄적인 개념이었다.\footnote{여기서 말하는 것은 "에너지가 넘쳐보인다."와 같은 표현이 넓은 의미의 
 에너지로 활기, 행복과 같은 정량적으로 비교가능하지 않는 양을 말한다.}

17세기 후반에 라이프니츠(Gottfried Leibniz)는 공이 벽면에 충돌해서 반대로 이동하는 경우를 집중해보았다. 
당구공과 같이 공이 충분히 탄성이 있고 벽면이 딱딱하다면 공은 속력이 변하지 않으며 되튀어져 나온다. 
운동량 보존을 이용한다면,
공의 속도는 벡터량임으로 되튀어지며 운동량이 2배만큼 변한다. 충돌 된 
벽면의 질량은 너무 커 벽면과 공의 계의 운동량 보존으로 상황을 분석하기엔 적절하지 않아보인다.
라이프니츠는 똑같은 속력으로 되튀어 나오는 공의 운동과정
\footnote{이런 에너지의 손실이 없는 충돌 과정을 특별히 탄성 충돌이라 한다.}에서 
 \marginpar{
  \begin{center}
    \includegraphics[width=4cm]{사진모음/아리스토텔레스.jpg}
    \captionof{figure}{Aristoteles\\(B.C 384~B.C322)}
  \end{center}}
속력의 양으로 표현할만한 보존되는 스칼라 양이 필요하다고 생각했다. 

  %  \marginpar{
  %     \begin{center}
  %       \includegraphics[width=5.0cm]{사진모음/에너지/벽과의 충돌.eps}//eps파일은 영어 이름이어야만 한다. 
  %       \captionof{figure}{벽과의 탄성 충돌}
  %     \end{center}}
      
\begin{figure}[h]
\centering
   \includegraphics[width=5cm]{사진모음/에너지/crack.eps}
   \caption{벽과의 탄성 충돌}
\end{figure}

라이프니츠는 물체의 질량과 속도의 제곱의 곱으로 나타나는 새로운 스칼라 물리량(vis viva)을 제안했다.  
이러한 스칼라 물리량은 후대의 과학자들이 다듬어 에너지라는 보존되어지는 양으로 정의했다. 
보존되어지는 스칼라량이란 마치 계좌에 쌓여있는 돈처럼 지출을 할 수 있지만 
그 가치의 총량은 변하지 않는 무엇인가로 생각할 수 있다. 이러한 에너지는 다른 형태의 에너지로 전환되지만 그 총량은 변하지 않는다.
\footnote{이러한 에너지는 최종적으로는 가장 변환하기 힘든 형태인 열에너지로 전환된다.} 



\subsection{운동에너지}
\marginpar{
  \begin{center}
    \includegraphics[width=4cm]{사진모음/Thomas_Young_(scientist).jpg}
    \captionof{figure}{Thomas Young\\(1773~1829)}
  \end{center}}
1807년에 Thomas Young은 현대적인 의미에서 라이프니츠의 활력(vis viva) 대신에 \\"에너지"라는 용어를 처음으로 사용했다.
코리올리(Gustave-Gaspard Coriolis)는 1829년에 "운동 에너지"를 현대적인 의미로 설명했다.  

\[
  \begin{split}
  F &= m \dv{v}{t}\\
  \vec{F}\cdot d \vec{x} &= m \textrm{d}v \cdot \dv{\vec{x}}{t}\\
  \int_{x_i}^{x_f} \vec{F} \cdot \textrm{d} \vec{x} &= 
  \int_{v_i}^{v_f}m\vec{v} \cdot \textrm{d}\vec{v} = \int_{v_i}^{v_f}mv \textrm{d}v
  =\frac{1}{2}mv_f^2-\frac{1}{2}mv_i^2
  \end{split}
\]

운동에너지 K(kinetic energy)는 물체의 운동상태와 관련된 에너지로 질량 $m$인 물체의 속력 $v$가 광속보다 훨씬 작을 때,
운동에너지를 다음과 같이 정의한다. \footnote{보다 포괄적인 운동에너지는 상대론의 논리에 따라 정의된다.}


\marginpar{
  \phantom{h}
  \phantom{h}
  \begin{flushleft}
   운동에너지의 단위는 J이다. 더 확장하여 에너지, 열량, 일의 단위도 J이다. 
   $1 joule=1$J$=1$kg $\cdot$$m^2/s^2$
  \end{flushleft}}


\begin{defn}[운동에너지]
  \begin{align}
 &K=\frac{1}{2}mv^2
\end{align}
\centering{\textbf{물체의 질량이 클수록 더 빠른 속력을 가질수록 운동에너지가 크다.}}
\end{defn}




\subsection{일}
수평 $x$축으로 뻗어있는 마찰 없는 줄을 따라 미끄러지는 물체를 생각해보자. 줄과 각도 $\phi$를 이루는 방향으로 작용하는 일정한 
힘 $\va{F}$가 물체를 가속시키고 있다. 뉴턴의 제 2법칙에 따라 물체의 운동을 분석한다면 다음과 같이 나타낼 수 있다. 

\begin{figure}[h]
  \centering
   \includegraphics[width=5cm]{사진모음/에너지/work.eps}
   \caption{일-에너지}
 \end{figure}

 \[
\begin{split}
  &F\cos{\phi}= ma_{x}\\
  &F_{x}=ma_{x}
\end{split}
\]
일정한 힘이 작용하는 등가속도 운동의 상황임으로 등가속도 운동공식을 사용하면 다음과 같이 정리가 가능하다.
\[
\begin{split}
  &v^2={v_0}^2+2a_{x}d\\
  &\frac{1}{2}mv^2-\frac{1}{2}mv_{0}^2=F_{x}d=F\cos{\phi}d=\va{F}\cdot \va{d}
\end{split}
\]
이를 통해 일정한 \textbf{힘이 작용한 방향과 같은 방향의 변위의 곱만이 운동에너지의 변화와 같음}을 알 수 있다.
이는 두 벡터의 스칼라곱과 의미가 같다. 즉,
 \textbf{물체의 에너지를 변화시키는 힘과 그 물체의 변위의 스칼라곱은 에너지를 변화시키고 이를 "일"이라고 한다.} 
이를 정리하면 다음과 같이 정의할 수 있다.

\begin{defn}[일정한 힘이 한 일]
  \begin{align}
 &if \phantom{h}F=const\\
 &W =\va{F}\cdot \va{d}\\
 &=K_f-K_i =\frac{1}{2}mv^2-\frac{1}{2}m{v_0}^2
\end{align}
\end{defn}

\marginpar{
  \begin{center}
    \phantom{a}\\
    \phantom{a}\\
    \phantom{a}\\
    \includegraphics[width=5.5cm]{사진모음/에너지/storm.eps}
    \captionof{figure}{Task figure}
  \end{center}}

\begin{task}[일정한 힘이 한 일]
  \begin{flushleft}
 
  강풍이 부는 와중에 미끄러운 바닥 위의 물체가 $\va{d}=(-2m)\vu{i}$만큼 움직였다. 
  이때 강풍은 물체에 $\va{F}=(3N)\vu{i}+(-3N)\vu{j}$의 
  일정한 힘을 
  지속적으로 작용하고 있다. \\
  \phantom{h}\\
  1) 물체의 변위 동안 강풍이 물체에 한 일은 얼마인가?\\
\indent

  2) 이동하기 시작할 때 나무 짐짝이 10J의 운동에너지를 갖고 있다면 이동이 끝난 후의 운동에너지는 얼마인가?

\end{flushleft}
\end{task}



좀 더 일반적인 상황은 힘이 변화하는 경우일 것이다. 상황을 간단히 하기 위해 x축방향으로 작용하는 변화하는 힘과 
힘이 작용하는 물체를 생각해보자. 물체가 $x_i$에서 $x_f$까지 이동하는 경우에 짧은 구간에 대해서는 힘이 일정하다고 생각할 수 있다.
그리고 그 짧은 경우들을 더하면 전체 변위동안에 해준 일을 구할 수 있다. 이를 수식으로 전개하면 아래와 같다. 

\[
\begin{split}
 &W=\sum\Delta W_j=\sum F_{avg}\Delta x \\
 &W = \lim_{\Delta x \to 0} \sum F_{avg} \Delta x\\
 &= \int \; \vec{F} \; \textrm{d}\va{x} 
\end{split}
\]


\marginpar{
  \begin{center}
    \includegraphics[width=5cm]{사진모음/힘과 변위의 적분.png}
    \captionof{figure}{힘이 한 일은 F-x그래프의 넓이와 같다.}
  \end{center}}

이는, \textbf{$F-x$ 그래프의 넓이의 의미와 같다.} 따라서 변화하는 힘이 한 일은 확장하여 다음과 같이 정의할 수 있다. 

\begin{defn}[변화하는 힘이 한 일]
\[ 
  \textrm{if, F not const}
\]
\begin{align}
 &W =\int_{x_i}^{x_f}\va{F} \cdot d\va x\\
 &=K_f-K_i
\end{align}

\end{defn}


\marginpar{
  \begin{center}
    \phantom{a}\\
    \phantom{a}\\
    \phantom{a}\\
    \includegraphics[width=5.5cm]{사진모음/에너지/transfomating.eps}
    \captionof{figure}{Task figure }
  \end{center}}

\begin{task}[변하는 힘이 한 일]
  \begin{flushleft}
2kg의 토막이 힘이 가해짐에 따라 마찰이 없는 수평면의 바닥에서 미끄러지면서 $x_1=0$에서 출발하여 $x_3=6m$에서 끝난다. 토막이 움직이면서 힘의 크기는
그래프와 같이 수평 위치에 따라 변한다. 힘의 방향과 초기속도의 방향은 수평면의 양의 방향으로 같다.
\phantom{a}\\
\phantom{a}\\
\textbf{Q.} $x_1$에서 토막의 운동에너지는 $K_1=36$J이다. $x_1=0, x_2=4.0m, x_3=6m$에서 토막의 속력을 구하여라. 
\end{flushleft}
\end{task}


\marginpar{
  \begin{center}
    \includegraphics[width=5.5cm]{사진모음/힘-변위그래프.png}
    \captionof{figure}{Task Graph}
  \end{center}}



\newpage
\subsection{퍼텐셜에너지와 역학적 에너지의 보존}
\subsubsection*{퍼텐셜에너지}
 중력을 설명하기 위해 section\ref{sec:gravity}에서 도입했던 던져진 토마토를 다시한번 생각해보자. 토마토가 최고 높이에서 떨어질 때 
 중력이 토마토에 한 일은 토마토의 운동에너지의 증가값이 된다. 중력의 방향($\va{F}$)과 변위($\va{s}$)의 방향이 같기 때문이다. 
 따라서 운동에너지의 변화는 양이된다.
하지만 이 때에 지구와 토마토 계에 대해서 에너지는 보존되기를 기대한다. 운동에너지가 증가하는만큼 무엇이 감소했을까? 
보존을 위해\footnote{보존이라는 개념이 너무 당연하게 들어오는게 거부감 들수도 있다. 실제로 에너지 보존 법칙은 19세기초에나 처음으로 가정되었다. 하지만 만물의 상황에 대한 통찰과 전개는 "보존"과 "대칭"의 아이디어로 생겨났다.} 
이에 꼭맞는 \textbf{중력이 한일의 크기에 음(-)을 붙여 에너지를 정의하고 이를 퍼텐셜에너지의 변화량으로 정의하여보자.}

\marginpar{
  \phantom{h}
  \begin{center}
  \includegraphics[width=4cm]{사진모음/에너지/tomato.eps}\captionof{figure}{중력을 받는 토마토의 운동}
  \end{center}} 

  \begin{defn}[중력 퍼텐셜에너지와 퍼텐셜에너지의 정의]
  \begin{align}
 & -W_{grv}= \Delta U\\
 & =- \int_{0}^{h} \va{F}_{grv} \cdot\ d \va{s}\\
 & = -(-mg)\vu{j} \cdot h\vu{j} =mgh  
\end{align}
\centering 
\textbf{중력이 한일의 음의 값을 중력 퍼텐셜 에너지로 정의한다.}
\end{defn}


\subsubsection{역학적에너지}
그리고 이러한 퍼텐셜에너지($U$)와 운동에너지($K$)를 합쳐 역학적 에너지($E_{mec}$)라고 정의하여 보자. 

\begin{equation}
  E_{mec} =K+U
\end{equation}
  

그렇다면 
토마토를 중력이 작용하는 공간에서 수직으로 던졌을 때 에너지 보존을 설명할 수 있다. 
상승할 때에는 중력이 한일의 역수만큼인 퍼텐셜에너지가 증가하고 이에 따라 운동에너지는 감소한다. 반대로 
하강할 때에는 운동에너지는 증가하는 반면 꼭 그만큼의 퍼텐셜에너지가 줄어든다. 따라서 운동에너지와 퍼텐셜 에너지의 합은 일정하다. 
\newpage

\begin{defn}[중력장에서 역학적 에너지의 보존]
  \begin{align}
    & W_{grv}=\Delta K ,\qquad -W_{grv}= \Delta U \\
    & \Delta K = -\Delta U\\
    & =K_f-K_i= -(U_f-U_i)\\
    & E_{mec}=K_i+U_i=K_f+U_f
\end{align}
\centering
\textbf{중력장 내에서 계의 역학적에너지는 보존된다.}
\end{defn}

\subsubsection*{보존력}
이와 같이 에너지 보존을 위해 도입한 퍼텐셜 에너지는 \textbf{특정한 힘들에 의해서만 정의할 수 있다. 중력과 같은 이러한 힘을 보존력이라고 한다.} 
\begin{equation}
  \Delta U =-\int_{x_i}^{x_f} \vec{F_{con}} \cdot \textrm{d}\vec{s}
\end{equation}

보존력(Conservative Force)이란 힘에 의한 일을 구하였을 때, 그 일의 양이 경로에 무관하고 오직 물체의 처음 위치와 나중 위치에만 의존하는 힘이다. 
\footnote{보존력에 의한 일은 경로에 무관하므로 위치에만 의존하는 함수의 차로 정의할 수 있고 그런 맥락에서 '위치에너지'라고도 한다.}

\marginpar{
   \phantom{a}\\
    \phantom{a}\\
  \begin{center}
  \includegraphics[width=5.5cm]{사진모음/훅의 법칙.PNG}\captionof{figure}{훅의 법칙}
  \end{center}} 

\begin{task}[탄성퍼텐셜 에너지]
    \begin{flushleft}
  대표적인 보존력인 탄성력을 이용하여 탄성퍼텐셜에너지를 정의하고 이를 이용하여 보자. 탄성력은 Robert Hooke에 의해서 처음 정의되었다. 
  \begin{equation}
    \vec{F_x} = -k\va{x}
  \end{equation}
  옆의 그림과 같이 용수철이 늘어나거나 줄어든 변위 x와 힘 F는 항상 방향이 반대이다.\footnote{이러한 맥락에서 탄성력을 복원력이라고도 한다.}
  그래서 용수철 상수 $k$앞에는 (-)음의 부호가 붙는다. 
  이때 용수철 상수는 용수철의 탄성을 나타내는 양이다. $k$가 크면 용수철의 탄성이 더 크다. 

    \begin{tasks}[label=(\arabic*)](1)
      \task 용수철의 탄성 퍼텐셜에너지를 정의하라. \\
      \task 질량 $m=1$kg의 토막이 마찰없는 수평면을 따라 속력 $2m/s$로 움직이다. 용수철 상수 $20N/m$의 용수철에 부딪혀 용수철을 압축한다. 
      토막이 용수철에 의해 순간적으로 멈추었을 때 용수철의 압축된 길이 $d$를 구하여라.\\
   \end{tasks}
\end{flushleft}
\end{task}



보존력이 작용하는 경우에 한해서 우리는 퍼텐셜에너지를 정의하고 역학적 에너지 보존을 사용할 수 있다. 역학적 
에너지가 보존되지 않는 경우는 비보존력이 작용하는 것이다. 운동마찰력과 공기의 항력은 비보존력이다. 
예를 들어 마찰이 있는 마루 위에서 토막을 미끄러트려 보자. 미끄러지는 동안 마루가 작용하는 
마찰력은 토막에 음의 일을 하므로 운동에너지가 줄어든다. 하지만 퍼텐셜에너지의 증가는 없다. \\
이때의 에너지는 어디로 간 것일까?
이 에너지들은 열에너지로 전환된다. 물체와 표면이 뜨거워지는 것을 보면 알 수 있다. 이런 경우는 
역학적 에너지 보존으로 운동을 분석하기 어렵다.
반대로 보존력만 있는 경우에는 물체의 운동에 관한 어려운 문제들을 매우 간단하게 풀 수 있다. 
\subsection{외부력이 계에 한 일}

보존력이 한일은 운동에너지를 증가시키지만 위치에너지를 꼭 같이 감소시켜 역학적 에너지를 보존시킨다.
자, 이제 상황을 확장해보아 일반적인 경우를 보자. 계에 보존력 이외의 외부력이 일을 한다면 
이는 계의 역학적 에너지를 증가시킨다. 외부력이 해준 일은 상황에 따라 
운동에너지와 퍼텐셜에너지에 나누어져 역학적 에너지를 증가시킨다. 
\begin{equation}
  W=\Delta E_{\textrm{mec}}=\Delta K+\Delta U
\end{equation}

\begin{task}[외부력]
  \begin{flushleft}
질량 $m=10$kg의 포탄을 평균 20000N의 연직방향($\vu{j}$)의 힘으로 변위 $\va{s}=(0.05m) \vu{j}$만큼 폭약이 힘을 주어 
바닥에서 발사시켰다.  
    \begin{tasks}[label=(\arabic*)](1)
      \task 포탄이 올라갈 수 있는 최대 높이는 얼마인가? \\
      \task 포탄으로부터 6m 높이에 용수철 상수 100N/m의 용수철이 있다면 압축되는 길이는 얼마인가?\\
   \end{tasks}
\end{flushleft}
\end{task}


\marginpar{
  \begin{center}
  \includegraphics[width=5.5cm]{사진모음/에너지/spring.eps}\captionof{figure}{용수철 에너지 보존 상황}
  \end{center}} 




하지만 이는 마찰이 없을 때로 국한된다. 마찰이 있는 경우는 역학적 에너지 이외에 열에너지로 전환되는 에너지가 생긴다. 따라서 
마찰이 포함된 운동의 경우 계에 한일은 다음과 같이 정의된다. 

\begin{equation}
  W=\Delta E_{\textrm{mec}}+\Delta E_{\textrm{th}}
\end{equation}

\begin{task}[일, 마찰력]
  \begin{flushleft}
사람이 양배추를 담은 전체 질량 $m=10$kg의 나무궤짝을 크기 40N의 일정한 수평력 $F_0\vu{i}$으로 콘크리트 바닥에서 밀고 있다. 
변위는 $\va{s}=(0.5m) \vu{i}$이고, 궤짝의 속력은 $v_0=4$m/s에서 $v=2$m/s로 감소한다. 
    \begin{tasks}[label=(\arabic*)](1)
      \task 힘 $\va{F}$가 한 일은 얼마인가? \\
      \task 궤짝과 바닥의 열에너지 증가 $\Delta E_{th}$의 크기는 얼마인가?\\
   \end{tasks}
\end{flushleft}
\end{task}

\marginpar{
  \begin{center}
  \includegraphics[width=5.5cm]{사진모음/에너지/cabbage.eps}\captionof{figure}{마찰면 상황}
  \end{center}} 

\subsection{탄성충돌과 비탄성충돌}
우리가 배운 운동량 보존 법칙과 에너지 보존 법칙을 이용하면 보다 많은 상황을 분석할 수 있다. 
두 물체가 충돌하는 상황을 가정하자. 외력($\frac{d\vec P}{dt}$)이 0인 경우 운동량 보존에 의해 두 물체의 충돌 전과 충돌 후의 운동량은 보존될 것이다. 
하지만 운동에너지는 보존되지 않는다. 왜냐하면 보통의 경우에는 운동에너지 이외에 열에너지의 형태로 에너지가 변환되기 때문이다.
이때에 운동에너지가 보존되는 상황을 탄성충돌 그렇지 않은 상황을 비탄성 충돌이라고 한다. 



\marginpar{
  
  \begin{flushleft}
   \textbf{Question} 탄성충돌과 비탄성 충돌을 관찰할 때 어떤 차이점이 있을것이라고 예상하는가?
  \end{flushleft}}


\subsubsection{탄성충돌(elastic collision)}
fig. \ref{Energy collision}\과 같이 두 물체가 충돌하는 경우를 다룬다.
상황을 간단히 하기 위해 충돌과정은 1차원으로 제한하며 물체 A에 대한 표적(B)이 정지한 경우 ($v_{2i}=0$)를 생각해보자. 


모든 입자의 충돌에서 충돌 전후의 운동량은 보존된다. 따라서, 
\begin{equation} \label{Eq.운동량보존}
  m_1v_{1i} =m_1v_{1f} +m_2v_{2f} \Rightarrow m_1(v_{1i}-v_{1f})=m_2 v_{2f}
\end{equation}

탄성충돌은 운동에너지가 보존되는 경우이다. 따라서 역학적 에너지 보존을 적한다.

\begin{equation} \label{Eq.에너지보존}
  \frac{1}{2} m_1v^2_{1i} =\frac{1}{2}m_1v^2_{1f} +\frac{1}{2}m_2v^2_{2f} 
  \Rightarrow m_1(v^2_{1i}-v^2_{1f})=m_2v^2_{2f}
\end{equation}

\marginpar{
  \begin{center}
  \includegraphics[width=5.5cm]{사진모음/에너지/twobody.eps}\captionof{figure}{두 물체의 충돌 분석}
  \label{Energy collision}
  \end{center}} 


Eq \ref{Eq.운동량보존}, \ref{Eq.에너지보존}을 연립하면 다음과 같은 식을 얻는다. 
\[
\begin{split} 
   &m_2v_{2f}(v_{1i} +v_{1f})=m_2v^2_{2f} \rightarrow v_{1i}+v_{1f}=v_{2f}\\
   &\therefore \phantom{text} m_1(v_{1i}-v_{1f})=m_2(v_{1i}+v_{1f})\\
\end{split}
\]




이를 $v_{1f}$와 $v_{2f}$에 대해 정리하면 
\begin{align} 
   &v_{1f} =\frac{m_1-m_2}{m_1+m_2} v_{1i} \\
   &v_{2f}=\frac{2m_1}{m_1+m_2}v_{1i}  
\end{align}

\begin{task}[탄성충돌]
  \begin{flushleft}
  1차원에서 물체1이 정지해있던 물체2에 $\vec{v_{1i}}\vu{i}$의 속도로 충돌하는 경우를 생각해보자. 충돌이 탄성충돌이라 할 때 
  아래의 물음에 답하시오. \\
  \phantom{h}\\
  1) 물체1과 물체2의 질량이 같은 경우($m_1=m_2$), $v_{1f}$와 $v_{2f}$를 구하여라.\\
\indent

  2) 물체2의 질량이 많이 큰 경우($m_2 \gg m_1$), $v_{1f}$와 $v_{2f}$를 구하여라. \\
  \indent

  3) 물체1의 질량이 많이 큰 경우($m_1 \gg m_2$), $v_{1f}$와 $v_{2f}$를 구하여라. 

\end{flushleft}
\end{task}

\begin{task}[탄성충돌 예시문항]
  \begin{flushleft}
  2m/s의 속력으로 움직이는 물체 A($m_1=1$kg)가 반대 방향, 같은 속력으로 움직이는 물체 B($m_2=0.5$kg)와 탄성충돌을 하였다.
  충돌 후 각각의 속력을 구하시오.\\
\end{flushleft}
\end{task}

\subsubsection{비탄성충돌(inelastic collision)}
충돌과정에서 계에 작용하는 외력이 없는 상황에서 운동량은 보존된다. 하지만 운동에너지는 비보존력에 의해 열에너지로 손실되면
보존되지 않는다. 따라서 자연에서 일어나는 대부분의 거시적 충돌은 비탄성 충돌이다. 특별히, 
충돌 후 물체가 하나로 합쳐지는 경우를
\textbf{완전비탄성 충돌}이라고 한다. 

\[
\begin{split} 
   &\va{p}_{1i}+\va{p}_{2i} =m_1\va{v}_{1i}+m_2\va{v}_{2i}=(m_1+m_2)\va{v}_f 
\end{split}
\]

완전 비탄성 충돌 후 물체의 운동은 질량중심의 운동과 일치한다.

\[
\begin{split} 
   &\va{v}_{cm}=\frac{m_1\va{v}_{1i}+m_2\va{v}_{2i}}{(m_1+m_2)}
\end{split}
\]


\begin{task}[비탄성충돌 탄동진자]
  \begin{flushleft}
  질량 M=1kg인 나무로 만든 탄동 진자에 질량 m=0.01kg 인 탄이 $v$의 속력으로 날아와 박혔더니, 이 진자가 높이 $h=0.5$m만큼 올라간
  후, 잠시 멈추고 처음 위치로 내려왔다. 탄환의 속력 $v$를 구하시오. (단, 줄의 질량과 공기마찰은 무시하고 
  계산기를 이용하여 해결한다.)
\end{flushleft}
\end{task}

\marginpar{
  \begin{center}
  \includegraphics[width=5.5cm]{사진모음/에너지/pendlum.eps}\captionof{figure}{탄동진자}
  \label{Energy collision}
  \end{center}} 

\end{flushleft}

% \begin{flushleft}
    \section{에너지-열현상}
    물이 찬 주전자를 가스레인지에 가열한다. 그러면 수증기가 생겨 주전자 뚜껑을 움직이고 요란한 소리까지 만든다.
    이 과정을 곱씹으면 우리가 주위에서 느끼는 열(heat)현상은 역학적 현상과 연관되어 보인다. 
    그렇다면 엄밀하게 그 양들을 따질 수 있어야 한다.
    이번 Section에서는 열현상을 분석할 것이다. 그러기 위해서 우리는 주변을 보는 시각을 원자적으로 바꾸어야만 한다. 
    왜냐하면 우리가 배운 뉴턴의 역학을 이용하여 유체, 기체 등을 입자로 보고 그 입자의 역학적 에너지를 확장하여 
    정의할 것이기 때문이다. 이렇듯 물리학은 연역적이다.  열역학을 시작하기 이전에 열역학에서 사용하는 물리량을 
    정의하고 가는 것은 유용하고 필수적이다. \textbf{부피, 밀도, 압력}으로 이 세 물리량은 모두 스칼라량이다. 부피는
    직관적으로 알고 있는 물체가 공간에 차지하는 양으로 생각하면 된다. 밀도와 압력은 다음과 같다.

      \begin{defn}[밀도($\rho$)]
       \begin{equation}
        \rho=\frac{\Delta m}{\Delta V} =\frac{m}{V} (\phantom{o} in\phantom{o}const \phantom{o}density)
      \end{equation}
      \centering 
      \begin{flushleft}
      \textbf{밀도 $\rho$는 그 점을 포함하는 작은 부피안에 질량이 무거울수록 크다.}
      \end{flushleft}
      \end{defn}


      \begin{defn}[압력($p$)]
       \begin{equation}
        p=\frac{\Delta F}{\Delta A} =\frac{F}{A} (\phantom{o} in\phantom{o}equal \phantom{o}force \phantom{o}to \phantom{0}Area)
      \end{equation}
      \centering 
      \begin{flushleft}
      \textbf{압력 $p$는 점을 중심으로 하는 면의 면적 $\Delta A$가 무한히 작아지는 극한값이다. 면에 대해 힘이 일정한 경우 $\frac{F}{A}$로 쓸 수 있다.}
      \end{flushleft}
      \end{defn}

% %  \begin{defn}[압력($p$)]
%          \begin{equation}
%          \p=\frac{\Delta F}{\Delta A}=\frac{F}{A} (\phantom{o} in\phantom{o}equal \phantom{o}force\phantom{o} to\phantom{o} Area)
%       \end{equation}
%       \centering 
%       \begin{flushleft}
%       \textbf{압력 $\p$는 점을 중심으로 하는 면의 면적 $\Delta A$가 무한히 작아지는 극한값이다. 면에 대해 힘이 
%       일정한 경우 $\frac{F}{A}$로 쓸 수 있다.}
%       \end{flushleft}
% %       \end{defn}



     \subsection{열에너지와 온도}
     여러 서적의 텍스트들은 온도를 '차고 뜨거운 정도'라는 개념으로 표현한다. 하지만 이는 정량적으로 따질 수 있는
     것이 아니고 측정자의 주관적인 해석도 들어갈 여지가 크다. 따라서 온도를 정량적인 에너지와 연관짓고 해석할
     필요가 있다. 이에 첫째로 열 에너지를 먼저 정의해야 한다.
     우리가 명확하고 정량적으로 열현상을 분석하고자 한다면 에너지 보존에서 학습한 열에너지를 사용하자.
    \textbf{열 에너지란 물질 안에 존재하는 입자 운동 에너지와 결합 에너지의 총합을 의미한다.\footnote{이러한 에너지를 하나도 갖고 있지 
    않은 입자를 상상할 수 있다. 입자가 이러한 상태일 때의 온도를 절대온도 '0K'로 정의한다. 절대온도의 단위는 K이다. 일상생활에서는
    섭씨나 화씨를 주로 사용한다.}} 
    이러한 에너지는 하나의 계(system)에서 보존되어진다. 따뜻한 공기를 이루는 입자의 운동에너지가 차가운 공기를 이루는 입자의 운동에너지보다
     더 크다. 두 공기들이 만나 열적인 평형을 이루게 된다면 이는 서로 충돌에 의해 에너지가 오간 것으로 생각할 수 있다.  
     \marginpar{
      \begin{center}
        \includegraphics[width=4cm]{사진모음/열현상/열에너지-입자의 운동에너지.jpg}
        \captionof{figure}{입자의 운동-열에너지}
      \end{center}}
    
    \subsubsection*{이상기체}
우리는 온도와 같은 입자의 통계적이고 거시적인 거동을 기체를 이루고 있는 분자들의 운동으로 설명할 것이다. 그렇다면 우리는 수많은 종류의 분자들을 대표할
공통적인 모델을 제시할 수 있어야 한다. 이에 상황을 간단히 하기 위해 \textbf{이상기체}라는 것을 정의하고 이를 연역적으로 확장하고 활용한다. 이상기체는 
기체의 운동에너지만을 내부에너지로 따지기 위한 이상기체로 충분히 낮은 밀도에서 기체들은 이상기체와 같이 행동한다. 
\textbf{즉, 기체분자들이 서로 충분히 멀리 떨어져 있어서 상호작용을 할 수 없는 조건에서는 이상적인 상태로 접근해간다.} 
이렇게 낮은 밀도에서 기체들은 다음의 이상기체 상태방정식을 따른다. 

    \begin{defn}[이상기체 상태방정식]

    \begin{equation}\label{eq:ideal gas}
    pV=NkT
    \end{equation}

  \begin{itemize}
    \item  $p$: 기체의 압력
    \item k: 볼츠만(Boltzmann) 상수($=1.38\cross 10^{-23} $J/K)
    \item $V$: 기체의 부피
    \item $T$: 기체의 온도
  \end{itemize}  
  \end{defn}

 



    \subsubsection*{이상기체의 내부에너지}
지금부터는 정량적으로 이상기체를 질점의 물체로 보고 이상기체 입자가 만드는 압력을 계산하여 보고, 더 나아가 온도와 내부에너지의 
관계를 알아보고자 한다. 질량이 $m$이고, 속도가 $\va{v}$인 한 분자가 벽면에 충돌하는 경우를 생각해보자. 충돌은 탄성충돌로 
운동에너지가 보존되고, 상황을 간단히 하기 위하여 x축방향의 속도만 있고 운동량 변화도 해당 방향으로만 있다고 가정한다. 

입자의 선운동량 변화량은
\begin{equation}
  \Delta p_x=(-mv_x)-(mv_x)=-2mv_x
\end{equation}

   \marginpar{
      \begin{center}
        \includegraphics[width=5.0cm]{사진모음/열현상/test01.eps}
        \captionof{figure}{기체분자가 벽면에 작용하는 힘}
      \end{center}}


이다. 그림의 분자는 색칠한 벽을 계속해서 반복적으로 때릴 것이다. 반복되는 충돌 사이의 시간간격 $\Delta t$는 속력 $v_x$의 분자가
반대쪽 벽까지 갔다가 다시 돌아오는데 걸리는 시간이다. 따라서
   \begin{equation}
  \Delta t=\frac{2L}{v_x}
\end{equation}
이에 따라 분자 하나가 색칠한 벽에 전달하는 평균 단위시간당 선운동량은 다음과 같다.
 \begin{equation}
  \frac{\Delta p_x}{\Delta t}=\frac{2mv_x}{2L/v_x}=\frac{mv_x^2}{L}
\end{equation}
이는 뉴턴의 제2법칙($F=\frac{dp}{dt}$)에 의해 입자하나가 벽에 작용하는 힘이다. 전체 힘은

\begin{align}
  &F_{x}=\frac{mv_{x1}^2}{L}+\frac{mv_{x2}^2}{L}+\frac{mv_{x2}^2}{L}+\cdots\frac{mv_{xN}^2}{L}\\
  &=\frac{m}{L^2}(v^2_{x1}+v^2_{x2}+v^2_{x3}\cdots+v^2_{xN}) 
\end{align}

이 때 모든 입자의 속력 제곱 평균인 $(v^2_x)_{avg}$을 이용하여 단위면적당 힘을 정리하여 다음과 같이 수정할 수 있다.


 \begin{itemize}
    \item  $N= nN_A$: $N$은 전체 입자의 갯수, $n$은 몰수, $N_A$는 아보가드로 수
    \item  $p$: 열역학에서 문자 $p$는 운동량이 아닌 압력(pressure)으로 사용한다.
    \item  $M=mN_A$: $m$은 이상기체 입자 하나의 질량이다. 따라서 $M$은 1mol의 질량
    \item  $V=L^3$: V(volume)은 부피이다.
  \end{itemize} 

\begin{align}
 &p=\frac{nmN_A}{L^3}(v^2_x)_{avg}\\
 & mN_A =M, \phantom{o} L^3 =V \textrm{(Volume)}\\
 & \therefore p=\frac{nM(v^2_x)_{avg}}{V}
\end{align}
이를 좀더 일반적인 상황인 3차원에 확장한다면 분자의 속도의 크기는 $v^2=v^2_{x}+v^2_{y}+v^2{z}$이다. 또한 많은 분자들에 의해
통계적으로 각 분자의 각 속도성분의 제곱의 평균값은 모두 같다라고 볼 수 있어, $v^2_x=\frac{1}{3}v^2$이다.\footnote{x,y,z축에 대해서
특정방향의 평균값이 크거나 작은 것이 더 평범하지 않다.} 그리고 $(v^2)_{avg}$
의 제곱근은 제곱평균제곱근(rms)속력이라고 부르고 $v_{rms}$로 표기한다. 이에 따라 최종적으로 벽에 작용하는 압력은 다음과 
같이 쓸 수 있다. 

\begin{equation}
  p=\frac{nMv^2_{rms}}{3V}
\end{equation}

이 식으로 인하여 우리는 기체의 압력이 아주 작은 분자들의 속력에 어떻게 좌우되는지 정량적으로 구하고 연결하였다. 
이를 이상기체 상태방정식($pV=nRT$)과 연결하면 rms속력을 구할 수 있고 이에 따라 우리는 기체분자의 
병진운동에너지를 구할 수 있다. 이 병진운동에너지는 각 입자간의 상호작용이 없는 이상기체의 내부에너지가 된다.

  \begin{defn}[이상기체의 rms 속도, 병진운동에너지, 내부에너지]

    \begin{align}\label{eq:ideal gas's energy}
    &v_{rms}=\sqrt{\frac{3RT}{M}}, \phantom{o} K_{avg}=\frac{1}{2}mv^2_{rms}\\
    &\therefore K_{avg}=\frac{3RT}{2N_A}=\frac{3}{2}kT
    \end{align}

  \begin{itemize}
    \item  $nR=Nk$ : $n$은 몰수, $R$은 기체상수이다.
  \end{itemize}  
  \end{defn}



     \subsection{기체가 한 일}
     계에 대한 설명 

     \subsection{열역학 제1법칙}
         \subsubsection*{등적과정}
        \subsubsection*{등압과정}
        \subsubsection*{등온과정}
        \subsubsection*{단열과정}
     \subsection{열역학 제2법칙}
     \subsection{열효율}



\end{flushleft}

%% 부록 관련 명령
%% \appendix 또는 appendices 환경
%\appendix
%% 부록 면주에 \hparttitlehead, \hchaptertitlehead 를 표시하지 않는다.
\def\partmark#1{\markboth{#1}{#1}}
\def\chaptermark#1{\markright{#1}}
%
\AppendixTitleToToc
\AttachAppendixTitleToSecnum
%\begin{appendices}
\appendix
\appendixpage*
\renewcommand\thechapter{\Roman{APPchapter}}
\renewcommand\thesubsection{\thesection.\arabic{APPsubsection}}
\setcounter{APPchapter}{0}
\chapterstyle{appendixdefaul교}
\renewcommand*\prechapternum{\chapnamefont 부록\ \ 제}
\renewcommand*\postchapternum{\chapnamefont 장}
\renewcommand*\printchapternum{\chapnumfont\thechapter}
%% appendix에서는 chaptersyle을 appendixcompanion,
%% appendixdefault, appendixsection 등으로 지정할 것.
%% 사용자가 새로운 chapterstyle을 설정하려 할 때는
%% appendixXXXX 환경을 새로 만들어야 한다.
%% appendix에서의 절 모양은 \thechapter.\arabic{section}으로
%% 된다. 이것은 renewcommand할 수 있다터



\pagestyle{hangul}
\chapter{수학적 도구들}\label{chap:math in physics}
\chapterprecis{부록에서는 물리학의 논리를 엄밀하게 펼치기 위한 수학에 대해서 설명한다.}
\ResetHangulspace{1.333}{1.2}
\paragraphfootnotes

\section{벡터의 성분과 표현}\label{sec:appsec}
\begin{flushleft}
온도와 같은 \textbf{스칼라}\footnote{스칼라 물리량의 대표적인 예는 온도, 압력, 에너지, 질량, 시간 등이 있으며, 
이들은 방향성이 없다.}는 크기만 있다. 이들은 단위를 가진 숫자로 나타내고, 일반적인 산술과 대수의 규칙을 따른다.
크기와 방향 둘 다 필요한 물리량을 기술하려면 새로운 수학언어인 \textbf{벡터}\footnote{벡터 물리량의 예로는 변위, 
속도, 가속도, 운동량 등이 있다.}가 필요하다. 예를들면 힘, 자기력, 전기장과 같은 물리량을 기술하는데 벡터는 필수적이다. 
가장 간단한 경우의 벡터량은 변위로서 위치의 변화가 변위\footnote{위치는 기준점에 따라 달라진다. 
하지만 변위벡터는 기준점과 관계없이 변화량만 다루기 때문에 기준이 필요치 않아 유용하다. 이러한 관계는 시각과 시간의 차이와 유사하다.}
이다. 이러한 \textbf{벡터는 크기와 방향을 갖는다.}
입자의 위치가 $A$에서 $B$로 변한다면 그림과 같이 변위벡터를
화살표로 나타낼 수 있다. 이 때 벡터는 문자의 위에 화살표($\vec{AB}$) 를 붙여 표현하며 화살표의 길이로 벡터의 크기를 표현한다.
중요한 사실 중 하나는 \textbf{벡터는 크기와 방향이 변하지 않는한 동일한 벡터라는 것}이다. 
벡터는 시작점의 정보를 갖지 않는다는 것은 오래도록 곱씹을 만한 내용이다.  

\begin{figure}[h]
     \centering
     \begin{subfigure}[h]{0.45\textwidth}
         \centering
         \includegraphics[width=\textwidth]{사진모음/벡터의 표현.jpg}
         \caption{벡터의 화살표 표현}
     \end{subfigure}
     \hfill
     \begin{subfigure}[h]{0.5\textwidth}
         \centering
         \includegraphics[width=\textwidth]{벡터의 동등.PNG}
         \caption{벡터의 동등}
      \end{subfigure}
        \caption{벡터의 표현과 동등}
\end{figure}
하지만, 이러한 표현은 기하학적인 표현만 할 수 있기 때문에 보다 정밀하게 좌표계를 이용한 성분표현을 할 수 있다면 좋을것 같다.
그럴려면 \textbf{단위벡터}를 도입해야한다. 단위벡터는 크기가 1이며 특정한 방향을 갖는 벡터이다. 단위벡터를 사용하는 목적은 
벡터의 방향을 나타내기 위한 것이다. 우리가 주로 다루는 직각좌표계(cartesian coordinate)에서 x,y,z축 양의 방향을 향하는 단위벡터를 각각 \
$\vu{i},\vu{j},\vu{k} $\footnote{다른 서적에서는 $\vu{x},\vu{y},\vu{z} $로 표기하기도 하며 단위벡터에는 
모자기호($\vu{\quad}$)를 붙여준다.}으로 표기한다. 단위벡터는 다른 벡터들을 표기할 때 유용하다. 예컨데 그림의 벡터 $\va{A}$
를 다음과 같이 나타낼 수 있다. 
\begin{figure}[h]
 \centering
  \includegraphics[width=5cm]{벡터의 성분표현.PNG}
  \caption{벡터의 성분 표현}\label{벡터의 성분표현}
\end{figure}

\begin{equation}
  \va{A} = \va{A_x}+\va{A_y} =A_x\vu{i}+A_y\vu{j} =4\vu{i}+3\vu{j}\footnote{($A_x,A_y$)=(4,3)로 표현하기도 하니 
  알아두자. 이 때 벡터의 크기는 화살표의 길이로 $\abs{\va{A}}=A=\sqrt{A_x^2+A_y^2}=5 $이다.  }
\end{equation}

위의 그림 \ref{벡터의 성분표현}에서 $A_x\vu{i}, A_y\vu{j}$는 벡터로서 벡터성분이라고 한다. 이 벡터의 성분은
직관적으로 알 수 있듯 각 좌표축에 벡터를 투영시킨것으로 x축과 벡터가 이루는 각도를 $\theta$라고 할 때, 
\begin{equation}
  A_x=A\cos\theta, A_y=A\sin\theta
\end{equation}

  로 표기할 수 있다. $A_x,A_y$는 스칼라로서 벡터 $\va{A}$의 스칼라성분이라 한다. 




  \begin{task}
공항을 떠난 비행기가 공항에서 설정한 직교좌표계의 양의 y축방향에서 30도만큼 양의 x축 방향으로 기운만큼 공항으로부터 200km를 날아갔다.
비행기의 변위벡터를 (cartesian coordinate)로 표현하고, 그 변위벡터의 단위방향벡터를 구하시오.   

\end{task}
\end{flushleft}







\begin{flushleft}
  


\subsection{벡터의 덧셈}
벡터는 벡터만의 연산규칙이 있다. 화살표를 이용한 기하학적인 벡터의 연산규칙과 성분을 이용한 벡터의 계산의 연산규칙은 동등하다. 
일반적으로 전문적이고 정량적인 분석을 하기위해서는 성분을 이용한 벡터의 계산에 익숙해질 필요가 있다. 
벡터의 성분에 대해서는 위의 section에서 배웠으니 벡터의 덧셈과 뺄셈은 직관적으로 이해하기 쉬울것이다. 
하지만 벡터의 곱셈은 스칼라곱과 벡터곱으로 나뉘어져 있다. 
example을 풀고 이 문서가 참조한 Halliday의 일반물리학 예제 연습하기를 추천한다. 

벡터의 덧셈에는 기학학적인 뎃셈과 성분을 이용한 덧셈이 있다.


\begin{figure}[h]
  \centering
  \begin{subfigure}[h]{0.4\textwidth}
      \centering
      \includegraphics[width=\textwidth]{벡터의 합.PNG}
      \caption{$\va{c}=\va{a}+\va{b}$}
      \label{그림A}
  \end{subfigure}
  \hfill
  \begin{subfigure}[h]{0.4\textwidth}
      \centering
      \includegraphics[width=\textwidth]{벡터의 합2.PNG}
      \caption{ $(\va{a}+\va{b})+\va{c}=\va{a}+(\va{b}+\va{c})$ }
      \label{그림B}
   \end{subfigure}
     \caption{벡터의 기하학적인 합}
  \end{figure}

 그림 \ref{그림A}은 $\va{a}, \va{b}$가 더해져 $\va{c}$가 되는것을 나타낸다. 더하고자하는 벡터의 머리와 꼬리를 이어 합을 
 만들수가 있다. 벡터의 기하학적인 합은 다음과 같은 벡터방정식으로 표기할 수 있다.
  \begin{equation}
  \va{c}=\va{a}+\va{b}  
\end{equation}
그림 \ref{그림A}을 보면 다음의 특성을 보인다. 덧셈순서와 무관하다. 
  \begin{equation}
\va{a}+\va{b}=\va{b}+\va{a}   
\end{equation}
\noindent 둘째, \ref{그림B}를 보면, 두 개 이상의 벡터들을 어떤 순서로도 더할 수 있다. 
\begin{equation}
  (\va{a}+\va{b})+\va{c}=\va{a}+(\va{b}+\va{c}) 
\end{equation}

벡터의 덧셈은 교환법칙과 결합법칙을 만족한다.



  벡터 성분으로 벡터 더하기는 이보다 직관적으로 이해하기 쉽다.


 \begin{align}
  & \va{c}= \va{a}+\va{b} \\
  & \va{a}= a_x\vu{i}+a_y\vu{j}+a_z\vu{k} \\
  & \va{b}= b_x\vu{i}+b_y\vu{j}+b_z\vu{k} 
   \end{align}

  일때, 
  \begin{align}
    &\va{c}= (a_x\vu{i}+a_y\vu{j}+a_z\vu{k}) + (b_x\vu{i}+b_y\vu{j}+b_z\vu{k}) \\
    &= (a_x+b_x)\vu{i}+(a_y+b_y)\vu{j}+(a_z+b_z)\vu{k}
    \end{align}
벡터의 더하기와 빼기는 같은 성분끼리 더하거나 빼주면 되어 간단하다. 
\newpage



  \begin{task}

1)변위 $\va{a}$와 $\va{b}$의 크기가 각각 3m와 4m이며, $\va{c}=\va{a}+\va{b}$이다. 
변위 $\va{a}$와 $\va{b}$의 가능한 방향을 모두 고려할 때, $\va{c}$의 최대크기와 최소크기를 구하시오.\\
\phantom{text}\\
\phantom{text}\\
\phantom{text}\\
\phantom{text}\\
2)다음에 제시된 벡터들의 합을 표시하시오.\\
\phantom{text}\\

    \centering{
    {\IfFileExists{사진모음/벡터의 합성.png}
    {\includegraphics[width=0.6\linewidth]{사진모음/벡터의 합성.png}
  \label{fig:벡터화살표 합성}}%
    {\rule{\linewidth}{4cm}}}
    }

    \phantom{text}\\
    \phantom{text}\\
    \phantom{text}\\

\begin{flushleft}
3)벡터 $\va{A}= 2\vu{i}-3\vu{j}+6\vu{k}$와 $\va{B}= \vu{i}+2\vu{j}-3\vu{k}$ 가 있다. 다음을 구하라.
    \\
    \begin{tasks}[label=(\arabic*)](1)
      \task $A+B$ \\
      \task $\abs{\va{A}+\va{B}}$\\
      \task $2\vec{A}-3\vec{B}$  \\
   \end{tasks}
  \end{flushleft}

  \end{task}

\subsection{벡터의 곱셈}

벡터 곱하기는 세 가지 방법이 있으며 보통의 대수 곱하기와는 전혀 다르다. 그 세가지는 
\begin{itemize}
\item 벡터에 스칼라 곱하기
\item 스칼라곱 
\item 벡터곱 
\end{itemize}
으로 나눌 수 있다. 

벡터에 스칼라 곱하기는 간단하다. 
벡터 $\va{a}$ 스칼라 s를 곱하는 것은 벡터 $\va{a}$의 크기에 s의 절대값을 곱한값으로 $s\va{a}$로 나타내며 방향은 
s의 부호에 따라 s가 양이면 같은 방향 음이면 반대방향을 가리키게 된다. 

\begin{figure}[h]
  \centering
  \includegraphics[width=0.5\textwidth]{벡터의 실수배.PNG}
  \caption{벡터의 실수배}
\end{figure}

\subsubsection{스칼라곱}
두 벡터 $\va{a}, \va{b}$의 스칼라곱은 $\va{a}\vdot\va{b}$로 표기하고 다음과 같이 정의한다. 
\begin{equation}
  \va{a}\vdot\va{b}=ab\cos\theta
\end{equation}
이 때 $\theta$는 벡터 $\va{a}$의 방향과 $\va{b}$의 방향 사이의 각도이다. \textbf{우변은 스칼라값들뿐이므로 이를 벡터의 스칼라곱
점곱이라고 한다.} 성분을 강조하면 다음과 같이 표기할 수 있다. 
\begin{equation}
  \begin{split}
    \va{a}\vdot\va{b}&= (a_x\vu{i}+a_y\vu{j}+a_z\vu{k}) \vdot (b_x\vu{i}+b_y\vu{j}+b_z\vu{k}) 
  \\& = (a_x b_x)+(a_y b_y)+ (a_z b_z)
  \end{split}
\end{equation}

\begin{task}
두 벡터 $\va{a}=3.0\vu{i}-4.0\vu{j}$와 $\va{b}=2.0\vu{i}-3.0\vu{k}$의 사이각 $\theta$의 cos값을 구하시오.

\end{task}




\subsubsection{벡터곱} 
  두 벡터 $\va{a}, \va{b}$의 벡터곱은 $\va{a}\cp\va{b}$로 표기하고 그 크기는 다음과 같이 정의한다. 
\begin{equation}
  \abs{\va{a}\cp\va{b}}=ab\sin\theta
\end{equation}
이 때 $\theta$는 벡터 $\va{a}$의 방향과 $\va{b}$의 방향 사이의 작은 각도이다. 이 때 스칼라곱과 큰 차이점이 있다.
\textbf{우변은 벡터값이므로 방향이 있다. 이 방향은 $\va{a},\va{b}$
가 이루는 평면에 수직이며 구체적으로는 그림                                                                                                                                                                                                                                                                                                                                                                                                                                                                                                                                                                                                                                                    \ref{오른손규칙}의 오른손 규칙을 따른다.\footnote{오른손 법칙은 "벡터곱을 외자" 활동을 통해 익혀보자.}}
 따라서 벡터곱은 두 벡터의 순서가 중요하다. 
\marginpar{
  \begin{center}
  \includegraphics[width=4cm]{오른손법칙.png}\captionof{figure}{오른손 규칙}\label{오른손규칙}
  \end{center}} 

\begin{equation}
  (\va{a}\cp\va{b})=  -(\va{b}\cp\va{a})
\end{equation}

따라서 벡터곱에서는 교환법칙이 성립하지 않는다.
성분을 이용하려면 단위벡터간의 벡터곱을 알아야한다. 

\begin{equation}
  \begin{split}
    \va{i}\cp\va{j}=\va{k}=-\va{j}\cp\va{i}
  \end{split}
\end{equation}

\marginpar{
  \begin{center}
  \includegraphics[width=4cm]{단위벡터의 외적.png}\captionof{figure}{단위벡터의 외적}
  \end{center}} 


이를 알고 두 벡터의 벡터곱을 성분별로 표현하면 아래와 같다.
\begin{equation}
  \begin{split}
    \va{a}\cp\va{b}&= (a_y b_z-b_y a_z)\vu{i}+(a_z b_x - b_z a_x)\vu{j}+(a_x
    b_y-b_x a_y)\vu{k} 
  \end{split}
\end{equation}

백터곱은 비교적 복잡하므로 행렬식을 이용한 방법을 익히는 것을 추천한다. 



\begin{task}
물리학에서 지레의 회전중심에서 회전팔에 힘이 작용하는 지점까지의 크기와 방향을 $\va{r}$, 회전팔에 작용하는 힘의 
크기와 방향을 $\va{F}$로 나타낼 때, 지레가 받는 돌림힘을 다음과 같이 정의한다. 
\begin{equation}
  \vec{\tau}= \va{r} \cross \va{F} [N\cdot m]
\end{equation} 

지레의 회전팔이 y-z 평면에서 회전가능하며 $\va{r} =4\vu{j}+3\vu{k},  \va{F} =-5\vu{i} -3\vu{j}$ 일 때 지레가 받는
돌림힘의 벡터를 구하시오. 
\end{task}
\end{flushleft}

\marginpar{
  \begin{center}
  \includegraphics[width=5cm]{사진모음/돌림힘_1.jpg}\captionof{figure}{돌림힘}\label{돌림힘}
  \end{center}} 

\include{미분과 적분.tex}

\section{다양한 좌표계}

\begin{flushleft}
더 높은 차원의 다양한 좌표계를 명확히 이해하고 활용하는 것 자체로 복합적인 물체의 운동을 이해할 수 있다.
  
\subsection{데카르트 좌표계}
르네 데카르트가 발명한\footnote{프랑스의 수학자이자 철학자로 천장을 날아다니며 옮겨붙은 파리를 통해 영감을 얻어 좌표계를 발명했다.}
데카르트 좌표계(Cartesian coordinate frame)는 물리학에서 가장 흔하게 볼 수 있는 좌표계로 
문제를 다루는 좌표공간은 x,y,z의 서로 직교하는 좌표로 정의한다. 
데카르트 좌표계는 나타내는 대상이 평행 이동에 대한 대칭을 가질 때 유용하나, 회전 대칭 등 다른 꼴의 대칭은 쉽게 나타내지 못한다.
\marginpar{
  \begin{center}
  \includegraphics[width=4cm]{데카르트 좌표계.jpg}\captionof{figure}{cartesian coordinate}
  \end{center}} 


\subsection{극 좌표계}
극 좌표계(polar coordinate system)는 평면 위의 위치를 각도와 거리를 써서 나타내는 2차원 좌표계이다. 극 좌표계는 두 점 사이의 관계가
각이나 거리로 쉽게 표현되는 경우에 유용하다. 데카르트 좌표계에서 삼각함수로 복잡하게 나타나는 관계가 극 좌표계에서는 간단하게 표현되는
경우가 많다. 극 좌표는 r로 나타내는 반지름 성분과 $\theta$ 로 나타내는 각 성분으로 이루어져 있다. (r, $\theta$) 반지름 성분 r은 좌표계의 원점
에서의 거리를 나타내고, 각 성분은 x축의 양의 뱡향을 0$^{\circ}$로 반시계 방향으로 잰 각의 크기를 나타낸다.  
\marginpar{
  \begin{center}
  \includegraphics[width=4cm]{극좌표.jpg}\captionof{figure}{polar coordinate}
  \end{center}} 

  \begin{task}
직교 좌표계보다 극 좌표계가 더 활용하기 좋은 경우를 알아보자. Eq. \ref{eq:universal gravity}에서 
배운 중력법칙을 적용해보자. 질량이 큰 행성 A를 중심으로 공전하는 행성 B가 있다. A를 고정하여 원점으로 할 때, 
행성 B의 변위벡터가 극좌표계로 $\va{r_1}=(10, \frac{\pi}{3})$, $\va{r_2}=(10,{\pi})$ 일 때 행성 B가 
A로부터 받는 중력 벡터를 직교좌표계와 극좌표계로 각각 표현하시오. (단, 계산을 단순히 하기 위해 행성 A의 질량을 100kg, 행성 B의 질량을 1kg,
중력상수의 값을 1로 하자.)
\end{task}


\subsection{구면 좌표계}
구면 좌표계(spherical coordinate system)는 공간 위의 위치를 각도와 거리를 써서 나타내는 3차원 좌표계이다. 3차원 공간 상의 점들을 나타내는 좌표계의 하나로, 보통 
(r,$\theta,\phi$)로 나타낸다. 원점에서의 거리 r은 0부터 $\infty$ 까지, 양의 방향의 z축과 원점과 위치를 이루는 선이 이루는 각도 $\theta$ 는 0부터 
$\pi$까지, z축을 축으로 양의 방향의 x축과 이루는 각 $\phi$ 는 0부터 $2\pi$  까지의 값을 갖는다. $\theta$ 는 위도로, $\phi$ 는 경도로 표현되는 경우도 있다.
\marginpar{
  \begin{center}
  \includegraphics[width=4cm]{Spherical_coordi.png}\captionof{figure}{spherical coordinate}
  \end{center}} 


  \end{flushleft}


\chapter{물리학 실험}\label{chap:Experiment in physics}
\chapterprecis{다양한 실험은 물리학 개념을 정립하는데 중요한 거름이 된다. 실험 보고서의 작성법과 유의사항에 대해서 설명한다.}

\begin{flushleft}
\section{실험보고서 작성법}
\begin{exper}{ \Large {실험보고서의 형식과 주안점} }
  \begin{enumerate}[(1)]\setlength{\itemsep}{10mm}
    \item 실험제목 :실험 제목, 실험 일자, 실험실 조건, 공동 실험자의 이름을 보고서 시작에 기록한다.

    \item 준비물 : 실험에 사용한 준비물들을 상세하게 기록한다. 

    \item 이론적 배경 : 실험과정에 대한 이론을 기술한다. \textbf{이론적 배경에서 기술한 내용은 실험 결과를 설명할 수 있는 것을 쓴다.}
   
    \item 실험 방법 : 실험 장치와 실험 과정, 구체적인 실험 방법들을 순서대로 기술한다. 실험 구성에서 실험 구성자의 창의성이 가장 돋보인다.
   
    \item 실험 결과 : 실험 과정에서 얻은 자료를 기록하는 것으로 \textbf{표, 그래프, 그림} 등을 중심으로 측정자료를 한눈에 알아보기 쉽게 작성하며
    측정값 사이의 관계가 명확해야한다. 즉 \textbf{조작변인과 종속변인의 관계가 명확히 보이는 것}이 핵심이다. 

    \item 결론 및 토의 :실험결과를 토대로 실험 목적과 관련하여 가설에 대한 결론과 실험을 통해 알게된 점을 명확하게 기술한다. 또한, 실험 중 
    가장 중요한 오차를 찾고, 이를 개선할 수 있는 방안을 제시하는 것이 좋다. 그리고 오차는 구체적으로나 또는 구체적으로 제시할 수 있어야하고
    그렇지 않다면 기술하지 않는것이 더 좋다. 

\end{enumerate}
\end{exper}

\subsection{유효 숫자}



\marginpar{
  \begin{center}
  \includegraphics[width=5.5cm]{사진모음/그래프 나타내기.png}\captionof{figure}{그래프 나타내기}
  \end{center}}
  

\subsection{그래프로 나타내기}
      \marginpar{
      \begin{center}
      \includegraphics[width=5.5cm]{사진모음/여러개의 선 표기.png}\captionof{figure}{여러개의 데이터 표기}
      \end{center}} 
    \begin{enumerate}[가.]\setlength{\itemsep}{1mm}
     \item 적절한 제목을 선택하기 - 조작 변인, 종속 변인, 중요한 통제 변인을 포함하여 짧게 서술한다. 
     \item 변수에 맞는 적절한 축을 선택하기 - 조작 변인은 가로축에, 종속 변인은 세로축에 위치한다. 
     \item 축에 변수의 이름을 붙이고, 단위를 나타내기
     \item 축에 눈금 선택하기 - 눈금 간격을 균등하게 나눈다. 가로축과 세로축의 눈금 간격이 같을 필요는 없다. 눈금이 반드시 0에서 시작할 필요는 없다.
     \item 변수가 연속적일 때에는 데이터 값을 연결하는 적절한 선을 그릴 수 있다. 이 때의 이 선을 추세선이라고 한다. 이 때의 추세선은 일치하지 않는 점을 
     울퉁불퉁하게 그리지 말고, 그 점을 가능한 근접하게 지나가는 직선이나 곡선이 되도록 그려넣어야 한다. 또한 하나의 그래프에 여러개의 선을 그려야 할 경우
     에는 각 측정점을 $\circledcirc,\bigtriangleup, \diamond$와 같이 서로 다르게 표시하는 것이 좋다.
 
     \item 측정값의 불확실도(오차)를 알고 있는 경우에는 오차막대를 사용하여 그 크기를 나타낸다.
     
    \end{enumerate}
  
\marginpar{
  \begin{center}
  \includegraphics[width=5.5cm]{사진모음/오차막대.png}\captionof{figure}{오차막대}
  \end{center}} 




\end{flushleft}

%\restorechapsec \showcommand{restorechapsec}
%% 만약 appendix가 문서의 가장 끝에 오는 것이 아니라면
%% 이 명령을 appendix 이후에 실행해준다.
%% 아래와 같이 appendices 환경을 쓰는 경우에는
%% 환경을 종료하기만 하면 된다.
%\end{appendices}

%%% 본문의 끝.
\backmatter
\chapterstyle{demo}

%% \bibintoc 하면 참고문헌이 목록에 나온다.
%% 기본값이므로 별도로 설정할 필요는 없다.
%\bibintoc
\renewcommand\prebibhook{}
\begin{thebibliography}{00}
\bibitem{Halliday} Halliday\&Walker\&Resnick, "Principles of Physics"
\bibitem{권재술} 권재술 외 9명, "과학교육론"
\bibitem{과학기술정보통신부} 과학기술정보통신부, 학국과학창의재단, KAIST 과학영재교육연구원 "AP 일반물리학1 교사용 지도"
\end{thebibliography}

\indexintoc 
\renewcommand\preindexhook{%
  찾아보기는 테스트를 위해서 임의의 단어들로 선정되었다.
  \bigskip}
\printindex

%% memoir에서는 \listof... 명령을 아무데나 선언할
%% 수 있다. 신기하다.


\end{document}