\chapter{에너지와 열역학}
\chapterprecis{\noindent 이 장에서는 에너지에 대해 배우고 열현상을 해석한다.}

\begin{flushleft}
\section{일과 에너지}
    
에너지라는 단어는 고대 그리스어 energia에서 en(안, 내부)+ ergon(일) 파생되었다.
 기원전 4세기 아리스토텔레스의 기록에서 사용되었고, 
 현대의 엄밀한 정의와 달리 에네르게이아는 포괄적인 개념이었다.\footnote{여기서 말하는 것은 "에너지가 넘쳐보인다."와 같은 표현이 넓은 의미의 
 에너지로 활기, 행복과 같은 정량적으로 비교가능하지 않는 양을 말한다.}

17세기 후반에 라이프니츠(Gottfried Leibniz)는 공이 벽면에 충돌해서 반대로 이동하는 경우를 집중해보았다. 
당구공과 같이 공이 충분히 탄성이 있고 벽면이 딱딱하다면 공은 속력이 변하지 않으며 되튀어져 나온다. 
운동량 보존을 이용한다면,
공의 속도는 벡터량임으로 되튀어지며 운동량이 2배만큼 변한다. 충돌 된 
벽면의 질량은 너무 커 벽면과 공의 계의 운동량 보존으로 상황을 분석하기엔 적절하지 않아보인다.
라이프니츠는 똑같은 속력으로 되튀어 나오는 공의 운동과정
\footnote{이런 에너지의 손실이 없는 충돌 과정을 특별히 탄성 충돌이라 한다.}에서 
 \marginpar{
  \begin{center}
    \includegraphics[width=4cm]{사진모음/아리스토텔레스.jpg}
    \captionof{figure}{Aristoteles\\(B.C 384~B.C322)}
  \end{center}}
속력의 양으로 표현할만한 보존되는 스칼라 양이 필요하다고 생각했다. 

  %  \marginpar{
  %     \begin{center}
  %       \includegraphics[width=5.0cm]{사진모음/에너지/벽과의 충돌.eps}//eps파일은 영어 이름이어야만 한다. 
  %       \captionof{figure}{벽과의 탄성 충돌}
  %     \end{center}}
      
\begin{figure}[h]
\centering
   \includegraphics[width=5cm]{사진모음/에너지/crack.eps}
   \caption{벽과의 탄성 충돌}
\end{figure}

라이프니츠는 물체의 질량과 속도의 제곱의 곱으로 나타나는 새로운 스칼라 물리량(vis viva)을 제안했다.  
이러한 스칼라 물리량은 후대의 과학자들이 다듬어 에너지라는 보존되어지는 양으로 정의했다. 
보존되어지는 스칼라량이란 마치 계좌에 쌓여있는 돈처럼 지출을 할 수 있지만 
그 가치의 총량은 변하지 않는 무엇인가로 생각할 수 있다. 이러한 에너지는 다른 형태의 에너지로 전환되지만 그 총량은 변하지 않는다.
\footnote{이러한 에너지는 최종적으로는 가장 변환하기 힘든 형태인 열에너지로 전환된다.} 



\subsection{운동에너지}
\marginpar{
  \begin{center}
    \includegraphics[width=4cm]{사진모음/Thomas_Young_(scientist).jpg}
    \captionof{figure}{Thomas Young\\(1773~1829)}
  \end{center}}
1807년에 Thomas Young은 현대적인 의미에서 라이프니츠의 활력(vis viva) 대신에 \\"에너지"라는 용어를 처음으로 사용했다.
코리올리(Gustave-Gaspard Coriolis)는 1829년에 "운동 에너지"를 현대적인 의미로 설명했다.  

\[
  \begin{split}
  F &= m \dv{v}{t}\\
  \vec{F}\cdot d \vec{x} &= m \textrm{d}v \cdot \dv{\vec{x}}{t}\\
  \int_{x_i}^{x_f} \vec{F} \cdot \textrm{d} \vec{x} &= 
  \int_{v_i}^{v_f}m\vec{v} \cdot \textrm{d}\vec{v} = \int_{v_i}^{v_f}mv \textrm{d}v
  =\frac{1}{2}mv_f^2-\frac{1}{2}mv_i^2
  \end{split}
\]

운동에너지 K(kinetic energy)는 물체의 운동상태와 관련된 에너지로 질량 $m$인 물체의 속력 $v$가 광속보다 훨씬 작을 때,
운동에너지를 다음과 같이 정의한다. \footnote{보다 포괄적인 운동에너지는 상대론의 논리에 따라 정의된다.}


\marginpar{
  \phantom{h}
  \phantom{h}
  \begin{flushleft}
   운동에너지의 단위는 J이다. 더 확장하여 에너지, 열량, 일의 단위도 J이다. 
   $1 joule=1$J$=1$kg $\cdot$$m^2/s^2$
  \end{flushleft}}


\begin{defn}[운동에너지]
  \begin{align}
 &K=\frac{1}{2}mv^2
\end{align}
\centering{\textbf{물체의 질량이 클수록 더 빠른 속력을 가질수록 운동에너지가 크다.}}
\end{defn}




\subsection{일}
수평 $x$축으로 뻗어있는 마찰 없는 줄을 따라 미끄러지는 물체를 생각해보자. 줄과 각도 $\phi$를 이루는 방향으로 작용하는 일정한 
힘 $\va{F}$가 물체를 가속시키고 있다. 뉴턴의 제 2법칙에 따라 물체의 운동을 분석한다면 다음과 같이 나타낼 수 있다. 

\begin{figure}[h]
  \centering
   \includegraphics[width=5cm]{사진모음/에너지/work.eps}
   \caption{일-에너지}
 \end{figure}

 \[
\begin{split}
  &F\cos{\phi}= ma_{x}\\
  &F_{x}=ma_{x}
\end{split}
\]
일정한 힘이 작용하는 등가속도 운동의 상황임으로 등가속도 운동공식을 사용하면 다음과 같이 정리가 가능하다.
\[
\begin{split}
  &v^2={v_0}^2+2a_{x}d\\
  &\frac{1}{2}mv^2-\frac{1}{2}mv_{0}^2=F_{x}d=F\cos{\phi}d=\va{F}\cdot \va{d}
\end{split}
\]
이를 통해 일정한 \textbf{힘이 작용한 방향과 같은 방향의 변위의 곱만이 운동에너지의 변화와 같음}을 알 수 있다.
이는 두 벡터의 스칼라곱과 의미가 같다. 즉,
 \textbf{물체의 에너지를 변화시키는 힘과 그 물체의 변위의 스칼라곱은 에너지를 변화시키고 이를 "일"이라고 한다.} 
이를 정리하면 다음과 같이 정의할 수 있다.

\begin{defn}[일정한 힘이 한 일]
  \begin{align}
 &if \phantom{h}F=const\\
 &W =\va{F}\cdot \va{d}\\
 &=K_f-K_i =\frac{1}{2}mv^2-\frac{1}{2}m{v_0}^2
\end{align}
\end{defn}

\marginpar{
  \begin{center}
    \phantom{a}\\
    \phantom{a}\\
    \phantom{a}\\
    \includegraphics[width=5.5cm]{사진모음/에너지/storm.eps}
    \captionof{figure}{Task figure}
  \end{center}}

\begin{task}[일정한 힘이 한 일]
  \begin{flushleft}
 
  강풍이 부는 와중에 미끄러운 바닥 위의 물체가 $\va{d}=(-2m)\vu{i}$만큼 움직였다. 
  이때 강풍은 물체에 $\va{F}=(3N)\vu{i}+(-3N)\vu{j}$의 
  일정한 힘을 
  지속적으로 작용하고 있다. \\
  \phantom{h}\\
  1) 물체의 변위 동안 강풍이 물체에 한 일은 얼마인가?\\
\indent

  2) 이동하기 시작할 때 나무 짐짝이 10J의 운동에너지를 갖고 있다면 이동이 끝난 후의 운동에너지는 얼마인가?

\end{flushleft}
\end{task}



좀 더 일반적인 상황은 힘이 변화하는 경우일 것이다. 상황을 간단히 하기 위해 x축방향으로 작용하는 변화하는 힘과 
힘이 작용하는 물체를 생각해보자. 물체가 $x_i$에서 $x_f$까지 이동하는 경우에 짧은 구간에 대해서는 힘이 일정하다고 생각할 수 있다.
그리고 그 짧은 경우들을 더하면 전체 변위동안에 해준 일을 구할 수 있다. 이를 수식으로 전개하면 아래와 같다. 

\[
\begin{split}
 &W=\sum\Delta W_j=\sum F_{avg}\Delta x \\
 &W = \lim_{\Delta x \to 0} \sum F_{avg} \Delta x\\
 &= \int \; \vec{F} \; \textrm{d}\va{x} 
\end{split}
\]


\marginpar{
  \begin{center}
    \includegraphics[width=5cm]{사진모음/힘과 변위의 적분.png}
    \captionof{figure}{힘이 한 일은 F-x그래프의 넓이와 같다.}
  \end{center}}

이는, \textbf{$F-x$ 그래프의 넓이의 의미와 같다.} 따라서 변화하는 힘이 한 일은 확장하여 다음과 같이 정의할 수 있다. 

\begin{defn}[변화하는 힘이 한 일]
\[ 
  \textrm{if, F not const}
\]
\begin{align}
 &W =\int_{x_i}^{x_f}\va{F} \cdot d\va x\\
 &=K_f-K_i
\end{align}

\end{defn}


\marginpar{
  \begin{center}
    \phantom{a}\\
    \phantom{a}\\
    \phantom{a}\\
    \includegraphics[width=5.5cm]{사진모음/에너지/transfomating.eps}
    \captionof{figure}{Task figure }
  \end{center}}

\begin{task}[변하는 힘이 한 일]
  \begin{flushleft}
2kg의 토막이 힘이 가해짐에 따라 마찰이 없는 수평면의 바닥에서 미끄러지면서 $x_1=0$에서 출발하여 $x_3=6m$에서 끝난다. 토막이 움직이면서 힘의 크기는
그래프와 같이 수평 위치에 따라 변한다. 힘의 방향과 초기속도의 방향은 수평면의 양의 방향으로 같다.
\phantom{a}\\
\phantom{a}\\
\textbf{Q.} $x_1$에서 토막의 운동에너지는 $K_1=36$J이다. $x_1=0, x_2=4.0m, x_3=6m$에서 토막의 속력을 구하여라. 
\end{flushleft}
\end{task}


\marginpar{
  \begin{center}
    \includegraphics[width=5.5cm]{사진모음/힘-변위그래프.png}
    \captionof{figure}{Task Graph}
  \end{center}}



\newpage
\subsection{퍼텐셜에너지와 역학적 에너지의 보존}
\subsubsection*{퍼텐셜에너지}
 중력을 설명하기 위해 section\ref{sec:gravity}에서 도입했던 던져진 토마토를 다시한번 생각해보자. 토마토가 최고 높이에서 떨어질 때 
 중력이 토마토에 한 일은 토마토의 운동에너지의 증가값이 된다. 중력의 방향($\va{F}$)과 변위($\va{s}$)의 방향이 같기 때문이다. 
 따라서 운동에너지의 변화는 양이된다.
하지만 이 때에 지구와 토마토 계에 대해서 에너지는 보존되기를 기대한다. 운동에너지가 증가하는만큼 무엇이 감소했을까? 
보존을 위해\footnote{보존이라는 개념이 너무 당연하게 들어오는게 거부감 들수도 있다. 실제로 에너지 보존 법칙은 19세기초에나 처음으로 가정되었다. 하지만 만물의 상황에 대한 통찰과 전개는 "보존"과 "대칭"의 아이디어로 생겨났다.} 
이에 꼭맞는 \textbf{중력이 한일의 크기에 음(-)을 붙여 에너지를 정의하고 이를 퍼텐셜에너지의 변화량으로 정의하여보자.}

\marginpar{
  \phantom{h}
  \begin{center}
  \includegraphics[width=4cm]{사진모음/에너지/tomato.eps}\captionof{figure}{중력을 받는 토마토의 운동}
  \end{center}} 

  \begin{defn}[중력 퍼텐셜에너지와 퍼텐셜에너지의 정의]
  \begin{align}
 & -W_{grv}= \Delta U\\
 & =- \int_{0}^{h} \va{F}_{grv} \cdot\ d \va{s}\\
 & = -(-mg)\vu{j} \cdot h\vu{j} =mgh  
\end{align}
\centering 
\textbf{중력이 한일의 음의 값을 중력 퍼텐셜 에너지로 정의한다.}
\end{defn}


\subsubsection{역학적에너지}
그리고 이러한 퍼텐셜에너지($U$)와 운동에너지($K$)를 합쳐 역학적 에너지($E_{mec}$)라고 정의하여 보자. 

\begin{equation}
  E_{mec} =K+U
\end{equation}
  

그렇다면 
토마토를 중력이 작용하는 공간에서 수직으로 던졌을 때 에너지 보존을 설명할 수 있다. 
상승할 때에는 중력이 한일의 역수만큼인 퍼텐셜에너지가 증가하고 이에 따라 운동에너지는 감소한다. 반대로 
하강할 때에는 운동에너지는 증가하는 반면 꼭 그만큼의 퍼텐셜에너지가 줄어든다. 따라서 운동에너지와 퍼텐셜 에너지의 합은 일정하다. 
\newpage

\begin{defn}[중력장에서 역학적 에너지의 보존]
  \begin{align}
    & W_{grv}=\Delta K ,\qquad -W_{grv}= \Delta U \\
    & \Delta K = -\Delta U\\
    & =K_f-K_i= -(U_f-U_i)\\
    & E_{mec}=K_i+U_i=K_f+U_f
\end{align}
\centering
\textbf{중력장 내에서 계의 역학적에너지는 보존된다.}
\end{defn}

\subsubsection*{보존력}
이와 같이 에너지 보존을 위해 도입한 퍼텐셜 에너지는 \textbf{특정한 힘들에 의해서만 정의할 수 있다. 중력과 같은 이러한 힘을 보존력이라고 한다.} 
\begin{equation}
  \Delta U =-\int_{x_i}^{x_f} \vec{F_{con}} \cdot \textrm{d}\vec{s}
\end{equation}

보존력(Conservative Force)이란 힘에 의한 일을 구하였을 때, 그 일의 양이 경로에 무관하고 오직 물체의 처음 위치와 나중 위치에만 의존하는 힘이다. 
\footnote{보존력에 의한 일은 경로에 무관하므로 위치에만 의존하는 함수의 차로 정의할 수 있고 그런 맥락에서 '위치에너지'라고도 한다.}

\marginpar{
   \phantom{a}\\
    \phantom{a}\\
  \begin{center}
  \includegraphics[width=5.5cm]{사진모음/훅의 법칙.PNG}\captionof{figure}{훅의 법칙}
  \end{center}} 

\begin{task}[탄성퍼텐셜 에너지]
    \begin{flushleft}
  대표적인 보존력인 탄성력을 이용하여 탄성퍼텐셜에너지를 정의하고 이를 이용하여 보자. 탄성력은 Robert Hooke에 의해서 처음 정의되었다. 
  \begin{equation}
    \vec{F_x} = -k\va{x}
  \end{equation}
  옆의 그림과 같이 용수철이 늘어나거나 줄어든 변위 x와 힘 F는 항상 방향이 반대이다.\footnote{이러한 맥락에서 탄성력을 복원력이라고도 한다.}
  그래서 용수철 상수 $k$앞에는 (-)음의 부호가 붙는다. 
  이때 용수철 상수는 용수철의 탄성을 나타내는 양이다. $k$가 크면 용수철의 탄성이 더 크다. 

    \begin{tasks}[label=(\arabic*)](1)
      \task 용수철의 탄성 퍼텐셜에너지를 정의하라. \\
      \task 질량 $m=1$kg의 토막이 마찰없는 수평면을 따라 속력 $2m/s$로 움직이다. 용수철 상수 $20N/m$의 용수철에 부딪혀 용수철을 압축한다. 
      토막이 용수철에 의해 순간적으로 멈추었을 때 용수철의 압축된 길이 $d$를 구하여라.\\
   \end{tasks}
\end{flushleft}
\end{task}



보존력이 작용하는 경우에 한해서 우리는 퍼텐셜에너지를 정의하고 역학적 에너지 보존을 사용할 수 있다. 역학적 
에너지가 보존되지 않는 경우는 비보존력이 작용하는 것이다. 운동마찰력과 공기의 항력은 비보존력이다. 
예를 들어 마찰이 있는 마루 위에서 토막을 미끄러트려 보자. 미끄러지는 동안 마루가 작용하는 
마찰력은 토막에 음의 일을 하므로 운동에너지가 줄어든다. 하지만 퍼텐셜에너지의 증가는 없다. \\
이때의 에너지는 어디로 간 것일까?
이 에너지들은 열에너지로 전환된다. 물체와 표면이 뜨거워지는 것을 보면 알 수 있다. 이런 경우는 
역학적 에너지 보존으로 운동을 분석하기 어렵다.
반대로 보존력만 있는 경우에는 물체의 운동에 관한 어려운 문제들을 매우 간단하게 풀 수 있다. 
\subsection{외부력이 계에 한 일}

보존력이 한일은 운동에너지를 증가시키지만 위치에너지를 꼭 같이 감소시켜 역학적 에너지를 보존시킨다.
자, 이제 상황을 확장해보아 일반적인 경우를 보자. 계에 보존력 이외의 외부력이 일을 한다면 
이는 계의 역학적 에너지를 증가시킨다. 외부력이 해준 일은 상황에 따라 
운동에너지와 퍼텐셜에너지에 나누어져 역학적 에너지를 증가시킨다. 
\begin{equation}
  W=\Delta E_{\textrm{mec}}=\Delta K+\Delta U
\end{equation}

\begin{task}[외부력]
  \begin{flushleft}
질량 $m=10$kg의 포탄을 평균 20000N의 연직방향($\vu{j}$)의 힘으로 변위 $\va{s}=(0.05m) \vu{j}$만큼 폭약이 힘을 주어 
바닥에서 발사시켰다.  
    \begin{tasks}[label=(\arabic*)](1)
      \task 포탄이 올라갈 수 있는 최대 높이는 얼마인가? \\
      \task 포탄으로부터 6m 높이에 용수철 상수 100N/m의 용수철이 있다면 압축되는 길이는 얼마인가?\\
   \end{tasks}
\end{flushleft}
\end{task}


\marginpar{
  \begin{center}
  \includegraphics[width=5.5cm]{사진모음/에너지/spring.eps}\captionof{figure}{용수철 에너지 보존 상황}
  \end{center}} 




하지만 이는 마찰이 없을 때로 국한된다. 마찰이 있는 경우는 역학적 에너지 이외에 열에너지로 전환되는 에너지가 생긴다. 따라서 
마찰이 포함된 운동의 경우 계에 한일은 다음과 같이 정의된다. 

\begin{equation}
  W=\Delta E_{\textrm{mec}}+\Delta E_{\textrm{th}}
\end{equation}

\begin{task}[일, 마찰력]
  \begin{flushleft}
사람이 양배추를 담은 전체 질량 $m=10$kg의 나무궤짝을 크기 40N의 일정한 수평력 $F_0\vu{i}$으로 콘크리트 바닥에서 밀고 있다. 
변위는 $\va{s}=(0.5m) \vu{i}$이고, 궤짝의 속력은 $v_0=4$m/s에서 $v=2$m/s로 감소한다. 
    \begin{tasks}[label=(\arabic*)](1)
      \task 힘 $\va{F}$가 한 일은 얼마인가? \\
      \task 궤짝과 바닥의 열에너지 증가 $\Delta E_{th}$의 크기는 얼마인가?\\
   \end{tasks}
\end{flushleft}
\end{task}

\marginpar{
  \begin{center}
  \includegraphics[width=5.5cm]{사진모음/에너지/cabbage.eps}\captionof{figure}{마찰면 상황}
  \end{center}} 

\subsection{탄성충돌과 비탄성충돌}
우리가 배운 운동량 보존 법칙과 에너지 보존 법칙을 이용하면 보다 많은 상황을 분석할 수 있다. 
두 물체가 충돌하는 상황을 가정하자. 외력($\frac{d\vec P}{dt}$)이 0인 경우 운동량 보존에 의해 두 물체의 충돌 전과 충돌 후의 운동량은 보존될 것이다. 
하지만 운동에너지는 보존되지 않는다. 왜냐하면 보통의 경우에는 운동에너지 이외에 열에너지의 형태로 에너지가 변환되기 때문이다.
이때에 운동에너지가 보존되는 상황을 탄성충돌 그렇지 않은 상황을 비탄성 충돌이라고 한다. 



\marginpar{
  
  \begin{flushleft}
   \textbf{Question} 탄성충돌과 비탄성 충돌을 관찰할 때 어떤 차이점이 있을것이라고 예상하는가?
  \end{flushleft}}


\subsubsection{탄성충돌(elastic collision)}
fig. \ref{Energy collision}\과 같이 두 물체가 충돌하는 경우를 다룬다.
상황을 간단히 하기 위해 충돌과정은 1차원으로 제한하며 물체 A에 대한 표적(B)이 정지한 경우 ($v_{2i}=0$)를 생각해보자. 


모든 입자의 충돌에서 충돌 전후의 운동량은 보존된다. 따라서, 
\begin{equation} \label{Eq.운동량보존}
  m_1v_{1i} =m_1v_{1f} +m_2v_{2f} \Rightarrow m_1(v_{1i}-v_{1f})=m_2 v_{2f}
\end{equation}

탄성충돌은 운동에너지가 보존되는 경우이다. 따라서 역학적 에너지 보존을 적한다.

\begin{equation} \label{Eq.에너지보존}
  \frac{1}{2} m_1v^2_{1i} =\frac{1}{2}m_1v^2_{1f} +\frac{1}{2}m_2v^2_{2f} 
  \Rightarrow m_1(v^2_{1i}-v^2_{1f})=m_2v^2_{2f}
\end{equation}

\marginpar{
  \begin{center}
  \includegraphics[width=5.5cm]{사진모음/에너지/twobody.eps}\captionof{figure}{두 물체의 충돌 분석}
  \label{Energy collision}
  \end{center}} 


Eq \ref{Eq.운동량보존}, \ref{Eq.에너지보존}을 연립하면 다음과 같은 식을 얻는다. 
\[
\begin{split} 
   &m_2v_{2f}(v_{1i} +v_{1f})=m_2v^2_{2f} \rightarrow v_{1i}+v_{1f}=v_{2f}\\
   &\therefore \phantom{text} m_1(v_{1i}-v_{1f})=m_2(v_{1i}+v_{1f})\\
\end{split}
\]




이를 $v_{1f}$와 $v_{2f}$에 대해 정리하면 
\begin{align} 
   &v_{1f} =\frac{m_1-m_2}{m_1+m_2} v_{1i} \\
   &v_{2f}=\frac{2m_1}{m_1+m_2}v_{1i}  
\end{align}

\begin{task}[탄성충돌]
  \begin{flushleft}
  1차원에서 물체1이 정지해있던 물체2에 $\vec{v_{1i}}\vu{i}$의 속도로 충돌하는 경우를 생각해보자. 충돌이 탄성충돌이라 할 때 
  아래의 물음에 답하시오. \\
  \phantom{h}\\
  1) 물체1과 물체2의 질량이 같은 경우($m_1=m_2$), $v_{1f}$와 $v_{2f}$를 구하여라.\\
\indent

  2) 물체2의 질량이 많이 큰 경우($m_2 \gg m_1$), $v_{1f}$와 $v_{2f}$를 구하여라. \\
  \indent

  3) 물체1의 질량이 많이 큰 경우($m_1 \gg m_2$), $v_{1f}$와 $v_{2f}$를 구하여라. 

\end{flushleft}
\end{task}

\begin{task}[탄성충돌 예시문항]
  \begin{flushleft}
  2m/s의 속력으로 움직이는 물체 A($m_1=1$kg)가 반대 방향, 같은 속력으로 움직이는 물체 B($m_2=0.5$kg)와 탄성충돌을 하였다.
  충돌 후 각각의 속력을 구하시오.\\
\end{flushleft}
\end{task}

\subsubsection{비탄성충돌(inelastic collision)}
충돌과정에서 계에 작용하는 외력이 없는 상황에서 운동량은 보존된다. 하지만 운동에너지는 비보존력에 의해 열에너지로 손실되면
보존되지 않는다. 따라서 자연에서 일어나는 대부분의 거시적 충돌은 비탄성 충돌이다. 특별히, 
충돌 후 물체가 하나로 합쳐지는 경우를
\textbf{완전비탄성 충돌}이라고 한다. 

\[
\begin{split} 
   &\va{p}_{1i}+\va{p}_{2i} =m_1\va{v}_{1i}+m_2\va{v}_{2i}=(m_1+m_2)\va{v}_f 
\end{split}
\]

완전 비탄성 충돌 후 물체의 운동은 질량중심의 운동과 일치한다.

\[
\begin{split} 
   &\va{v}_{cm}=\frac{m_1\va{v}_{1i}+m_2\va{v}_{2i}}{(m_1+m_2)}
\end{split}
\]


\begin{task}[비탄성충돌 탄동진자]
  \begin{flushleft}
  질량 M=1kg인 나무로 만든 탄동 진자에 질량 m=0.01kg 인 탄이 $v$의 속력으로 날아와 박혔더니, 이 진자가 높이 $h=0.5$m만큼 올라간
  후, 잠시 멈추고 처음 위치로 내려왔다. 탄환의 속력 $v$를 구하시오. (단, 줄의 질량과 공기마찰은 무시하고 
  계산기를 이용하여 해결한다.)
\end{flushleft}
\end{task}

\marginpar{
  \begin{center}
  \includegraphics[width=5.5cm]{사진모음/에너지/pendlum.eps}\captionof{figure}{탄동진자}
  \label{Energy collision}
  \end{center}} 

\end{flushleft}
