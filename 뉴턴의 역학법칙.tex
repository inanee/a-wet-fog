\chapter{뉴턴의 3가지 역학법칙}
\chapterprecis{\noindent 이 장에서는 뉴턴의 역학법칙을 다양한 관점과 도구를 통해 이해한다.}


이제 본격적으로 뉴턴의 운동법칙을 배우고자 한다. 뉴턴의 운동법칙은 기계적인 결정론과 연관된다. 
프랑스의 수학자 라플라스의 에세이에선 "우주에 있는 모든 원자의 정확한 위치와 운동량을 알고 있는 존재가 있다면, 이것은 뉴턴의 운동법칙을 이용해 과거, 
현재의 모든 현상을 설명해주고, 미래까지 예언할 수 있다"고 서술하였다.
그 말의 의미를 곱씹다보면 우리가 배울 뉴턴의 운동법칙이 와닿을 것이다.
본격적으로 뉴턴의 운동법칙을 학습하기에 앞서 2가지를 학습하자 \\
\marginpar{
  \begin{center}
    \includegraphics[width=4cm]{laplace_jpg.jpg}
    \captionof{figure}{Pierre Simon Laplace\\(1749.3.23.-1827.3.5.)}
  \end{center}} 

\noindent
첫째로는 벡터량이다. 이를 위해 한번쯤은 듣거나 책에서 봤을 아래의 수식을 잘 들여다보자. 
    \begin{equation}\label{eq:newton's law 1}
      \va{F}=m\va{a} 
    \end{equation}
이 짧은 수식을 이해하는 것은 뉴턴 역학을 절반 이상을 이해하는 것인만큼 함축하는 것이 아주 많은 식이다. 
잘 보면 질량을 뜯하는 문자에는 화살표가 붙지 않고 힘과 가속도 부분의 문자는 화살표가 붙는다.
이 차이는 \textbf{질량은 양만 있는 스칼라량, 힘과 가속도는 양과 방향이 있는 벡터량}이라는 차이가 있다. 
벡터는 스칼라보다 다루기가 매우 까다롭다. 벡터에 대해서 부록을 통해 학습해보자.


두번째는 다른것으로 대체 정의할 수 없는 물리량인 시간 $t$, 길이 $l$, 질량 $m$을 이용하여 속도와 가속도를 정의하여 보자. 
이 때 변위 벡터 $\vec{s}$는 움직이는 물체의 나중 위치벡터 $\va{x}$와 처음 위치벡터 $\va{x}_0$의 차이다. 
즉, $\vec{s} = \vec{x}-\vec{x_0}$이다. 
위치는 기준점에 따라 달라진다. 
하지만 변위벡터는 기준점과 관계없이 변화량만 다루기 때문에 기준이 필요치 않아 유용하다. 
이러한 관계는 시각과 시간의 차이와 유사하다.

\marginpar{
  \begin{center}
    \includegraphics[width=5.5cm]{사진모음/벡터의 표현.png}
    \captionof{figure}{위치벡터와 변위벡터}
  \end{center}} 



  \begin{defn}[속도와 가속도]
\begin{equation}\label{def:acceleration}
  \va{v} = \frac{\va{x_2}-\va{x_1}}{t_2-t_1} \Rightarrow \lim_{\Delta{t}\to 0}\frac{\Delta{\va{x}}}{\Delta{t}}=\dv{\va{x}}{t}
\end{equation}
\begin{equation}
  \va{a} = \frac{\va{v_2}-\va{v_1}}{t_2-t_1} \Rightarrow \lim_{\Delta{t}\to 0}\frac{\Delta{\va{v}}}{\Delta{t}}=\dv{\va{v}}{t}
\end{equation}

이 때 시간간격을 무한소($\Delta t \rightarrow 0$)로 취했다. 
만약 무한소가 아니라면 그 시간간격 동안의 \textbf{평균 속도($\vec{v}_{avg}$)와, 평균 가속도($\vec{a}_{avg}$)}라고 하고, 
위의 정의와 같이 무한소를 취한 경우는 \textbf{순간 속도, 순간 가속도}라고 한다.\footnote{Eq. \ref{def:acceleration}는 시간에 대한 변위 함수의 미분으로
볼 수 있다. 미분의 방법에 대하여 부록에서 공부하고 오자.}
  \end{defn}






이 장을 읽은 후에는 다음을 이해하길 기대한다.



\tightlists
\begin{itemize}
\item 힘이 벡터임을 알고 벡터의 특징, 벡터합과 곱을 할 수 있다.\footnote{부록의 "vector"으로 이동하여 학습하고 오자.}
\item 갈릴레이의 사고실험을 접하고, 뉴턴의 제1 운동법칙을 이해한다.
\item 벡터를 이용해 한 물체에 작용하는 힘(외부력)의 자유물체도를 그릴 수 있다. 
\item 실험을 통해 질량의 개념을 받아들이고 운동량 개념을 접한다.
\item 운동량의 시간변화량과 힘의 관계를 받아들이고 운동량 보존 법칙\\을 이해한다.
\item 힘과 운동량 관계를 통해 뉴턴 제2 운동법칙을 설명할 수 있다.
\item 뉴턴의 제3 운동법칙을 이해한다.
\item 간단하고 다양한 물리상황에 뉴턴의 역학법칙을 예외없이 적용시킬수 있다.
\end{itemize}

\section{뉴턴의 운동법칙과 $\va{F}=m\va{a}$}
뉴턴의 운동법칙은 원자크기부터 거대한 천체운동까지 적용할 수 있는 매우 유용한 도구이다. \footnote{광속에 가까운 상황(상대성이론)
과 원자 단위의 매우 작은 미시세계(양자역학)에 대해서 설명할 때는 다른 설명체계를 가져와야 한다.} 실제 생활간에는 경험에 따른
선지식이 생기고 그것에는 많은 오개념들이 있다. 예를 들면 우리는 발로찬 축구공이 아무런 힘을 주지 않더라도 정지한다고 경험한다. 
하지만 이것은 뉴턴 역학에 맞지 않는것이다.
이러한 생각들을 지혜롭게 해석하고 풀어간 앞선 현인들의 생각을 통해 자연스럽게 뉴턴역학을 습득하여보자.  

\subsection{갈릴레이의 사고실험과 관성(뉴턴의 제1 운동법칙)}
물리학에서 유용하게 사용되는 사고실험을 도입해보자. 사고실험은 실제의 자연현상에는 고려할 것이 너무 많아 
머릿속에서만 조건을 단순하게 가정하고 이론을 바탕으로 일어날 현상을 예측해보는 실험으로 물리학에서 유용하다. 
뉴턴이전에는 갈릴레이라는 천재가 있었다. 그는 매끄러운 경사면 위에 공을 굴리면 반대쪽 경사면의 같은 높이까지 공이 올라간다는 
점을 관찰하였다. 정확히는 같지 않았지만 사면을 매끄럽게 하고 공을 사용하여 마찰력을 최대한 줄이면 거의 같은 높이까지 
공이 올라간다. 이 때 사면을 점차 펼친다면 공은 영원히 굴러갈 것이다. 그는 이것이 자연스러운 운동이라고 생각했다. 
\marginpar{
  \begin{center}
  \includegraphics[width=4cm]{갈릴레이.jpg}\captionof{figure}{Galilei,Galileo(1564-1642)}
  \end{center}} 

\begin{figure}[h]
  \centering
   \includegraphics[width=5cm]{갈릴레이의 사고실험.jpg}
   \caption{갈릴레이의 사고실험}\label{fig:Galilei's thought}
 \end{figure}
그는 사고실험을 통해 즉, 머릿속의 실험을 통해 \textbf{물체가 힘을 받지 않으면 계속 원래 운동상태를 유지}한다고 증명하여 
당시의 힘을 받지않을 때 물체는 자연스럽게 정지한다는 자연관을 반박하고 \textbf{관성}의 개념을 발견했다. 


\subsection{힘 (Force)}\label{sec:힘}
\begin{flushleft}
  
필자가 강조하는 $\va{F}=m\va{a}$ 를 다시 들여다보자. 이제, 좌변의 F를 들여다보자. 
이것은 힘을 나타내며 생긴모양으로 벡터임을 알 수 있다. 이에 힘은 벡터의 성질을 따른다. 
우변에 있는 가속도 $\va{a}$는 속도의 변화를 나타내며 Eq. \ref{def:acceleration}에서 정의하였다.

따라서 $\va{F}=m\va{a}$로 알고 있던 뉴턴의 운동법칙은 \textbf{힘이란 관성을 갖는 질량의 운동상태를 바꾸어주는
(가속도를 만드는) "원인"}으로 해석이 가능하다. 이 때 운동
상태는 속도를 뜻하고 속도의 변화인 가속도는 운동상태의 바뀜으로 해석한다. 
더 정확하게는 \textbf{1kg으로 정의된 물체에 힘을 작용시켜 물체의 가속도로 $1m/s^2$을 얻을 때에 물체에 작용한 힘이 
1N의 크기를 갖는다.}                                                              

좌변의 F에 들어갈 수 있는 원인이 되는 힘은 가볍게는 "밀고 당기는 힘"으로 알고 있을것이다. 하지만 이러한 힘들은 기본적인 4개의 
힘의 다른표현에 불과하다. 기본적인 힘은 4가지 종류로 \textbf{중력, 전자기력, 약력, 강력}이 있다. 

\begin{figure}[h]
  \centering
   \includegraphics[width=8cm]{funddamental interaction.jpg}
   \caption{기본적인 상호작용}
 \end{figure}


이 이외에 들어보았던 마찰력, 장력(실이 당기는 힘), 부력, 자기력 등은 모두 이 4가지 힘의 서로 다른 표현이며 특히, 대부분은 전자기력의 표현이다. 
약력과 강력은 원자 크기에서 작용하여 일반물리학 수준에서는 다루지 않는다. \footnote{주의할 점으로 이후에 배울 구심력과 원심력은 
"원인"에 해당하는 힘이 아니다. 즉 좌변에 원심력과 구심력이라는 원인은 고려하면 안된다.}


모든 힘, 운동상태의 변화는 본질적으로 물체쌍 사이의 상호작용에 의한 것이다.

\marginpar{
  \begin{center}
  \includegraphics[width=4.0cm]{사진모음/손흥민 슛.jpg}\captionof{figure}{힘이 주어지는 사건은 혼자 일어나지 않는다.}
  \end{center}} 
예를 들어 손흥민 선수가 슛을 하여 축구공의 운동에 가속도가 있는 상황을 보자. 
공의 입장에서 \textbf{손흥민의 발이 공을 찼다.} 이 때에는 힘을 준 손흥민의 발이 있고
힘을 받은 공이 있다. 이렇듯 \textbf{힘은 단일로 일어나지 않으며 힘을 준 주어와 받는 목적어가 필연적이다.} 
이러한 맥락에서 우주의 4가지 기본적인 힘들은 모두 상호작용(interaction)이라고 한다. 
\end{flushleft}





\subsection{자유물체도 Free Diagram}
이제 힘에 대해서 배웠으니 실제 실생활의 문제들을 분석하기 위해 힘을 표현하는 "자유물체도"를 배우고 적용해보자. 
\textbf{자유물체도는 물체(계)에 작용하는 모든 "힘(외부력)"을 벡터 화살표로 표시한다.} \footnote{여기에서
벡터의 꼬리는 물체의 중심 또는 무게 중심에 두는것으로 약속한다. 물체의 병진운동의 경우 물체를 점질량으로 근사하여 하나의 점 또는 
그림의 편의상 크기를 가지는 점으로 표시한다.}
이것은 관심있는 물체(계)에만 집중하기 위해 고안되었다. 자유물체도를 명확히 이해하기 위해서는 "외부력\footnote{이후의 외부력을 
외력이라고 줄여서 언급하겠다.} (external force)"과 "계 (system)"를 정의해야 한다. 
\textbf{계는 한 개 이상의 물체로 구성되어 있으며 계의 외부에서 계에 작용하는 힘을 외부력이라고 한다.} 
만약 물체들이 서로 단단하게 붙어있으면 이 계를 하나의 물체로 간주하며 이 계에 작용하는 힘을 외부력이라고 한다. 
중요한 것은 물체 내부에서 상호작용하는 내부력은 고려하지 않는다. \footnote{"조별 자유물체도 그리기" 활동을 수행하여보자.}

이 때 물체에 작용하는 모든 외부력들을 표현하기에 외부력의 벡터합을 \textbf{알짜힘}(net force)이라고 하며 
$\vec{F}_{net}$으로 표현한다. 그리고 이 때 대상이되는 물체는 계라고 하며 이 계의 총질량이 m이 된다.
따라서 뉴턴의 역학법칙식을 다음과 같이 확장하여 쓸 수 있다.          

    \begin{defn}[알짜힘과 뉴턴의 운동 제2법칙]
    \begin{equation}\label{eq:newton's law 2}
      \sum\vec{F}_{ext}=\vec{F}_{net}=m\va{a} 
    \end{equation}
  \end{defn}

  \clearpage
  \begin{task}[줄의 장력, 수직항력]
 \begin{flushleft}
  {\IfFileExists{사진모음/연습하기1.PNG}
  {\includegraphics[width=0.8\linewidth]{사진모음/연습하기1.PNG}
 \label{fig:연습1}}%
  {\rule{\linewidth}{4cm}}}
 \end{flushleft}
\end{task}

\begin{task}[object on a horizontal object]
  \begin{flushleft}
    {\IfFileExists{사진모음/연습5.png}
    {\includegraphics[width=0.6\linewidth]{사진모음/연습5.png}
  \label{fig:물체 위의 물체}}%
    {\rule{\linewidth}{4cm}}}
   \end{flushleft}
\end{task}
