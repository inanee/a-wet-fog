
\section{실험으로 알아보는 힘의 특성}
\begin{flushleft}
이번 section의 concept들은 아래의 실험을 재료로 설명할 것이다. 아래의 실험 과정을 수행하여보고 
머릿속에 실험장면과 데이터를 있는 그대로 받아들여보자. 
 

\marginpar{
  \begin{center}
  \includegraphics[width=5.5cm]{스마트카트 동영상 설명자료.png}
  \captionof{figure}{YouTube : Pasco smartcart 검색}
  \end{center}} 

\begin{exper}{ \Large {수레의 분열 실험} }

  \label{exp:Momentum}

  \begin{enumerate}
  \item 준비물 : 역학용 스마트 카트, 수레 멈춤용 막대, 레일, 추(200g), 파스코 소프트웨어와 PC
  \footnote{실험전에 충분히 파스코 실험기기의 사용법을 익혀두자.}\\


\item 실험 방법 : 

  \IfFileExists{운동량 보존 실험.PNG}
  {\includegraphics[width=0.8\linewidth]{운동량 보존 실험.PNG}
  \captionof{figure}{실험 장면}\label{fig:Momentum}}%
  {\rule{\linewidth}{4cm}}%


  \begin{enumerate}
    \item 스마트카트와 CAPSTONE 프로그램을 연결하여 스마트 카트의 물리량을 컴퓨터로 확인한다.
    \item 레일 정중앙에 스마트 카트를 압축시켜 놓는다.
    \item 카트의 운동에 영향을 주지 않도록 조심하여 두 카트를 분열시킨다. 이 때 분열되어 힘을 받아 생성된 초기 수레의 최대 속도를 
    소프트웨어로 측정하고 기록한다.
    \item 두 카트의 질량(조작변인)을 달리하며 위의 과정을 반복한다. 질량과 분열속도의 곱-운동량(종속변인)을 표로 정리한다.
    \item 분열될때에 두 카트의 시간-속도 그래프를 보고서에 첨부한다.
  \end{enumerate}

  
  \phantom{dh}3. 실험해석 
  \begin{enumerate}
    \item 분열하는 도중에 보존되는 양이 있는가? 그 물리량은 무엇인가?
    \item 왜 보존된다고 생각하는가?
    \item 위의 해석을 통해 질량을 어떻게도 정의할 수 있겠는가?
  \end{enumerate}

\end{enumerate}
\end{exper}


\subsection{운동량(momentum)}


갈릴레이와 뉴턴 사이에는 데카르트라는 현인이 있었다. 그는 갈릴레이와 같이 한 물질은 자신의 운동상태를 
지속하려는 경향 '관성'을 갖는다고 생각하였다. 따라서 물체의 운동에서 보존되는 양으로 운동량의 개념을 생각해내었다. 
이후 뉴턴은 수학적으로 더 엄밀하게 운동량을 그 물체의 질량과 속도벡터를 곱한것으로 다음과 같이 정의하였다. 

\marginpar{
  \begin{center}
  \includegraphics[width=4cm]{Decartes.jpg}\captionof{figure}{Decaretes(1596-1650)}
  \end{center}} 


  \begin{defn}[운동량]

\begin{equation}
  \va{p} = m\va{v}
\end{equation}
운동량은 속도벡터에 질량이 배수된 \textbf{벡터}임으로 계산에 주의해야 한다.  

  \end{defn}

\marginpar{
  \begin{flushleft}
   운동량의 단위는 kg$\cdot$m/s, 또는 N$\cdot$s 이다. 
   $1$kg$\cdot$m/s=1N$\cdot$s 
  \end{flushleft}}

\noindent



\subsection{운동량과 힘의 관계(뉴턴의 제2 운동법칙)}
  
Eq. \titleref{eq:newton's law 2}($\vec{F}_{net}=m\vec{a}$)을 참고하여 실험 \ref{exp:Momentum}을 분석하여 보자. 
수레가 서로 용수철로 묵여 있을 때부터 두 수례를 하나의 계로 생각해보자.
 분열하는 순간에 두 수레로 이루어진 계에 작용하는 외력 $\vec{F}_{ext}$은\footnote{힘을 서술할 때 주어와 목적어 모두가 
계의 성분이라면 그것은 내력이다. 예를 들어 수레 A가 수레 B를 미는 힘은 수레A,B가 계의 성분임으로 내력이다.} 없다. 이에 의해 
계의 운동상태의 변화는 없어야 한다. 이전에 우리는 운동상태의 변화를 물체의 가속도로 두었으나 
이제는 가속도가 두 수레에 대해 존재하므로 대표할만한 가속도가 없다. 그렇다면 그 순간 계에 변하지 않은 운동상태의 변화를 대변하는 양은 무엇일까? \\
데카르트가 도입한 운동량을 사용하여 보자. 실험 \ref{exp:Momentum}에서 벡터로 이루어진 운동량 벡터의 전체합은 0이었다.
\begin{equation}
  m_1\va{v}_{1i}+m_2\va{v}_{2i}=m_1\va{v}_{1f}+m_2\va{v}_{2f}=0
\end{equation}



\textbf{외력이 없는 경우 두 개의 입자로 이루어진 계의 분열에서 변하지 않은 값은 바로 선운동량이다.}
\footnote{추후에 회전운동에는 병진운동의 운동량과 구분되는
각운동량이 등장한다. 따라서 이를 구분하기 위해 선운동량(linear momentum)이라고 한다.}
가속도를 운동상태의 변화로 두었던 우리의 생각을 확장하여 여러 입자로 이루어진 계에 대해 운동량 변화를 
운동상태의 변화로 확장시킬 수 있다. 
\marginpar{
  \begin{center}
  \includegraphics[width=4.0cm]{사진모음/운동량/수레의 분열.png}\captionof{figure}{두 수레 계에 작용하는 외력은 없다.}
  \end{center}} 

    \begin{defn}
      \centering{\textbf{선운동량의 시간변화율은 입자에 작용하는 알짜힘과 같다.}} 
    \begin{equation}\label{eq:newton's law 3}
      \vec{F}_{net}
      =\frac{\Delta \sum \vec{p}_{net}}{\Delta t}=\dv{\vec{p}_{net}}{t}
    \end{equation}
  \end{defn}
  
   $\va{p}=m\va{v}$으로 정의하므로 Eq. \ref{eq:newton's law 3}는 우리가 익히 사용하는 $\va{F} =m\va{a}$를 포함함을 알 수 있다.
  하지만 로켓과 같이 질량이 변하며 운동하는 물체의 운동을 분석할 때에는 외력이 없는데 로켓이 나아가는 것을 설명할 수 없다.
  이런경우엔 Eq. \ref{eq:newton's law 3}을 활용해야만 한다. 한마디로 더 넓은 경우에 대해서 설명해주는 더 범용적인 Eq.라고 할 수 있다. 
  실제 뉴턴이 집필한 프린키피아에도 Eq. \ref{eq:newton's law 3}의 식으로 적혀있는것을 이후의 과학자들이 $\va{F} =m\va{a}$로 
  변형한것이다.  
  \marginpar{
  \begin{center}
  \includegraphics[width=4.0cm]{사진모음/로켓.jpg}\captionof{figure}{로켓의 추진}
  \end{center}} 

  \noindent 
  수레의 분열 상황과 같이 \textbf{충돌, 분열 등 2개 이상의 물체로 이루어진 계에 작용하는 외력이 0일 때
   계의 운동량 변화가 없이 보존되는 것을 "운동량 보존법칙"}이라고 한다. 

   \begin{task}[운동량 보존법칙]
    
  \begin{flushleft}
     수평면에 정지해 있는 물체 B를 향해 물체 A가 $4v$의 속력으로 등속 직선 운동하고있다. 그래프는 A와 B가 충돌한 이후부터 
     B의 위치를 시간에 따라 나타낸 것이다. A와 B의 질량은 2kg으로 같다. A와 B는 충돌한 후 붙어서 함께 운동하며, 2초일 때 벽과 
     충돌 후 반대 방향으로 함께 운동한다. 각 질문에 답하여라. 
  
 \begin{tasks}[label=(\arabic*)](1)
      \task 속력 $v$의 값을 구하시오. \\
      \task A의 1초부터 6초까지 운동량 변화량을 구하시오.\\
   \end{tasks}
    
 \end{flushleft}
\end{task}
\end{flushleft}

\marginpar{
  \begin{center}
  \includegraphics[width=5.5cm]{사진모음/운동량/시간위치 그래프.png}\captionof{figure}{위치-시간 그래프}
  \end{center}} 




\subsection{운동량과 충격량(Impulse)}

\begin{flushleft}
  
Eq. \ref{eq:newton's law 3}에 따라서 우리는 한가지 물리량을 정의할 수 있다. 한 공이 힘을 받는 단순한 경우는 다른 
공과 충돌한 경우이다. 공이 받는 힘 $\vec{F}(t)$는 충돌하는 동안 변하면서 공의 선운동량 $\vec{p}$를 바꾼다. 이 것은 Eq. \ref{eq:newton's law 3}
에 따라서 다음과 같이 쓸 수 있다. 

\marginpar{
  \begin{center}
  \includegraphics[width=4.0cm]{사진모음/운동량/공의 충돌.png}\captionof{figure}{충돌}
  \end{center}} 


\begin{equation}
  d\vec{p}=\vec{F}(t)dt
\end{equation}

충돌 직적의 시각 $t_i$부터 충돌 직후의 시각 $t_f$까지 위의 식을 정적분 해주면 다음과 같이 표기할 수 있다. 

\begin{equation}
  \int_{t_i}^{t_f} d\vec{p}= \int_{t_i}^{t_f} \vec{F}(t)dt
\end{equation}
이 식의 좌측항 $\vec{p}_f-\vec{p}_i= \Delta\vec{p}$ 선운동량의 변화량을 뜻하고 오른쪽 항을 충격량이라고 부른다. 충격량은 
충격력과 시간간격의 곱으로 이를 정의 하면 아래와 같다. 

\begin{defn}[충격량의 정의]
  \begin{align}
 &I = \int_{t_i}^{t_f}\vec{F}(t)dt\\
 & \textrm{if}, \phantom{o} F=\textrm{const},\phantom{o}I=F \Delta t
\end{align}
\centering{\textbf{선운동량의 변화량은 물체에 가해지는 충격량과 같다.}}
\end{defn}
이를 기하학적인 그래프로 확장하면 $\vec{F}(t)$의 함수를 시간 변수에 대해 적분하면 충격량(Impulse) $\vec{I}$을 구할 수 있다.
따라서 적분의 기하학적 의미에 따라 F-t그래프에서 그래프 개형과 t축 사이의 넓이를 구한다면 충격량을 알 수 있다. 

\marginpar{
  \begin{center}
  \includegraphics[width=4.5cm]{사진모음/운동량/F-t 그래프.png}\captionof{figure}{F-t 그래프의 넓이는 충격량과 같다.}
  \end{center}} 
\end{flushleft}

\begin{task}[운동량 충격량 관계]
    
  \begin{flushleft}
      호수 위에서 질량이 다른($m, 2m$)의 두 보트 A,B가 일정한 바람의 힘($F$)을 받아 움직인다. 
      두 보트가 정지 상태에서 출발하여 
      거리 $L$만큼 떨어진 선착장에 도착하였다. 어느 보트가 더 큰 운동량을 갖고 선착장에 도착하는가? 단, 모든 마찰력은 무시하고 가능한한
      정량적으로 비교하여보라.
 \end{flushleft}
\end{task}

   
\subsection{작용-반작용(뉴턴의 제3 운동법칙)}

\begin{flushleft}
힘의 특성을 파악하기 위해 분열하는 두 수레의 실험 상황을 들여다보자.

분열되는 두 수레의 계에는 외력이 0이므로 두 수레의 전체 운동량이 보존된다. 
\begin{equation}
  m_1 \va{v_1} +m_2 \va{v_2} =0      \Rightarrow   m_1 \va{v_1} =-m_2 \va{v_2}
\end{equation}
이것은 Eq.\ref{eq:newton's law 3}에 따라 아래와 같이 전개 수 있다.\footnote{운동량 보존과 작용-반작용 법칙은 맥이 같다.}
\marginpar{
  \begin{center}
  \includegraphics[width=4cm]{분열.png}\captionof{figure}{분열}
  \end{center}} 

\begin{equation}
  \dv{(m_1 \va{v_1})}{t} =-\dv{(m_2 \va{v_2})}{t} 
\end{equation}

\begin{equation}
  \vec{F}_{21}=-\vec{F}_{12}
\end{equation}
좌변항은 물체1이 계인 경우로 물체2로부터 받는 힘을 나타내고, 우변항은 물체2가 1로부터 받는 힘이다. 
이 두 힘은 크기가 정확히 갖고 방향이 반대이다.

이것이 바로 뉴턴의 제3 운동법칙 작용-반작용 법칙이다. 작용과 반작용에 해당하는 힘의 서술은 힘을 주는 주어와 힘이 작용하는 
목적어가 서로 상호교환 하면 된다. 예를 들어 \textbf{물체 1이 물체 2를 미는 힘의 반작용은 물체 2가 물체 1을 미는힘이 된다.}
이때 주의할 점은 작용 반작용 관계에 있는 힘들은 서로 다른 물체에 작용하여 서로 상쇄되지 않는다. 
\textbf{한 물체에 서로다른 방향의 같은 힘이 작용하여 정지해있는 힘의 평형과 잘 구분하여야 한다.}
\newpage
\begin{task}[object on a horizontal object]
  \begin{flushleft}
    {\IfFileExists{사진모음/운동방정식/reactioan_task.eps}
    {\includegraphics[width=0.6\linewidth]{사진모음/운동방정식/reactioan_task.eps}
  \label{fig:물체옆의물체}}%
    {\rule{\linewidth}{4cm}}}
   \end{flushleft}
\end{task}



\subsection{뉴턴의 운동법칙}
이제까지 뉴턴 이전의 현인들의 생각을 알아보고 간단한 실험을 진행했다. 이를 통해 최종적으로 필자가 말하고자 한것은 
3가지로 축약가능한 뉴턴의 운동법칙이다. 이를 정리하면 아래와 같다. 


\begin{defn}[뉴턴의 3가지 운동법칙]
  \begin{enumerate}
  \item First motion law : 계에 외부력이 작용하지 않으면($\sum \overrightarrow{F_{ext}} = 0$) 계는 운동상태를 유지한다. 이것을 \textbf{관성법칙}이라고 한다.
  \item Second motion law: $\sum \overrightarrow{F_{ext}} \neq 0$이고 $\Delta m =0$ 이라면   $\overrightarrow{F_{net}}
      =\dv{\overrightarrow{p_{net}}}{t} =\dv{m \va{v}}{t}= m\va{a}$ 이것을 \textbf{가속도 법칙}이라고 한다.
  \item Third motion law: 힘은 상호작용으로 물체쌍 사이에서 일어나고 서로에게 작용하는 힘은 같은 크기로 반대방향이다. 
  ($\overrightarrow{F_{21}}=-\overrightarrow{F_{12}}$) 이를 \textbf{작용-반작용 법칙}이라고 한다.
\end{enumerate}
\end{defn}

이 3가지 운동법칙은 역학의 문제 상황에 예외없이 적용되는 명제들이다. 
이러한 예외없음이 비약적인 과학발전을 일으켰고 주변 물체들의 운동에 대한 이해를 인간에게 갖고 왔다. 
뉴턴의 운동법칙으로 시작되는 고전역학의 정립을
과학혁명이라고 말하는 이유도 여기에 있다. \footnote{물리학에는 
너무나 많은 문자들이 등장한다. 이 문자들은 물리량을 추상적으로 표현하며 [m]미터와 같은 단위와는 그 의미가 다르다. 
"물리학 문자게임"을 통해 다양한 문자의 의미를 정확히 알고 친숙해져보자.} 

\end{flushleft}