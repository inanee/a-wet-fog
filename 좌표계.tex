
\section{다양한 좌표계}

\begin{flushleft}
더 높은 차원의 다양한 좌표계를 명확히 이해하고 활용하는 것 자체로 복합적인 물체의 운동을 이해할 수 있다.
  
\subsection{데카르트 좌표계}
르네 데카르트가 발명한\footnote{프랑스의 수학자이자 철학자로 천장을 날아다니며 옮겨붙은 파리를 통해 영감을 얻어 좌표계를 발명했다.}
데카르트 좌표계(Cartesian coordinate frame)는 물리학에서 가장 흔하게 볼 수 있는 좌표계로 
문제를 다루는 좌표공간은 x,y,z의 서로 직교하는 좌표로 정의한다. 
데카르트 좌표계는 나타내는 대상이 평행 이동에 대한 대칭을 가질 때 유용하나, 회전 대칭 등 다른 꼴의 대칭은 쉽게 나타내지 못한다.
\marginpar{
  \begin{center}
  \includegraphics[width=4cm]{데카르트 좌표계.jpg}\captionof{figure}{cartesian coordinate}
  \end{center}} 


\subsection{극 좌표계}
극 좌표계(polar coordinate system)는 평면 위의 위치를 각도와 거리를 써서 나타내는 2차원 좌표계이다. 극 좌표계는 두 점 사이의 관계가
각이나 거리로 쉽게 표현되는 경우에 유용하다. 데카르트 좌표계에서 삼각함수로 복잡하게 나타나는 관계가 극 좌표계에서는 간단하게 표현되는
경우가 많다. 극 좌표는 r로 나타내는 반지름 성분과 $\theta$ 로 나타내는 각 성분으로 이루어져 있다. (r, $\theta$) 반지름 성분 r은 좌표계의 원점
에서의 거리를 나타내고, 각 성분은 x축의 양의 뱡향을 0$^{\circ}$로 반시계 방향으로 잰 각의 크기를 나타낸다.  
\marginpar{
  \begin{center}
  \includegraphics[width=4cm]{극좌표.jpg}\captionof{figure}{polar coordinate}
  \end{center}} 

  \begin{task}
직교 좌표계보다 극 좌표계가 더 활용하기 좋은 경우를 알아보자. Eq. \ref{eq:universal gravity}에서 
배운 중력법칙을 적용해보자. 질량이 큰 행성 A를 중심으로 공전하는 행성 B가 있다. A를 고정하여 원점으로 할 때, 
행성 B의 변위벡터가 극좌표계로 $\va{r_1}=(10, \frac{\pi}{3})$, $\va{r_2}=(10,{\pi})$ 일 때 행성 B가 
A로부터 받는 중력 벡터를 직교좌표계와 극좌표계로 각각 표현하시오. (단, 계산을 단순히 하기 위해 행성 A의 질량을 100kg, 행성 B의 질량을 1kg,
중력상수의 값을 1로 하자.)
\end{task}


\subsection{구면 좌표계}
구면 좌표계(spherical coordinate system)는 공간 위의 위치를 각도와 거리를 써서 나타내는 3차원 좌표계이다. 3차원 공간 상의 점들을 나타내는 좌표계의 하나로, 보통 
(r,$\theta,\phi$)로 나타낸다. 원점에서의 거리 r은 0부터 $\infty$ 까지, 양의 방향의 z축과 원점과 위치를 이루는 선이 이루는 각도 $\theta$ 는 0부터 
$\pi$까지, z축을 축으로 양의 방향의 x축과 이루는 각 $\phi$ 는 0부터 $2\pi$  까지의 값을 갖는다. $\theta$ 는 위도로, $\phi$ 는 경도로 표현되는 경우도 있다.
\marginpar{
  \begin{center}
  \includegraphics[width=4cm]{Spherical_coordi.png}\captionof{figure}{spherical coordinate}
  \end{center}} 


  \end{flushleft}