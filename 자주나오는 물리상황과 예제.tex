
\chapter{다양한 예제}\label{sec:math}
%% 절의 카운터를 \pnum으로.
\renewcommand\thesection{\pnum{section}}
\showcommand{makepagestyle}\showcommand{copypagestyle}
\section{}
이 절은 \cite{Kentaro}\을 인용하였다. \env{singlespacing}\을
사용하였다.\showenv{singlespacing}
%\begin{singlespacing}
\[
e^x =1+ \frac{x}{1!} + \frac{x^2}{2!} + \frac{x^3}{3!} + \cdots
\]
이라는 사실이 알려져 있다. 여기에서 $x=1$이라 하면,
\[
e=1+\frac{1}{1!}+\frac{1}{2!}+\frac{1}{3!} +\cdots
\]
가 된다.
우선
\[
\lim_{n\to\infty}\left(1+ \frac{1}{n}\right)^{n} = e
\]
에서 $1/n = h$라 두면,
\[
\lim_{h\to 0}\left(1+h\right)^{\frac{1}{h}} = e
\]
라고 쓸 수 있다. $e$를 밑으로 하는 대수를 $\log$라고 표시하면,
\[
\frac{\log(1+h)}{h} = \log(1+h)^{\frac{1}{h}}
\]
인데, 여기에서 $h \to \infty$이라면,
\[
\lim_{h\to 0}\frac{\log(1+h)}{h} = \lim_{h\to 0}\log(1+h)^{\frac{1}{h}} =\log e.
\]
따라서,
\[
\log(1+h)=x, \quad\text{즉}\quad h=e^{x} -1
\]
이다. 여기에서,
\[
1=\lim_{h\to0}\frac{\log(1+h)}{h}=\lim_{x\to0}\frac{x}{e^x -1}.
\]
따라서,
\[
\lim_{x\to0}\frac{e^x -1}{x}=1
\]
을 얻는다. 그런데 여기에서,
\[
y=e^x
\]
의 도함수 $y'$를 구해본다.
\[
y'=\lim_{h\to0}\frac{e^{x+h}-e^x}{h}=e^x \lim_{h\to0}\frac{e^h -1}{h}=e^x.
\]
따라서,
\[
y=e^x \text{라면}, \qquad y'=e^x
\]
이다.
또, $y=\log x$의 도함수를 구해본다.
%\begin{displaymath}
%\[
\begin{align*}
y' &= \lim_{h\to0}\frac{\log(x+h)-\log x}{h} \\
   &= \lim_{h\to0}\frac{1}{h}\log\left(1+ \frac{h}{x}\right) \\
   &= \frac{1}{x}\lim_{h\to0}\frac{x}{h}\log\left(1+\frac{h}{x}\right) \\
   &= \frac{1}{x}\lim_{\frac{h}{x}\to0}\log\left(1+\frac{h}{x}\right)^{\frac{x}{h}} \\
   &= \frac{1}{x}\log e \\
   &= \frac{1}{x}.
\end{align*}
%\]
%\end{displaymath}
따라서, 
\[
y=\log x \text{라면}, \qquad y'=\frac{1}{x}.
\]
%\end{singlespacing}
이와 같이 간단한 공식으로 얻어진 것은 대수의 밑으로 $e$를 썼기
때문이다. $e$ 이외의 밑을 사용하면, 공식은 보다 복잡하게 된다.
이런 의미에서 $e$를 밑으로 하는 대수를 \emph{자연로그}라고 부른다.\showcommand{emph}
이상으로부터 상상할 수 있듯이, 미적분학과 같은 이론을 전개할 때는
$e$를 밑으로 하는 대수를 사용하고, 실제의 수치계산에서는
$10$을 밑으로 하는 상용로그를 사용한다.
\fancybreak{* * *}
$0 \le t \le 1$에 있어서, $f(t)$는 연속인 도함수 $f'(t)$를 가지고,\showcommand{fancybreak}
$0 < f'(t) \le 1,\quad f(0)=0$이다. 이 때 다음 부등식이 성립함을 보여라.
\[
\left[ \int_{0}^{1} f(t)dt \right ] ^2 \ge \int_{0}^{1} [f(t)]^3 dt 
\]
\vskip\onelineskip
적분구간의 상한을 변수 $x$로 바꾸어본다.
\[
\left[ \int_{0}^{x} f(t)dt \right]^2 \ge \int_0^x [f(t)]^3 dt
\]
이 식의 좌변에서 우변을 빼서 그것을 $F(x)$라고 두자.
\[
F(x) = \left[ \int_{0}^{x} f(t)dt \right]^2 - \int_0^x [f(t)]^3 dt
\]
이 때 $F(0)=0$이다.
$F(x)$를 $x$로 미분하면
\begin{equation}
\begin{split}
F'(x) &= 2\left[ \int_{0}^{x} f(t)dt \right]f(x) - \{f(x)\}^3 \\
  &= f(x) \left[ 2\int_{0}^{2} f(t)dt - \{f(x)\}^2 \right]
\end{split}
\end{equation}
문제의 의미에 의해 $0<x<1$에서 $f(x) > 0$이다. 
\[
G(x) = 2 \int_{0}^{x} f(t)dt - \{f(x)\}^2 \qquad (0 \le x \le 1)
\]
이라 놓으면 $G(0)=0$이고,
\begin{equation}
\begin{split}
G'(x) &= 2f(x) - 2f(x)f'(x) \\
      &= 2f(x) \{1-f'(x)\} \ge 0
\end{split}
\end{equation}
이다. $0 \le x \le 1$인 모든 $x$에 관하여 $f'(x) \le 1$이므로
$G'(x) \ge 0$. 따라서 $G(x) \ge 0$임을 말할 수 있다. 그러므로
\uline{$F(x) \ge 0$이 성립한다.}\showcommand{uline (ulem package)}