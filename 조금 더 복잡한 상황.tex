

\subsection{더 다룰만한 상황 : 경사면}
\begin{flushleft}
  
지금부터는 상황이 조금 추가된 상황을 자유물체도와 뉴턴의 운동법칙을 적용하여 분석하도록 해보자. 아래의 빈칸의 그림에 있는 상황에서 각 물체에
작용하는 힘을 자유물체도로 표현하고 각 물체에 대한 운동방정식($\va{F}=m\va{a}$)을 적용하여 물체의 운동을 예측해보자. 



\begin{task}[slope inclined plane]
    \begin{flushleft}
  {\IfFileExists{사진모음/운동방정식/slope_task.eps}
  {\includegraphics[width=0.6\linewidth]{사진모음/운동방정식/slope_task.eps}
  \label{fig:빗면}}%
  {\rule{\linewidth}{4cm}}}
 \end{flushleft}
\end{task}


\begin{task}[object on the object]
  \begin{flushleft}
    {\IfFileExists{사진모음/연습4.png}
    {\includegraphics[width=0.6\linewidth]{사진모음/연습4.png}
  \label{fig:물체위의물체}}%
    {\rule{\linewidth}{4cm}}}
   \end{flushleft}
\end{task}


\begin{task} [조별 운동분석 직소활동]
  각 조별로 위의 물체의 상황과 같이 물체의 상황을 만들어보고 
  이에 대한 자유물체도와 운동방정식을 정리하여보자. 
  이후에는 다른 조들의 상황을 정답없이 공유하여보고 이에 대해
  토론해보며 운동분석에 자신감을 갖도록하자. 
\end{task}

    \end{flushleft}     

 