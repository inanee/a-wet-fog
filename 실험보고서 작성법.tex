

\chapter{물리학 실험}\label{chap:Experiment in physics}
\chapterprecis{다양한 실험은 물리학 개념을 정립하는데 중요한 거름이 된다. 실험 보고서의 작성법과 유의사항에 대해서 설명한다.}

\begin{flushleft}
\section{실험보고서 작성법}
\begin{exper}{ \Large {실험보고서의 형식과 주안점} }
  \begin{enumerate}[(1)]\setlength{\itemsep}{10mm}
    \item 실험제목 :실험 제목, 실험 일자, 실험실 조건, 공동 실험자의 이름을 보고서 시작에 기록한다.

    \item 준비물 : 실험에 사용한 준비물들을 상세하게 기록한다. 

    \item 이론적 배경 : 실험과정에 대한 이론을 기술한다. \textbf{이론적 배경에서 기술한 내용은 실험 결과를 설명할 수 있는 것을 쓴다.}
   
    \item 실험 방법 : 실험 장치와 실험 과정, 구체적인 실험 방법들을 순서대로 기술한다. 실험 구성에서 실험 구성자의 창의성이 가장 돋보인다.
   
    \item 실험 결과 : 실험 과정에서 얻은 자료를 기록하는 것으로 \textbf{표, 그래프, 그림} 등을 중심으로 측정자료를 한눈에 알아보기 쉽게 작성하며
    측정값 사이의 관계가 명확해야한다. 즉 \textbf{조작변인과 종속변인의 관계가 명확히 보이는 것}이 핵심이다. 

    \item 결론 및 토의 :실험결과를 토대로 실험 목적과 관련하여 가설에 대한 결론과 실험을 통해 알게된 점을 명확하게 기술한다. 또한, 실험 중 
    가장 중요한 오차를 찾고, 이를 개선할 수 있는 방안을 제시하는 것이 좋다. 그리고 오차는 구체적으로나 또는 구체적으로 제시할 수 있어야하고
    그렇지 않다면 기술하지 않는것이 더 좋다. 

\end{enumerate}
\end{exper}

\subsection{유효 숫자}



\marginpar{
  \begin{center}
  \includegraphics[width=5.5cm]{사진모음/그래프 나타내기.png}\captionof{figure}{그래프 나타내기}
  \end{center}}
  

\subsection{그래프로 나타내기}
      \marginpar{
      \begin{center}
      \includegraphics[width=5.5cm]{사진모음/여러개의 선 표기.png}\captionof{figure}{여러개의 데이터 표기}
      \end{center}} 
    \begin{enumerate}[가.]\setlength{\itemsep}{1mm}
     \item 적절한 제목을 선택하기 - 조작 변인, 종속 변인, 중요한 통제 변인을 포함하여 짧게 서술한다. 
     \item 변수에 맞는 적절한 축을 선택하기 - 조작 변인은 가로축에, 종속 변인은 세로축에 위치한다. 
     \item 축에 변수의 이름을 붙이고, 단위를 나타내기
     \item 축에 눈금 선택하기 - 눈금 간격을 균등하게 나눈다. 가로축과 세로축의 눈금 간격이 같을 필요는 없다. 눈금이 반드시 0에서 시작할 필요는 없다.
     \item 변수가 연속적일 때에는 데이터 값을 연결하는 적절한 선을 그릴 수 있다. 이 때의 이 선을 추세선이라고 한다. 이 때의 추세선은 일치하지 않는 점을 
     울퉁불퉁하게 그리지 말고, 그 점을 가능한 근접하게 지나가는 직선이나 곡선이 되도록 그려넣어야 한다. 또한 하나의 그래프에 여러개의 선을 그려야 할 경우
     에는 각 측정점을 $\circledcirc,\bigtriangleup, \diamond$와 같이 서로 다르게 표시하는 것이 좋다.
 
     \item 측정값의 불확실도(오차)를 알고 있는 경우에는 오차막대를 사용하여 그 크기를 나타낸다.
     
    \end{enumerate}
  
\marginpar{
  \begin{center}
  \includegraphics[width=5.5cm]{사진모음/오차막대.png}\captionof{figure}{오차막대}
  \end{center}} 




\end{flushleft}