
\pagestyle{hangul}
\chapter{수학적 도구들}\label{chap:math in physics}
\chapterprecis{부록에서는 물리학의 논리를 엄밀하게 펼치기 위한 수학에 대해서 설명한다.}
\ResetHangulspace{1.333}{1.2}
\paragraphfootnotes

\section{벡터의 성분과 표현}\label{sec:appsec}
\begin{flushleft}
온도와 같은 \textbf{스칼라}\footnote{스칼라 물리량의 대표적인 예는 온도, 압력, 에너지, 질량, 시간 등이 있으며, 
이들은 방향성이 없다.}는 크기만 있다. 이들은 단위를 가진 숫자로 나타내고, 일반적인 산술과 대수의 규칙을 따른다.
크기와 방향 둘 다 필요한 물리량을 기술하려면 새로운 수학언어인 \textbf{벡터}\footnote{벡터 물리량의 예로는 변위, 
속도, 가속도, 운동량 등이 있다.}가 필요하다. 예를들면 힘, 자기력, 전기장과 같은 물리량을 기술하는데 벡터는 필수적이다. 
가장 간단한 경우의 벡터량은 변위로서 위치의 변화가 변위\footnote{위치는 기준점에 따라 달라진다. 
하지만 변위벡터는 기준점과 관계없이 변화량만 다루기 때문에 기준이 필요치 않아 유용하다. 이러한 관계는 시각과 시간의 차이와 유사하다.}
이다. 이러한 \textbf{벡터는 크기와 방향을 갖는다.}
입자의 위치가 $A$에서 $B$로 변한다면 그림과 같이 변위벡터를
화살표로 나타낼 수 있다. 이 때 벡터는 문자의 위에 화살표($\vec{AB}$) 를 붙여 표현하며 화살표의 길이로 벡터의 크기를 표현한다.
중요한 사실 중 하나는 \textbf{벡터는 크기와 방향이 변하지 않는한 동일한 벡터라는 것}이다. 
벡터는 시작점의 정보를 갖지 않는다는 것은 오래도록 곱씹을 만한 내용이다.  

\begin{figure}[h]
     \centering
     \begin{subfigure}[h]{0.45\textwidth}
         \centering
         \includegraphics[width=\textwidth]{사진모음/벡터의 표현.jpg}
         \caption{벡터의 화살표 표현}
     \end{subfigure}
     \hfill
     \begin{subfigure}[h]{0.5\textwidth}
         \centering
         \includegraphics[width=\textwidth]{벡터의 동등.PNG}
         \caption{벡터의 동등}
      \end{subfigure}
        \caption{벡터의 표현과 동등}
\end{figure}
하지만, 이러한 표현은 기하학적인 표현만 할 수 있기 때문에 보다 정밀하게 좌표계를 이용한 성분표현을 할 수 있다면 좋을것 같다.
그럴려면 \textbf{단위벡터}를 도입해야한다. 단위벡터는 크기가 1이며 특정한 방향을 갖는 벡터이다. 단위벡터를 사용하는 목적은 
벡터의 방향을 나타내기 위한 것이다. 우리가 주로 다루는 직각좌표계(cartesian coordinate)에서 x,y,z축 양의 방향을 향하는 단위벡터를 각각 \
$\vu{i},\vu{j},\vu{k} $\footnote{다른 서적에서는 $\vu{x},\vu{y},\vu{z} $로 표기하기도 하며 단위벡터에는 
모자기호($\vu{\quad}$)를 붙여준다.}으로 표기한다. 단위벡터는 다른 벡터들을 표기할 때 유용하다. 예컨데 그림의 벡터 $\va{A}$
를 다음과 같이 나타낼 수 있다. 
\begin{figure}[h]
 \centering
  \includegraphics[width=5cm]{벡터의 성분표현.PNG}
  \caption{벡터의 성분 표현}\label{벡터의 성분표현}
\end{figure}

\begin{equation}
  \va{A} = \va{A_x}+\va{A_y} =A_x\vu{i}+A_y\vu{j} =4\vu{i}+3\vu{j}\footnote{($A_x,A_y$)=(4,3)로 표현하기도 하니 
  알아두자. 이 때 벡터의 크기는 화살표의 길이로 $\abs{\va{A}}=A=\sqrt{A_x^2+A_y^2}=5 $이다.  }
\end{equation}

위의 그림 \ref{벡터의 성분표현}에서 $A_x\vu{i}, A_y\vu{j}$는 벡터로서 벡터성분이라고 한다. 이 벡터의 성분은
직관적으로 알 수 있듯 각 좌표축에 벡터를 투영시킨것으로 x축과 벡터가 이루는 각도를 $\theta$라고 할 때, 
\begin{equation}
  A_x=A\cos\theta, A_y=A\sin\theta
\end{equation}

  로 표기할 수 있다. $A_x,A_y$는 스칼라로서 벡터 $\va{A}$의 스칼라성분이라 한다. 




  \begin{task}
공항을 떠난 비행기가 공항에서 설정한 직교좌표계의 양의 y축방향에서 30도만큼 양의 x축 방향으로 기운만큼 공항으로부터 200km를 날아갔다.
비행기의 변위벡터를 (cartesian coordinate)로 표현하고, 그 변위벡터의 단위방향벡터를 구하시오.   

\end{task}
\end{flushleft}







\begin{flushleft}
  


\subsection{벡터의 덧셈}
벡터는 벡터만의 연산규칙이 있다. 화살표를 이용한 기하학적인 벡터의 연산규칙과 성분을 이용한 벡터의 계산의 연산규칙은 동등하다. 
일반적으로 전문적이고 정량적인 분석을 하기위해서는 성분을 이용한 벡터의 계산에 익숙해질 필요가 있다. 
벡터의 성분에 대해서는 위의 section에서 배웠으니 벡터의 덧셈과 뺄셈은 직관적으로 이해하기 쉬울것이다. 
하지만 벡터의 곱셈은 스칼라곱과 벡터곱으로 나뉘어져 있다. 
example을 풀고 이 문서가 참조한 Halliday의 일반물리학 예제 연습하기를 추천한다. 

벡터의 덧셈에는 기학학적인 뎃셈과 성분을 이용한 덧셈이 있다.


\begin{figure}[h]
  \centering
  \begin{subfigure}[h]{0.4\textwidth}
      \centering
      \includegraphics[width=\textwidth]{벡터의 합.PNG}
      \caption{$\va{c}=\va{a}+\va{b}$}
      \label{그림A}
  \end{subfigure}
  \hfill
  \begin{subfigure}[h]{0.4\textwidth}
      \centering
      \includegraphics[width=\textwidth]{벡터의 합2.PNG}
      \caption{ $(\va{a}+\va{b})+\va{c}=\va{a}+(\va{b}+\va{c})$ }
      \label{그림B}
   \end{subfigure}
     \caption{벡터의 기하학적인 합}
  \end{figure}

 그림 \ref{그림A}은 $\va{a}, \va{b}$가 더해져 $\va{c}$가 되는것을 나타낸다. 더하고자하는 벡터의 머리와 꼬리를 이어 합을 
 만들수가 있다. 벡터의 기하학적인 합은 다음과 같은 벡터방정식으로 표기할 수 있다.
  \begin{equation}
  \va{c}=\va{a}+\va{b}  
\end{equation}
그림 \ref{그림A}을 보면 다음의 특성을 보인다. 덧셈순서와 무관하다. 
  \begin{equation}
\va{a}+\va{b}=\va{b}+\va{a}   
\end{equation}
\noindent 둘째, \ref{그림B}를 보면, 두 개 이상의 벡터들을 어떤 순서로도 더할 수 있다. 
\begin{equation}
  (\va{a}+\va{b})+\va{c}=\va{a}+(\va{b}+\va{c}) 
\end{equation}

벡터의 덧셈은 교환법칙과 결합법칙을 만족한다.



  벡터 성분으로 벡터 더하기는 이보다 직관적으로 이해하기 쉽다.


 \begin{align}
  & \va{c}= \va{a}+\va{b} \\
  & \va{a}= a_x\vu{i}+a_y\vu{j}+a_z\vu{k} \\
  & \va{b}= b_x\vu{i}+b_y\vu{j}+b_z\vu{k} 
   \end{align}

  일때, 
  \begin{align}
    &\va{c}= (a_x\vu{i}+a_y\vu{j}+a_z\vu{k}) + (b_x\vu{i}+b_y\vu{j}+b_z\vu{k}) \\
    &= (a_x+b_x)\vu{i}+(a_y+b_y)\vu{j}+(a_z+b_z)\vu{k}
    \end{align}
벡터의 더하기와 빼기는 같은 성분끼리 더하거나 빼주면 되어 간단하다. 
\newpage



  \begin{task}

1)변위 $\va{a}$와 $\va{b}$의 크기가 각각 3m와 4m이며, $\va{c}=\va{a}+\va{b}$이다. 
변위 $\va{a}$와 $\va{b}$의 가능한 방향을 모두 고려할 때, $\va{c}$의 최대크기와 최소크기를 구하시오.\\
\phantom{text}\\
\phantom{text}\\
\phantom{text}\\
\phantom{text}\\
2)다음에 제시된 벡터들의 합을 표시하시오.\\
\phantom{text}\\

    \centering{
    {\IfFileExists{사진모음/벡터의 합성.png}
    {\includegraphics[width=0.6\linewidth]{사진모음/벡터의 합성.png}
  \label{fig:벡터화살표 합성}}%
    {\rule{\linewidth}{4cm}}}
    }

    \phantom{text}\\
    \phantom{text}\\
    \phantom{text}\\

\begin{flushleft}
3)벡터 $\va{A}= 2\vu{i}-3\vu{j}+6\vu{k}$와 $\va{B}= \vu{i}+2\vu{j}-3\vu{k}$ 가 있다. 다음을 구하라.
    \\
    \begin{tasks}[label=(\arabic*)](1)
      \task $A+B$ \\
      \task $\abs{\va{A}+\va{B}}$\\
      \task $2\vec{A}-3\vec{B}$  \\
   \end{tasks}
  \end{flushleft}

  \end{task}

\subsection{벡터의 곱셈}

벡터 곱하기는 세 가지 방법이 있으며 보통의 대수 곱하기와는 전혀 다르다. 그 세가지는 
\begin{itemize}
\item 벡터에 스칼라 곱하기
\item 스칼라곱 
\item 벡터곱 
\end{itemize}
으로 나눌 수 있다. 

벡터에 스칼라 곱하기는 간단하다. 
벡터 $\va{a}$ 스칼라 s를 곱하는 것은 벡터 $\va{a}$의 크기에 s의 절대값을 곱한값으로 $s\va{a}$로 나타내며 방향은 
s의 부호에 따라 s가 양이면 같은 방향 음이면 반대방향을 가리키게 된다. 

\begin{figure}[h]
  \centering
  \includegraphics[width=0.5\textwidth]{벡터의 실수배.PNG}
  \caption{벡터의 실수배}
\end{figure}

\subsubsection{스칼라곱}
두 벡터 $\va{a}, \va{b}$의 스칼라곱은 $\va{a}\vdot\va{b}$로 표기하고 다음과 같이 정의한다. 
\begin{equation}
  \va{a}\vdot\va{b}=ab\cos\theta
\end{equation}
이 때 $\theta$는 벡터 $\va{a}$의 방향과 $\va{b}$의 방향 사이의 각도이다. \textbf{우변은 스칼라값들뿐이므로 이를 벡터의 스칼라곱
점곱이라고 한다.} 성분을 강조하면 다음과 같이 표기할 수 있다. 
\begin{equation}
  \begin{split}
    \va{a}\vdot\va{b}&= (a_x\vu{i}+a_y\vu{j}+a_z\vu{k}) \vdot (b_x\vu{i}+b_y\vu{j}+b_z\vu{k}) 
  \\& = (a_x b_x)+(a_y b_y)+ (a_z b_z)
  \end{split}
\end{equation}

\begin{task}
두 벡터 $\va{a}=3.0\vu{i}-4.0\vu{j}$와 $\va{b}=2.0\vu{i}-3.0\vu{k}$의 사이각 $\theta$의 cos값을 구하시오.

\end{task}




\subsubsection{벡터곱} 
  두 벡터 $\va{a}, \va{b}$의 벡터곱은 $\va{a}\cp\va{b}$로 표기하고 그 크기는 다음과 같이 정의한다. 
\begin{equation}
  \abs{\va{a}\cp\va{b}}=ab\sin\theta
\end{equation}
이 때 $\theta$는 벡터 $\va{a}$의 방향과 $\va{b}$의 방향 사이의 작은 각도이다. 이 때 스칼라곱과 큰 차이점이 있다.
\textbf{우변은 벡터값이므로 방향이 있다. 이 방향은 $\va{a},\va{b}$
가 이루는 평면에 수직이며 구체적으로는 그림                                                                                                                                                                                                                                                                                                                                                                                                                                                                                                                                                                                                                                                    \ref{오른손규칙}의 오른손 규칙을 따른다.\footnote{오른손 법칙은 "벡터곱을 외자" 활동을 통해 익혀보자.}}
 따라서 벡터곱은 두 벡터의 순서가 중요하다. 
\marginpar{
  \begin{center}
  \includegraphics[width=4cm]{오른손법칙.png}\captionof{figure}{오른손 규칙}\label{오른손규칙}
  \end{center}} 

\begin{equation}
  (\va{a}\cp\va{b})=  -(\va{b}\cp\va{a})
\end{equation}

따라서 벡터곱에서는 교환법칙이 성립하지 않는다.
성분을 이용하려면 단위벡터간의 벡터곱을 알아야한다. 

\begin{equation}
  \begin{split}
    \va{i}\cp\va{j}=\va{k}=-\va{j}\cp\va{i}
  \end{split}
\end{equation}

\marginpar{
  \begin{center}
  \includegraphics[width=4cm]{단위벡터의 외적.png}\captionof{figure}{단위벡터의 외적}
  \end{center}} 


이를 알고 두 벡터의 벡터곱을 성분별로 표현하면 아래와 같다.
\begin{equation}
  \begin{split}
    \va{a}\cp\va{b}&= (a_y b_z-b_y a_z)\vu{i}+(a_z b_x - b_z a_x)\vu{j}+(a_x
    b_y-b_x a_y)\vu{k} 
  \end{split}
\end{equation}

백터곱은 비교적 복잡하므로 행렬식을 이용한 방법을 익히는 것을 추천한다. 



\begin{task}
물리학에서 지레의 회전중심에서 회전팔에 힘이 작용하는 지점까지의 크기와 방향을 $\va{r}$, 회전팔에 작용하는 힘의 
크기와 방향을 $\va{F}$로 나타낼 때, 지레가 받는 돌림힘을 다음과 같이 정의한다. 
\begin{equation}
  \vec{\tau}= \va{r} \cross \va{F} [N\cdot m]
\end{equation} 

지레의 회전팔이 y-z 평면에서 회전가능하며 $\va{r} =4\vu{j}+3\vu{k},  \va{F} =-5\vu{i} -3\vu{j}$ 일 때 지레가 받는
돌림힘의 벡터를 구하시오. 
\end{task}
\end{flushleft}

\marginpar{
  \begin{center}
  \includegraphics[width=5cm]{사진모음/돌림힘_1.jpg}\captionof{figure}{돌림힘}\label{돌림힘}
  \end{center}} 
