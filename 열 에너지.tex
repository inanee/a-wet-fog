\begin{flushleft}
    \section{에너지-열현상}
    물이 찬 주전자를 가스레인지에 가열한다. 그러면 수증기가 생겨 주전자 뚜껑을 움직이고 요란한 소리까지 만든다.
    이 과정을 곱씹으면 우리가 주위에서 느끼는 열(heat)현상은 역학적 현상과 연관되어 보인다. 
    그렇다면 엄밀하게 그 양들을 따질 수 있어야 한다.
    이번 Section에서는 열현상을 분석할 것이다. 그러기 위해서 우리는 주변을 보는 시각을 원자적으로 바꾸어야만 한다. 
    왜냐하면 우리가 배운 뉴턴의 역학을 이용하여 유체, 기체 등을 입자로 보고 그 입자의 역학적 에너지를 확장하여 
    정의할 것이기 때문이다. 이렇듯 물리학은 연역적이다.  열역학을 시작하기 이전에 열역학에서 사용하는 물리량을 
    정의하고 가는 것은 유용하고 필수적이다. \textbf{부피, 밀도, 압력}으로 이 세 물리량은 모두 스칼라량이다. 부피는
    직관적으로 알고 있는 물체가 공간에 차지하는 양으로 생각하면 된다. 밀도와 압력은 다음과 같다.

      \begin{defn}[밀도($\rho$)]
       \begin{equation}
        \rho=\frac{\Delta m}{\Delta V} =\frac{m}{V} (\phantom{o} in\phantom{o}const \phantom{o}density)
      \end{equation}
      \centering 
      \begin{flushleft}
      \textbf{밀도 $\rho$는 그 점을 포함하는 작은 부피안에 질량이 무거울수록 크다.}
      \end{flushleft}
      \end{defn}


      \begin{defn}[압력($p$)]
       \begin{equation}
        p=\frac{\Delta F}{\Delta A} =\frac{F}{A} (\phantom{o} in\phantom{o}equal \phantom{o}force \phantom{o}to \phantom{0}Area)
      \end{equation}
      \centering 
      \begin{flushleft}
      \textbf{압력 $p$는 점을 중심으로 하는 면의 면적 $\Delta A$가 무한히 작아지는 극한값이다. 면에 대해 힘이 일정한 경우 $\frac{F}{A}$로 쓸 수 있다.}
      \end{flushleft}
      \end{defn}

% %  \begin{defn}[압력($p$)]
%          \begin{equation}
%          \p=\frac{\Delta F}{\Delta A}=\frac{F}{A} (\phantom{o} in\phantom{o}equal \phantom{o}force\phantom{o} to\phantom{o} Area)
%       \end{equation}
%       \centering 
%       \begin{flushleft}
%       \textbf{압력 $\p$는 점을 중심으로 하는 면의 면적 $\Delta A$가 무한히 작아지는 극한값이다. 면에 대해 힘이 
%       일정한 경우 $\frac{F}{A}$로 쓸 수 있다.}
%       \end{flushleft}
% %       \end{defn}



     \subsection{열에너지와 온도}
     여러 서적의 텍스트들은 온도를 '차고 뜨거운 정도'라는 개념으로 표현한다. 하지만 이는 정량적으로 따질 수 있는
     것이 아니고 측정자의 주관적인 해석도 들어갈 여지가 크다. 따라서 온도를 정량적인 에너지와 연관짓고 해석할
     필요가 있다. 이에 첫째로 열 에너지를 먼저 정의해야 한다.
     우리가 명확하고 정량적으로 열현상을 분석하고자 한다면 에너지 보존에서 학습한 열에너지를 사용하자.
    \textbf{열 에너지란 물질 안에 존재하는 입자 운동 에너지와 결합 에너지의 총합을 의미한다.\footnote{이러한 에너지를 하나도 갖고 있지 
    않은 입자를 상상할 수 있다. 입자가 이러한 상태일 때의 온도를 절대온도 '0K'로 정의한다. 절대온도의 단위는 K이다. 일상생활에서는
    섭씨나 화씨를 주로 사용한다.}} 
    이러한 에너지는 하나의 계(system)에서 보존되어진다. 따뜻한 공기를 이루는 입자의 운동에너지가 차가운 공기를 이루는 입자의 운동에너지보다
     더 크다. 두 공기들이 만나 열적인 평형을 이루게 된다면 이는 서로 충돌에 의해 에너지가 오간 것으로 생각할 수 있다.  
     \marginpar{
      \begin{center}
        \includegraphics[width=4cm]{사진모음/열현상/열에너지-입자의 운동에너지.jpg}
        \captionof{figure}{입자의 운동-열에너지}
      \end{center}}
    
    \subsubsection*{이상기체}
우리는 온도와 같은 입자의 통계적이고 거시적인 거동을 기체를 이루고 있는 분자들의 운동으로 설명할 것이다. 그렇다면 우리는 수많은 종류의 분자들을 대표할
공통적인 모델을 제시할 수 있어야 한다. 이에 상황을 간단히 하기 위해 \textbf{이상기체}라는 것을 정의하고 이를 연역적으로 확장하고 활용한다. 이상기체는 
기체의 운동에너지만을 내부에너지로 따지기 위한 이상기체로 충분히 낮은 밀도에서 기체들은 이상기체와 같이 행동한다. 
\textbf{즉, 기체분자들이 서로 충분히 멀리 떨어져 있어서 상호작용을 할 수 없는 조건에서는 이상적인 상태로 접근해간다.} 
이렇게 낮은 밀도에서 기체들은 다음의 이상기체 상태방정식을 따른다. 

    \begin{defn}[이상기체 상태방정식]

    \begin{equation}\label{eq:ideal gas}
    pV=NkT
    \end{equation}

  \begin{itemize}
    \item  $p$: 기체의 압력
    \item k: 볼츠만(Boltzmann) 상수($=1.38\cross 10^{-23} $J/K)
    \item $V$: 기체의 부피
    \item $T$: 기체의 온도
  \end{itemize}  
  \end{defn}

 



    \subsubsection*{이상기체의 내부에너지}
지금부터는 정량적으로 이상기체를 질점의 물체로 보고 이상기체 입자가 만드는 압력을 계산하여 보고, 더 나아가 온도와 내부에너지의 
관계를 알아보고자 한다. 질량이 $m$이고, 속도가 $\va{v}$인 한 분자가 벽면에 충돌하는 경우를 생각해보자. 충돌은 탄성충돌로 
운동에너지가 보존되고, 상황을 간단히 하기 위하여 x축방향의 속도만 있고 운동량 변화도 해당 방향으로만 있다고 가정한다. 

입자의 선운동량 변화량은
\begin{equation}
  \Delta p_x=(-mv_x)-(mv_x)=-2mv_x
\end{equation}

   \marginpar{
      \begin{center}
        \includegraphics[width=5.0cm]{사진모음/열현상/test01.eps}
        \captionof{figure}{기체분자가 벽면에 작용하는 힘}
      \end{center}}


이다. 그림의 분자는 색칠한 벽을 계속해서 반복적으로 때릴 것이다. 반복되는 충돌 사이의 시간간격 $\Delta t$는 속력 $v_x$의 분자가
반대쪽 벽까지 갔다가 다시 돌아오는데 걸리는 시간이다. 따라서
   \begin{equation}
  \Delta t=\frac{2L}{v_x}
\end{equation}
이에 따라 분자 하나가 색칠한 벽에 전달하는 평균 단위시간당 선운동량은 다음과 같다.
 \begin{equation}
  \frac{\Delta p_x}{\Delta t}=\frac{2mv_x}{2L/v_x}=\frac{mv_x^2}{L}
\end{equation}
이는 뉴턴의 제2법칙($F=\frac{dp}{dt}$)에 의해 입자하나가 벽에 작용하는 힘이다. 전체 힘은

\begin{align}
  &F_{x}=\frac{mv_{x1}^2}{L}+\frac{mv_{x2}^2}{L}+\frac{mv_{x2}^2}{L}+\cdots\frac{mv_{xN}^2}{L}\\
  &=\frac{m}{L^2}(v^2_{x1}+v^2_{x2}+v^2_{x3}\cdots+v^2_{xN}) 
\end{align}

이 때 모든 입자의 속력 제곱 평균인 $(v^2_x)_{avg}$을 이용하여 단위면적당 힘을 정리하여 다음과 같이 수정할 수 있다.


 \begin{itemize}
    \item  $N= nN_A$: $N$은 전체 입자의 갯수, $n$은 몰수, $N_A$는 아보가드로 수
    \item  $p$: 열역학에서 문자 $p$는 운동량이 아닌 압력(pressure)으로 사용한다.
    \item  $M=mN_A$: $m$은 이상기체 입자 하나의 질량이다. 따라서 $M$은 1mol의 질량
    \item  $V=L^3$: V(volume)은 부피이다.
  \end{itemize} 

\begin{align}
 &p=\frac{nmN_A}{L^3}(v^2_x)_{avg}\\
 & mN_A =M, \phantom{o} L^3 =V \textrm{(Volume)}\\
 & \therefore p=\frac{nM(v^2_x)_{avg}}{V}
\end{align}
이를 좀더 일반적인 상황인 3차원에 확장한다면 분자의 속도의 크기는 $v^2=v^2_{x}+v^2_{y}+v^2{z}$이다. 또한 많은 분자들에 의해
통계적으로 각 분자의 각 속도성분의 제곱의 평균값은 모두 같다라고 볼 수 있어, $v^2_x=\frac{1}{3}v^2$이다.\footnote{x,y,z축에 대해서
특정방향의 평균값이 크거나 작은 것이 더 평범하지 않다.} 그리고 $(v^2)_{avg}$
의 제곱근은 제곱평균제곱근(rms)속력이라고 부르고 $v_{rms}$로 표기한다. 이에 따라 최종적으로 벽에 작용하는 압력은 다음과 
같이 쓸 수 있다. 

\begin{equation}
  p=\frac{nMv^2_{rms}}{3V}
\end{equation}

이 식으로 인하여 우리는 기체의 압력이 아주 작은 분자들의 속력에 어떻게 좌우되는지 정량적으로 구하고 연결하였다. 
이를 이상기체 상태방정식($pV=nRT$)과 연결하면 rms속력을 구할 수 있고 이에 따라 우리는 기체분자의 
병진운동에너지를 구할 수 있다. 이 병진운동에너지는 각 입자간의 상호작용이 없는 이상기체의 내부에너지가 된다.

  \begin{defn}[이상기체의 rms 속도, 병진운동에너지, 내부에너지]

    \begin{align}\label{eq:ideal gas's energy}
    &v_{rms}=\sqrt{\frac{3RT}{M}}, \phantom{o} K_{avg}=\frac{1}{2}mv^2_{rms}\\
    &\therefore K_{avg}=\frac{3RT}{2N_A}=\frac{3}{2}kT
    \end{align}

  \begin{itemize}
    \item  $nR=Nk$ : $n$은 몰수, $R$은 기체상수이다.
  \end{itemize}  
  \end{defn}



     \subsection{기체가 한 일}
     계에 대한 설명 

     \subsection{열역학 제1법칙}
         \subsubsection*{등적과정}
        \subsubsection*{등압과정}
        \subsubsection*{등온과정}
        \subsubsection*{단열과정}
     \subsection{열역학 제2법칙}
     \subsection{열효율}



\end{flushleft}